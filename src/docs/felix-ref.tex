\documentclass[oneside]{book}
\usepackage{xcolor}
\definecolor{bg}{rgb}{0.95,0.95,0.95}
\definecolor{emphcolor}{rgb}{0.5,0.0,0.0}
\newcommand{\empha}{\bf\color{emphcolor}}
\usepackage[framemethod=TikZ]{mdframed}
\newtheorem{example}{Example}
\mdfdefinestyle{MyFrame}{innerleftmargin=20pt}

\definecolor{warningbg}{rgb}{0.95,0.90,0.90}
\newenvironment{warning}%
  {\begin{mdframed}[style=MyFrame,backgroundcolor=warningbg]}%
  {\end{mdframed}}
\definecolor{explanationbg}{rgb}{0.90,0.95,0.95}
\newenvironment{explanation}%
  {\begin{mdframed}[style=MyFrame,backgroundcolor=explanationbg]}%
  {\end{mdframed}}

\usepackage{parskip}
\usepackage{minted}
\usepackage{caption}
\usepackage{amsmath}
\usepackage{amssymb}
\usepackage{amscd}
\usepackage{makeidx}
\usepackage[toc]{glossaries}
\makeindex
\usemintedstyle{friendly}
\setminted{bgcolor=bg,xleftmargin=20pt}
\usepackage{hyperref}
\hypersetup{pdftex,colorlinks=true,allcolors=blue}
\usepackage{hypcap}
\newcommand*{\fullref}[1]{\hyperref[{#1}]{\autoref*{#1} \nameref*{#1}}}
\title{The FARM\\ Felix Annotated Reference Manual}
\author{John Skaller}
\makeglossaries

\newglossaryentry{coroutine}{name=coroutine, description={a programming unit 
which exchanges control with others on a neutral basis}}

\newglossaryentry{polymorphism}{name=polymorphism, description={a language
feature which allows a function to process arguments of
more than one type, and to behave differently depending
on the argument type}}

\newglossaryentry{pthread}{name=pthread, description={a programming unit 
which runs concurrently with others and which may be suspended and resumed
by the operating system at any time}}

\newglossaryentry{setwise lookup}{name=setwise lookup, description={
a method of lookup where a symbol may be used before or after it is
defined
}}

\begin{document}
\maketitle
\tableofcontents
\chapter{Introduction}
This reference is a guide to the Felix programming language.
It is not the usual reference because Felix differs from
other systems in that most of the grammar is part of the library,
in user space. In principle then, separating the library from the 
core language is impossible: if anything the core language is
defined by the compiler intermediate abstract machine, details
of little interest to most programmers.

Furthermore even that characterisation is weak, because considerable
functionality is actually embodied in the run time library. For example
the compiler knows what a service request is, but it has no idea
what an fthread is. It knows what a generator is, and it knows which
generators perform yields, but it has no idea what an iterator is,
despite the fact these are effectively a core language feature.

Therefore, our presentation cannot be complete, it cannot be precise,
and it cannot replace actually reading the library code. Felix is a highly
mutable language, major new features can often be introduced without
touching the compiler.

For example a complete set of primitive types with their base
operations cannot be presented, because, with a couple of exceptions
there aren't any such types. Instead, most primitive types are introduced
without knowledge of the compiler, by creating bindings to C++ in 
library code; these bindings defined the type name and some operations
on the types in terms of C++.

% =====================================
\part{Scripting}
Felix is primarily a scripting language, meaning, it is intended
for writing a program in one file and immediately running it.
It's hard to define what this means but the lack of separate 
interfaces  is one of the things that makes writing script simpler.

Although out of logical order, we present the basic statements,
expressions, and definitions first, because together with examples
that is enough to get most programmers writing script.

Felix code, like most other languages, is compiled before
execution. However it differs from many other systems which
compile to byte code and interpret it: Felix uses C++
as its bytecode, and compiles it to machine binary and
executes that. Compilation times are slower, but the
run time performace is better.

\chapter{Language Features}
\begin{itemize}
\item higher order functions, procedures, and coroutines
\item parametric polymorphism
\item type constraints
\item typeclass based \gls{polymorphism}
\item overloading
\item generics
\item row polymorphism
\item tuple, record and struct product types
\item sum, variant and union sum types
\item rich pattern matching
\item regular expressions
\item java like object sublanguage
\item circuit based \glspl{coroutine}
\item platform independent asynchronous socket I/O
\item platform independent pre-emptive threading
\item direct embedding of C and C++
\item can be embedded in framework idle loops
\item generates fast binary programs and libraries
\item run scripts directly without makefiles
\end{itemize}

\chapter{The flavour of Felix}
Although this is a reference manual, it seems wise to show
the flavour of the scripting subsystem with examples, and
what could be a better starting point than:

\begin{minted}{felix}
println "Hello World";
\end{minted}

\section{Simple calculations}
To perform a simple numerical calculation using a function:

\begin{minted}{felix}
fun hypot (x:double, y:double) => sqrt (x^2 + y^2);
var x = 4.0;
var y = 5.0;
var d = hypot (x,y);

println$ 
  "Hypotenuse of triangle base " + x.str + 
  ", side " + y.str " is " + 
  d.str
;
\end{minted}

A better looking way:

\begin{minted}[escapeinside=||]{text}
fun hypot (x:double, y:double) => |$\sqrt {x^2 + y^2}$|;
\end{minted}

encoded like this:

\begin{minted}{felix}
fun hypot (x:double, y:double) => \(\sqrt \(x^2 + y^2\)\);
\end{minted}

\section{Yielding Generators}\index{yield}\index{generator}
To calculate the number of rabbits using a yielding generator:

\begin{minted}{felix}
gen rabbits () = {
  var generation = 1;
  var last_gen = 1;
  var this_gen = 2;
  while true do
    yield (generation, this_gen);
    last_gen <-> this_gen;
    this_gen += last_gen;
  done
}

var fibonacci = rabbits;

for i in 1..10 do
  var bunnies = fibonacci(); 
  println$ bunnies;
done
\end{minted}

\section{Fibres}
To parse text into words, using a coroutine pipeline.
First a producer of characters:

\begin{minted}{felix}
proc char_stream (s:string) (out:%>char) {
  for c in s perform write (out,c);
  write (out,char "");
}
\end{minted}

and now a transducer that converts a character
stream into a word stream:

\begin{minted}{felix}
proc words (inp:%<char, out:%>string) {
  while true do
    var s = "";
    nextch: while true do
      var c = read inp;
      if isalnum c do
        s+=c;
      else 
        if s != "" call write (out,s);
        break nextch;
      done
    done // nextch
  done
}
\end{minted}

Finally we need a sink to emit the result:

\begin{minted}{felix}
proc emit (inp:%<string) {
  while true do
    var word = read inp;
    println$ "Word = " + word;
  done
}
\end{minted}

which you can use like this:

\begin{minted}{felix}
var text = 
  "The quick, brown fox jumped "
  "over the lazy dog"
;

#(char_stream (text) |-> words |-> emit);
\end{minted}

\section{Circuits}

Circuits generalise the capabilities of pipelines.
Two data sources:

\begin{minted}{felix}
open syntax chips;

chip w10 
  connector pins
     pin outpin: %>int 
{ 
  for i in 0..10 do 
    write (pins.outpin,i);
  done
}

chip w100
  connector pins pin outpin: %>int 
{ 
  for i in 100..110 do 
    write (pins.outpin,i);
  done
}
\end{minted}

and a data sink:

\begin{minted}{felix}
chip conout 
  connector pins pin inpin: %<int 
{ 
  while true do
    var x = read pins.inpin;
    println$ "Read " + x.str;
  done
}
\end{minted}

with a transducer:

\begin{minted}{felix}
chip doubler
  connector pins 
    pin inpin: %<int
    pin outpin: %>int
{
  while true do
     var x = read pins.inpin;
     write (pins.outpin, 2 * x);
  done
}
\end{minted}

can be combined like this:

\begin{minted}{felix}
/*
  w10 ---\
         \---> doubler --> conout
  w100---/
*/

chip fred 
  connector pins
    pin inch:%<int
{
  circuit
    connect w10.outpin to doubler.inpin
    connect w100.outpin to doubler.inpin
    wire pins.inch to doubler.outpin
  endcircuit
}
\end{minted}

to form a circuuit we can use like this:

\begin{minted}{felix}
var dummy_channel = mk_schannel[int](); 

circuit
  wire dummy_channel to fred.inch
endcircuit

while true do
  var x = read dummy_channel;
  println$ "Read " + x.str;
done
\end{minted}
 

\chapter{Variable Definitions}\index{variable}
\href{http://felix-lang.org/share/lib/grammar/variables.fsyn}{Syntax}

A {\em definition} is a statement, clause, expression, phrase
or other term which defines a name, but does
not cause any observable behavior, or, a class statement, or, 
a var or val statement. The latter two exceptions define a name
but may also have associated behaviour.

Felix provides two primary kinds of variables:

\begin{enumerate}
\item a \verb%var% variable names mutable store, and.
\item a \verb%val% variable names a functional expression
\end{enumerate}

\section{The {\tt var} binder}
\subsection{The {\tt var} statement}
The \verb%var% statement is used to introduce a variable name
and potential executable behaviour. It has one of three 
basic forms:

\begin{minted}{felix}
var x : int = 1;
var y : int;
var z = 1;
\end{minted}

\begin{enumerate}
\item The first form specifies the type and an initialising
expression which must be of the specified type or a subtype thereof.

\item The second form specifies a variable of the given type
without an explicit initialiser, however the variable
will be initialised anyhow with the default contructor
for the underlying C++ type, although that constructor
may be trivial.

\item The third form does not specify the type, it will be deduced
from the initialiser.
\end{enumerate}

\subsection{Semantic rules}
\begin{enumerate}
\item If the initialiser has observable behaviour it will be observed
if at all, when control passes through the variable statement.

\item If the variable introduced by the \verb%var% statement is not used,
the variable and its initaliser will be elided and any observable
behaviour will be lost.

To be used means to have its address taken in a used expression,
to occur in a used expression. A used expression is one which
initialises a used variable, or, is an argument to function
or generator in a used expression, or an argument to a procedure
through which control passes. 

In other words, the variable is used if the behaviour of
the program appears to depend on its value or its address.

\item The library procedure \verb%C_hack::ignore% ensures the compiler
believes a variable is used:

\begin{minted}{felix}
var x = expr;
C_hack::ignore x;
\end{minted}

so that any side effects of \verb%expr% will be seen.

\item In general the argument to any primitive function, generator
or procedure will be considered used if its containing 
entity is also considered used. In general this means there
is a possible execution path from a root procedure of the
program.
\end{enumerate}

\subsection{Multiple variables}
Multiple variables can be defined at once:


\begin{minted}{felix}
var m = 1,2;
var a,b = 1,2;
var c,d = m;
\end{minted}

With this syntax, no type annotation may be given. The RHS of the statement
must be an expression of tuple type, with the number of components of the
tuple equal to the number of variables being defined.

\subsection{Var expressions}
Variables can also be defined inside expressions:

\begin{minted}{felix}
var x = 1 + (2 as var y) + (var z = y+1);
println$ x,y,z;
\end{minted}

The \verb%as var% form has low precedence and usually requires
parenthesis, in the assignment form parenthesis are mandatory.

Nested \verb%var% definitions are evaluated before their containing expression,
and for several expressions directly contained within the same expression,
in left to right order. The order is therefore deterministic.

Unnamed one use temporary variables can also be specified in
a \verb%var% subexpression:

\begin{minted}{felix}
var x = 1 + (var rand());
println$ x;
\end{minted}

The effect is to enforce eager, single evaluation of the subexpression
prior to its containing expression.


\subsection{Addressing}
A variable may have its address taken:

\begin{minted}{felix}
var x = 1;
var rwpx = &x; // read-write pointer
var rpx = &<x; // read-only pointer
var wpx = &>x; // write-only pointer
\end{minted}

A variable may be assigned a value directly by an assignment
statement:

\begin{minted}{felix}
x = 2;
\end{minted}

or indirectly via a a call to the system intrinsic \verb%_storeat% procedure:

\begin{minted}{felix}
_storeat(rwpx,3);
_storeat(wpx,4);
\end{minted}

If a variable stored no previous value and assignment sets the first value.
Otherwise, if a variable previously stored a value, an assignment
replaces it.

The standard grammar also provides the infix operator \verb%<-% which invokes
the standard library procedure \verb%storeat% which by default 
invokes the procedure \verb%_storeat%:

\begin{minted}{felix}
storeat(rwpx,3);
wpx <- 4;
\end{minted}

The standard library procedure \verb%storeat% can be overloaded, which also
changes the behaviour of the infix operator \verb%<-%.

\subsection{Referencing}
If a value is stored in a variable it can be referred to in an expression by using
the name of the variable.

\begin{minted}{felix}
var x = 1;
println$ x; // prints 1
\end{minted}

If a read-write or read only pointer to a variable is obtained, the
stored value, if it exists, can be referred to by using the system intrinsic 
\verb%_deref% function:

\begin{minted}{felix}
println$ _deref &x; // prints 1
println$ _deref &<x; // prints 1
\end{minted}

The standard grammar also provides the prefix operator \verb%*% which 
invokes the standard library function \verb%deref% which by default invokes
the system intrinsic \verb%_deref%:

\begin{minted}{felix}
println$ deref &x; // prints 1
println$ *(&x); // prints 1
\end{minted}

The standard library function \verb%deref% can be overloaded, which also
changes the behaviour of the prefix operator \verb%*%.

\subsection{Object Concept}

A variable is said to name an object\index{object!variable as}, not a value.
This basically means it is associated with the address of a typed
storage location.

\subsection{Hoisting and lookup}
A variable can be assigned to, addressed, or otherwise used
before the \verb%var% statement or expression which introduces it.
This is because Felix does set-wise, not linear, lookup.

An initialising definition such as

\begin{minted}{felix}
var x = 1;
\end{minted}

is split into an uninitialised variable definition
and an assignment, and the definition is then hoisted
to the top of the scope. For example:

\begin{minted}{felix}
x = 1;
var x = 2;
\end{minted}

is converted to:

\begin{minted}{felix}
var x: int;
x = 1;
x = 2;
\end{minted}

Javascript behaves the same way, and the phenomena is called {\em hoisting}.

\begin{warning}
Care should be taken because it is possible to refer to an uninitialised variable,
a variable unexpectedly initialised by the underlying C++ default constructor,
or the wrong variable entirely. 

For example here:

\begin{minted}{felix}
var x = 1;     //1
begin          //2
  var y = x;   //3
  var x = 2;   //4
end            //5
\end{minted}

the variable y is set to an indeterminate value because its initialiser
x is the one which is subsequently set to 2 on line 4, not the one set 
to 1 on line 1. This also means this code is legitimate:

\begin{minted}{felix}
x = 1;
x = 2;
var x = 3;
\end{minted}

because $x$ is in fact defined as a variable on the third line.
\end{warning}

\subsection{{\tt var} parameters}
A function or procedure may have a parameter marked \verb%var%:

\begin{minted}{felix}
fun f(var x: int) => 1; //1
var g = f;              //2
println$ f (rand());    //3
println$ g (rand());    //4
\end{minted}

An argument bound to a \verb%var% parameter will always be evaluated,
if at all, before the main body of the function is entered; that is,
it ensure eager evaluation.

A var parameter in a direct call or application may or may not
be eliminated, unlike a local variable which will always be eliminated
if not used.

\begin{explanation}
If the function or procedure is inlined, it is likely but not assured
that the parameter and its initialising argument will be eliminated,
hence in the example the evaluation of \verb%rand()% on line 3
could be eliminated because the parameter $x$ is not ised in $f$.

On the other hand it is very unlikely that the argument to $g$ will
be eliminated, because parameters of a closure cannot be, without
changing the closure's type. Therefore the \verb%rand()% on line 4
will almost certainly be evaluated.
\end{explanation}

%%%%%%%%%%%%%%%%%%%%%%%%%%%%%%%%%%%%%%%%%%%%
%%% VAL 
%%%%%%%%%%%%%%%%%%%%%%%%%%%%%%%%%%%%%%%%%%%%

\section{The {\tt val} binder}
\subsection{The {\tt val} statement}
A \verb%val% statement defines a name for an expression.

\begin{minted}{felix}
val x : int = 1;
val z = 1;
\end{minted}

The value associated with a \verb%val% symbol may be computed
at any time between its definition and its use, and may
differ between uses, if the initialising expression depends
on variable state, such as a variable or call to a generator.

It is not an error to create such a dependence since either
the value may, in fact, not change, or the change may
not be significant.

Nevertheless the user must be warned to take care
with the indeterminate evaluation time and use
a \verb%var% when there is any doubt.

Since a \verb%val% simply names an expression, it is associated
with a value not an object and cannot be addressed
or assigned to. 

\begin{warning}
However this does NOT mean its value cannot
change:

\begin{minted}{felix}
for var i in 0 upto 9 do
  val x = i;
  println$ x;
done
\end{minted}

In this example, x isn't mutable but it does take on
all the values 0 to 9 in succession. This is just a 
most obvious case: a less obvious one:

\begin{minted}{felix}
var i = 0;
val x = i;
println$ x;
++i;
println$ x;
\end{minted}

which is clearly just an expansion of the the first two
iterations of the previously given for loop. However in
this case there is no assurance \verb%x% will change after \verb%i%
is incremented because the compiler is free to replace
any \verb%val% definition with a \verb%var% definition.
\end{warning}

\subsection{Multiple values}
Multiple values can be defined at once:

\begin{minted}{felix}
val m = 1,2;
val a,b = 1,2;
val c,d = m;
\end{minted}

With this syntax, no type annotation may be given.

\subsection{Val expressions}
Values can also be defined inside expressions:

\begin{minted}{felix}
val x = 1 + (2 as val y) + (val z = y+1);
println$ x,y,z;
\end{minted}

The \verb%as val% form has low precedence and usually requires
parenthesis, in the assignment form parenthesis are mandatory.

\subsection{Pattern variables}
In a pattern match, pattern variables are implicitly \verb%val% variables
with their scope limited to the branch for which they're specified:
For example the statement:

\begin{minted}{felix}
match 1,2 with
| (x,y) => println$ x +y;
endmatch;
\end{minted}

the variables $x$ and $y$ act as if:

\begin{minted}{felix}
begin
  val x = 1;
  val y =2 ;
  println$ x + y;
end
\end{minted}

had been written. 


\subsection{{\tt let} expressions}
A \verb%let% expresion can be used to introduce a named value within
an expression which has scope limited to the let expression:

\begin{minted}{felix}
var y = 42;
var x = let y = 1 in y + y;
println$ x,y; // 2,42
\end{minted}

\begin{explanation}
In fact, in a \verb%let% expression, the phrase after \verb%let%
and before %=% can be an arbitrary expression because the \verb%let% expression:

\begin{minted}{felix}
let pattern = expr1 in expr2
\end{minted}

is syntactic sugar for

\begin{minted}{felix}
match expr1 with
| pattern => expr2
endmatch
\end{minted}
\end{explanation}

\subsection{{\tt val} parameters}
A function or procedure may have a parameter marked \verb%var%:

\begin{minted}{felix}
fun f(val x: int, y: int) => x + y;
\end{minted}

If a parameter is not explicitly marked \verb%var%, then it is
a \verb%val% parameter. The explicit spelling of \verb%val% is allowed
but not required.

\subsection{{\tt val} semantics}
A name introduced by a \verb%val% or \verb%let% binder is assumed to
bind to a purely functional expression. This means the expression
depends only on explicitly visible terms of the expression, has
no side-effects, and, has a value which is temporaly invariant or
{\em referentially transparent}. 

\begin{explanation}
The compiler may treat the val precisely as a \verb%var% and allocate
store to hold the value its initialiser designates. Alternatvely,
the compiler may replace occurences of the use of the defined name
in an an expresion with the initialising value, or, a combination
of both. The compiler may also shift the position of the initialising
assignment, in the case it choses to allocate store, to any point
after the definition and before the first use. It may also evaluate
the expression earlier, provided it does not impact the evaluation.
\end{explanation}

The kind of evaluation strategy described is a key feature of Felix
and is called {\em indeterminate evaluation} strategy. It is one
of the primary semantic design principles which provide high performance
by allowing the compiler to choose whether to evaluate the val
eagerly or lazily. In particular, val parameters of a closure
are almost universally eagerly evaluated, which avoids the need
to pass the closure a closure over the expression. On the other hand
if a function is called directly and is inlined, the compiler is free
to elide the parameter and substitue the expression for its occurrences,
thus evaluating it lazily and saving a storage location as well.

\section{{\tt def} statement}

A special statement called a \verb%def% statement can be used to
introduce zero or more \verb%var% variables, zero or more \verb%val%
variables, and perform simple assignments, simultaneously:

\begin{minted}{felix}
var x = 1;
def var a, (val b, x, _) = 2,(3,4,42);
\end{minted}

The LHS pattern is called an {\em l-pattern} which consists of
a list of {\em l-terms} each of which may be either a \verb%var%
binder introducing a new variable, a \verb%val% binder introducing
a new variable, a simple name refering to an existing \verb%var%
variable, an underscore, or a nested l-pattern enclosed in parens. 

The components other than underscore, are assigned the corresponding values 
from the term on the RHS which
must be of tuple type with the same number of components, recursively
for nested l-patterns; underscore is a place holder for a component which is to be ignored.

A component matching an underscore may or may not be evaluated.

\subsection{Uniquely typed components}

Uniqueness typing is provided to provide a way to implement 
move semantics instead of the usual copy and overwrite semantics
of assignment; it allows the programmer to assert acquisition
of exclusive ownership of an object, and transfer that ownership
along with the objects value.

Felix provides an intrinsic type constructor \verb%_uniq% which
maps a type T to its unique equivalent.

The programmer announces a value is acquired using the \verb%box% intrinsic,
which when applied to an object of type T returns an object of
type \verb%_uniq T%. The programmer can then pass the object to
a function which requires a uniquely typed parameter. The function
can then pass the object on, or use the \verb%kill% intrinsic to
forget it, otherwise, to use the boxed object the \verb%unbox%
intrinsic is appled to return the boxed type T.

\begin{minted}{felix}
begin 
  fun toupper (var s: uniq varray[char]) { 
    var x = unbox s;
    for i in 0 ..< x.len.int perform
      set(x,i,toupper (get(x,i)));
    return box x;
  }

  var u = box (varray "hello");
  u = toupper u;
  println$ (unbox u);
end
\end{minted}

Because a function acquiring a uniquely typed object is the
exclusive owner, it is free to modify the object without
compromising referential transparency, because by specification
there can be no other observer. 

In other words box/unbox together perform a move from
the caller of a function to the function's implementation,
using the uniquness typing of the interface to facilitate
enforcement of the contract.

Variables which have a type which is \verb%_uniq%, or a product type
with one or more component \verb%_uniq% are treated specially.
Control flow analysis attempts to assign a consistent state to such
variables or components, either {\em dead} or {\em live}. 

The library provides a functor \verb%uniq: TYPE -> TYPE% which maps
a type $T$ to \verb%_uniq T%.

\begin{enumerate}
\item Assignment to a live variable or component is not permitted and
an error reported if detected. 

\item Use of a dead variable or component
in an expression is not permitted and an error reported if detected.

\item Using a live variable or component changes the state of the component
from live to dead. 

\item Assigning to a dead component changes the state
from dead to live. 

\item If the variable or component is a parameter or
part of a parameter it is initially considered live, otherwise
if it is a local variable it is considered dead. 

\item A variable must be dead when control exits.

\item If a variable is used in one branch of a pattern match or
conditional it must be used in all of them.

\item A uniquely typed component of an variable of a product type
is treated as if it were a whole variable.

\begin{explanation}
This rule is subtle! For example:
\begin{minted}{felix}
begin
  var x = 1, uniq 2;
  var y = x; // kills x.1
  println$ y.0;
end  // ERROR: y.1 unused! 
\end{minted}
\end{explanation}

\item \verb%_uniq T% is a subtype of $T$. Therefore,
a variable of unique type can be passed as an argument
to a function with the corresponding non-unique type,
the argument will be implicitly unboxed.
\end{enumerate}

\begin{warning}
Top level global variables are not analysed and should not be uniquely typed
because they're not considered by the flow analyser.
\end{warning}

\begin{warning}
Taking the address of a uniquely type variable is ignored by the flow analyser. 
For therefore the correctness of the analysis depends on the programmer
not abusing pointers to unique variables or components.
\end{warning}

The flow analyser will report an error if a consistent state cannot
be assigned to a variable or component.

\begin{warning}
The flow analyser does first order static analysis only. It cannot
cope with either indirect or dynamic access.
\end{warning}

\chapter{Assignment Statements}
\href{http://felix-lang.org/share/lib/grammar/assignment.fsyn}{Syntax}

\section{Basic Assignment}
In Felix, the core assignment operation is an instrinsic procedure

\begin{minted}{felix}
  _storeat: &>T * T -> 0
\end{minted}

which is wrapped by a library procedure:

\begin{minted}{felix}
  proc storeat[T] (p:&>T,v:T) => _storeat(p,v);
\end{minted}

so it can be overloaded. The parser, in turn, maps the
statement

\begin{minted}{felix}
  p <- v;
\end{minted}

to call the library procedure. The type \verb%&>T% denotes a write-only
pointer to T, however the type \verb%&T% of a read-write pointer to T
is a subtype, so a pointer of that type can also be used for assignment:

\begin{minted}{felix}
  var x = 1;
  &x <- 2;
\end{minted}

Felix provides a conventional assignment operator:

\begin{minted}{felix}
  var x = 1;
  x = 2;          // means &x <- 2;
\end{minted}

The left hand side of such an assignment must be a variable,
or a dereference of a pointer expression:

\begin{minted}{felix}
  var px = new 1; // on the heap
  *px = 2;        // means px <- 2;
\end{minted}

Initialisation of a variable in a \verb%var% statement, or \verb%var% l-term
of a \verb%def% statement is also an assignment.

\section{Read-modify-write operations}
The following read/modify/write operations are similar to those in C.
Each of these dispatches to a {\em procedure} of the same name,
taking two arguments, the first being a read-write pointer and the second a value.
The parser automatically takes the address of the LHS of the assignment
when calling the procedure. Therefore, the LHS must be a simple variable.

\begin{minted}{felix}
  var x = 1u; var y=1u;

  // from class Integer
  // all integers
  x += y; // means +=(&x,y)
  x -= y; // means -=(&x,y)
  x *= y; // means *=(&x,y)
  x /= y; // means /=(&x,y)
  x %= y; // means %=(&x,y)
  x <<= y; // means <<=(&x,y)
  x >>= y; // means >>=(&x,y)

  // from class Bits
  // unsigned integers only
  x ^= y; // means ^=(&b,y)
  x |= y; // means |=(&b,y)
  x &= y; // means &=(&b,y)
\end{minted}


\begin{warning}
Felix does not guarrantee these operations are atomic.
\end{warning}

\section{Swap operator}
Felix provides the \verb%_swap% intrinsic procedure
to swap the contents of two variables.

\begin{minted}{felix}
var x,y = 1,2;
_swap (&x,&y); // now x=2, y=1 
\end{minted}

The variables must be of the same type.
The parser maps the infix operator \verb%<->% to the procedure \verb%swap%
which is defined in the library to call \verb%_swap% by default.

\begin{minted}{felix}
  var x = 1;
  var y = 2;
  x <-> y; // swap so x = 2, y = 1
\end{minted}

\section{Pre and post increment and decrement}
Felix provides pre and post increment and decrement.
These are procedures taking a pointer argument,
not expressions. The procedures have special names
as indicated:

\begin{minted}{felix}
  var x = 1;
  ++x; // means pre_incr (&x);
  --x; // means pre_decr (&x);
  x++; // means post_incr (&x);
  x--; // means post_decr (&x);
\end{minted}

\chapter{Control Flow Statements}
\section{The {\tt goto} statement and label {\tt prefix}}\index{goto}\index{label}
Felix statements may be prefixed by a label
to which control may be transfered by a \verb%goto%
statement:

\begin{minted}{felix}
alabel:>
  dosomething;
  goto alabel;
\end{minted}

The label must be visible from the goto statement.

There are three kinds of gotos.

\begin{enumerate} 

\item A {\em local goto} is a direct jump
to a label in the same scope as the goto statement.

\item A {\em non-local goto} is a direct jump to any other label,
visible in an ancestor scope.

\item An {\em indirect goto} is a goto with a non-constant 
expression of \verb%LABEL% type as its operand.
\end{enumerate}

\subsection{Non-local gotos}
Non-local transfers of control may cross procedure
boundaries. They may not cross function or generator 
boundaries.

The procedure or function containing the label 
must be active at the time of the control transfer.

\begin{explanation}
A non-local goto may be wrapped in a procedure closure
and passed to a procedure from which the goto target
is not visible.


\begin{minted}{felix}
proc doit (err: 1 -> 0) { err; }

proc outer () {
  proc handler () { goto error; }
  doit (handler);
  return;

  error:> println$ error;
}
\end{minted}

This is a valid way to handle errors.
the code is correct because \verb%outer% is active
at the time that \verb%handler% performs the
control transfer.
\end{explanation}

\subsection{Label values}
Labels are first class values of the kind \verb%LABEL%.
They can be stored in variables, and targetted by a goto statement.

\begin{minted}{felix}
proc demo (selector:int) {
  var pos : LABEL = 
    if selector == 1 
    then lab1
    else lab2
    endif
  ;
  goto pos;
lab1:>
  println$ "Lab1"; return;
lab2:>
  println$ "Lab2"; return;
}
\end{minted}

Used of saved label values may not be sound unless the procedure
frame of the procedure containing the label at the time
it is saved remains live at the time it is jumped to.

\begin{explanation}
Label values are not merely code addresses, they are
closures which capture the state of the containing
procedure frame. This prevents the frame from being
garbage collected, but it does not prevent control
being returned from the frame. The effect of executing
an already returned frame will cause the program
to refer to local variables which have the values they had
when last modified.  This is well defined!

In the current implementation, returning from a procedure
resets the return address to NULL which means a return statement
will actually return to the invoking scheduler terminating the fibre.
Decoupling an exhausted frame from its caller is not done to
make returning to a dead frame safe. This is a side effect
of an optimisation which is intended to prevent a data pointer
into the frame making the call chain reachable.
\end{explanation}


\subsection{Exchange of control}
Built on top of label adressing and indirect gotos,
the \verb%branch-and-link% instruction is conceptually
the most fundamental control instruction. The library
implementation is 

\begin{minted}{felix}
inline proc branch-and-link (target:LABEL, save:LABEL)
{
  save <- next;
  goto target;
  next:>
}
\end{minted}

A good example is \href{http://felix-lang.org/share/src/test/regress/rt/coroutines-01.fdoc}{here},
which shows an example of coroutines.

\section{Conditional Statements}
\subsection{if/goto}
If the condition is true, jump to the goto target, otherwise
continue without jumping.

\begin{minted}{felix}
if c goto lab;
\end{minted}


\subsection{if/return}
If the condition is true, return from the current procedure, otherwise
continue without returning.

\begin{minted}{felix}
if c return;
\end{minted}


\subsection{if/call}
If the condition is true, call the specified procedure,
otherwise continue without calling it.

\begin{minted}{felix}
if c call f x;
\end{minted}

\subsection{if/perform}
If the condition is true, execute the specified statement,
otherwise continue without executing it.
\begin{minted}{felix}
if c perform x = 1; // equivalent to
\end{minted}

\subsection{if/do/elif/else/done}
The procedural conditional branch is used to select
a control path based on a boolean expression.

The \verb%else% and \verb%elif% clauses are optional.

\begin{minted}{felix}
if c1 do 
  stmt1;
  stmt2;
elif c2 do
  stmt3;
  stmt4;
else
  stmt5;
  stmt6;
done
\end{minted}


The \verb%elif% clause saves writing a nested conditional.
The above is equivalent to:

\begin{minted}{felix}
if c1 do 
  stmt1;
  stmt2;
else 
  if c2 do
    stmt3;
    stmt4;
  else
    stmt5;
    stmt6;
  done
done
\end{minted}

One or more statements may be given in the selected control path.
The bodies of the subclasses do not form a scope. Variables declared
therein are visible in the containing context. Labels therein can
be targetted by gotos.

\section{halt}
Stops the program with a diagnostic.

\begin{minted}{felix}
halt "Program complete";
\end{minted}


\section{try/catch/endtry}
The try/catch construction may only be used to wrap
calls to C++ primitives, so as to catch exceptions.

\begin{minted}{felix}
proc mythrow 1 = "throw 0;";
try
   mythrow;
catch (x:int) =>
   println$ "Caughht integer " + x.str;
endtry
\end{minted}


\section{match/endmatch}\index{pattern matching}
The form:

\begin{minted}{felix}
match expr with
| pattern1 => stmts1
| pattern2 => stmts2
...
endmatch;
\end{minted}

is an extension of the C switch statement. See \fullref{Pattern Matching}.

The match argument expression is compared in turn with each
pattern. If a match is found, the pattern variables in the pattern
are bound to the corresponding sub-expressions of the argument expression.
Then the statements of that branch are executed with those variables in scope.
The variables are \verb%val% bound and so immutable. The branch statements
form a scope.

\section{chainmatch/ormatch}\index{pattern matching}
The form:

\begin{minted}{felix}
chainmatch expr1 with
| pattern1 => stmts1
| pattern2 => stmts2
ormatch expr2 with
| pattern3 => stmts3
| pattern4 => stmts4
ormatch expr3 with
| pattern5 => stmts5
| pattern6 => stmts6
| _ =>  stmts7 
endmatch;
\end{minted}

is sugar for:

\begin{minted}{felix}
match expr1 with
| pattern1 => stmts1
| pattern2 => stmts2
| _ =>
  match expr2 with
  | pattern3 => stmts3
  | pattern4 => stmts4
  | _ =>
    match expr3 with
    | pattern5 => stmts5
    | pattern6 => stmts6
    | _ =>  stmts7 
    endmatch;
  endmatch;
endmatch;
\end{minted}

\section{call}\index{procedure call}
The \verb%call% statement is used to invoke a procedure.

\begin{minted}{felix}
proc p(x:int) { println$ x; }
call p 1;
\end{minted}

The word \verb%call% may be elided in a simple call:

\begin{minted}{felix}
p 1;
\end{minted}

If the argument is of unit type; that is, it is the
empty tuple, then the tuple may also be elided in
a simple call:

\begin{minted}{felix}
proc f() { println$ "Hi"; }
call f (); // is equivalent to
f(); // is equivalent to
f;
\end{minted}


\section{procedure return}
The procedural return is used to return control
from a procedure to its caller.

A return is not required at the end of a procedure
where control would otherwise appear to drop through,
a return is assumed:

\begin{minted}{felix}
proc f() { println$ 1; }
// equivalent to
proc f() { println$ 1; return; }
\end{minted}


\section{return from}
The return from statement allows control to be
returned from an enclosing procedure, provided that
procedure is active.

\begin{minted}{felix}
proc outer () {
  proc inner () {
     println$ "Inner";
     return from outer;
  }
  inner;
  println$ "Never executed";
}
\end{minted}


\section{jump}
The procedural jump is an abbreviation for 
the more verbose sequence:

\begin{minted}{felix}
jump procedure arg; // is equivalent to
call procedure arg;
return;
\end{minted}

\section{callcc}
The callcc statement 
\begin{minted}{felix}
  callcc f a;
\end{minted}

calls the function $f$ passing the location
of the next statement as its argument. The returned value is
then called with $a$ as its argument.

The location is represented
as a value of \verb%LABEL% type. If the procedure wishes
to return to this point it should goto the parameter which received
the argument.

\begin{minted}{felix}
  proc f(cc:LABEL) (x:int) {
    println$ "F got " + x.str;
    goto cc;
  }
  callcc f 42;
  println$ "Back";
\end{minted}

\begin{explanation}
The callcc statement above is equivalent to:
\begin{minted}{felix}
  var tmo = f next;
  call tmp 42;
next:>
\end{minted}
\end{explanation}

\section{function return}
The functional return statement returns a value from
a function.

\begin{minted}{felix}
fun f () : int = {
  return 1;
}
\end{minted}

Control may not fall through the end of a function.

\section{yield}\index{yield}
The yield statement returns a value from a generator
whilst retaining the current location so that execution
may be resumed at the point after the yield.

For this to work a closure of the generator must be stored
in a variable which is subsequently applied.

\begin{minted}{felix}
gen counter () = { 
  var x = 0;
next_integer:>
  yield x;
  ++x;
  goto next_integer;
}

var counter1 = counter;
var zero = counter1 ();
var one = counter1 ();
println$ zero, one;
\end{minted}

\section{{\tt call\_with\_trap} and {\tt throw\_continuation}}\index{trap}
\index{continuation}\index{throw}\index{call\_with\_trap}
The \verb%call_with_trap% statement calls a procedure,
catching any uncaught thrown continuation, if there is one.

If control returns normally, it continues normally.

A procedure of type \verb%1->0% may be thrown with a
\verb%throw_continuation% statment. It will be caught by
the closest dynamically enclosing \verb%call_with_trap% 
statement or the top level scheduler.

Such a trapped continuation then replaces the currently
running procedure and begins execution.

\begin{minted}{felix}
fun divide (error:1->0) (a:int, b:int) = {
  if b == 0 perform throw_continuation error;
  return a/b;
}

proc demo() {
  proc divide_by_zero () {
    println$ "Divide by zero";
    goto bad;
  }
  var q = divide divide_by_zero (1,0);
  println$ "Quotient " + q.str;
  return;
bad:>
  println$ "Error Handled";
}
call_with_trap demo;
println$ "Done";
\end{minted}



\section{loops}
\href{http://felix-lang.org/share/lib/grammar/loops.fsyn}{Library Reference}

Felix has two kinds of loops, low level loops and high level loops.

Felix loops consist of a control section and a body. The body must be
a group statement. There are two kinds of group statements:

\begin{enumerate}
\item An {\em basic group} is a \verb%perform% statement,
a \verb%do%/\verb%done% group, an \verb%if%/\verb%do%/\verb%done%
statment, or a loop whose body is a basic group.

Basic groups do not constitute a scope.

\item A {\em block} A \verb%begin%/\verb%end% statement
or a statement \verb%match% statement is a block.

Block define an anonymous scope. Variables defined in a block
do not escape, and you cannot jump to a label in a block.

\end{enumerate}

\subsection{C like loop}
This loop is similar to a C for loop except that the increment operation
is a statement, not an expression, and so may need a semi-colon as a terminator.
It is the fastest and most flexible loop, but it is hardest to optimise.

\begin{minted}{felix}
for(var i=0; i<20; ++i;) 
  perform println$ i;
println$ "Final i " + i.str;
\end{minted}

Note the control variable is defined in the containing scope
and is not local to the loop.

\subsection{for/in/upto/downto}\index{for loop}
A basic loop with an inclusive range.

\begin{minted}{felix}
// up
for var ti:int in 0 upto 9 do println$ ti; done
for var i in 0 upto 9 do println$ i; done
for i in  0 upto 9 do println$ i; done

// down
for var tj:int in 9 downto 0 do println$ j; done
for var j in 9 downto 0 do println$ j; done
for j in  0 upto 9 do println$ j; done
\end{minted}

The start and end expressions must be of the same type.

If the control variable is defined in the loop with a type
annotation, that type must agree with the control variable.

The type must support comparison with the equality operator \verb%==%
the less than or equals operator \verb%<=% and increment with 
the pre increment procedure \verb%pre_incr%

The loop logic takes care to ensure the control variable is not
incremented (resp. decremented) past the end (resp.start) value.

Loops over unsigned integer types cannot handle the empty case.
Loops over signed integer types cannot span the whole range of the type.

\subsection{while loop}\index{while loop}
The while loop executes the body repeatedly whilst the control
condition is true at the start of the loop body.

\begin{minted}{felix}
var i = 0;
while i < 10 do println$ i; ++i; done
\end{minted}


\subsection{until loop}\index{until loop}
The until loop executes the loop body repeatedly
until the control condition is false at the start of the loop,
it is equivalent o a while loop with a negated condition.

\begin{minted}{felix}
var i = 0;
until i == 9 do println$ i; ++i; done
\end{minted}


\subsection{Control Flow modifiers}
Low level loops may be labelled with a loop label
which is used to allow break, continue, and redo
statements to modify the usual control flow. 

\begin{minted}{felix}
outer:for var i in 0 upto 9 do
   inner: for var j in 0 upto 9 do
     println$ i,j;
     if i == j do break inner; done
     if i * j > 60 do break outer; done
   done
done
\end{minted}


\subsubsection{redo}\index{redo}
The redo statement causes control to jump to the start
of the specified loop without incrementing the control variable.

\subsubsection{break}\index{break}
The break statement causes control to jump past the end of
the specified loop, terminating iteration.

\subsubsection{continue}\index{continue}
The continue statement causes the control variable to
be incremented and tests and the next iteration commenced
or the loop terminated.

\subsection{Parallel for loop}
The pfor loop is a high level loop which uses a thread pool to execute loops
over subranges of the specified range in parallel.

\begin{minted}{felix}
proc copy(inp: varray[uinptr], out: varray[uintptr]) {
  pfor i in 0 upto 99 do
    set(out, i, f (get(inp,j));
  done
}
\end{minted}

If the system thread pool is not created, it is constructed.
The range of the control variable is divided into $n$ subranges,
where $n$ is the number of threads. Then $n$ procedures are created
which run the loop over each subrange and are submitted to the
thread pool. The statement after the \verb%pfor% statement is executed
after all the threads have completed.

\begin{warning}
Note the thread pool is not destroyed, even if it was created,
the programmer must do this or the program will not terminate.
[This is a design fault!]
\end{warning}

The \verb%pfor% statement requires that iterations of the loop
be either uncoupled, or coupled appropriately, for example,
by using atomic variables.

\begin{warning}
If \verb%pfor% statements are nested, the result is undefined.
It is always bad for performance and is likely to lead to a deadlock.
A nested \verb%pfor% is executed from a job running in the thread
pool and will submit further jobs to the thread pool and then
wait for them to complete. Unfortunately it is likely the won't
start because the submitting job is still active, even though
it is waiting.
\end{warning}

\section{Loops over iterators}
The iterator based for loop takes sets a variable to each value stored in any
data type with an iterator:

\begin{minted}{felix}
for i in (1,2,3) perform println$ i;
for i in 0..<10 perform println$ i;
\end{minted}

An iterator is a generator taking the data type as an argument
and returning an option type. The iterator over the data type 
constructs and iterator and then repeatedly calls it until
it returns None, storing the argument $v$ of the \verb%Some v% 
otherwise returned in $i$ and then executing the loop body.

The iterator is obtained generically, that is, by attempting
to apply a function named \verb%iterator% to the container.

The argument can also be any generator returning an option type,
including any iterator.

This loop is wrapped in a block so that the control variable
is {\em not} available at the end of the loop, definitions 
in the body do not leak into the surrounding context,
and jumps into the loop body are not possible.

\subsection{Recursive for loop}
The \verb%rfor% loop is similar to the iterator based for loop,
however it constructs a new data frame for each iteration.
It does this by converting the loop into a procedure which 
self-tail calls itself to implement the iteration.

The compiler may optimise an \verb%rfor% loop using self-tail-call
optimisation, so it uses a single frame, as for a \verb%for% loop.
However, the optimisation is not performed unless it is safe,
and in particular if a pointer to a local variable is exported 
from the body, it cannot be performed.

\begin{explanation}
An rfor loop should be used when creating a closure that
binds to the loop body, if you wish it to bind to the
current values of that iteration. If a for loop is used
instead the closure will bind to the sole data frame
and containing the values of the last iteration, not
the values current at the time the closure is formed.
For example:

\begin{minted}{felix}
begin
  println$ "for loop ------";
  var v = varray[1->int] 4uz;
  for i in (0,1,2,3) perform push_back(v,{i});
  for j in 0..3 perform println$ #(v.j);
  // 3,3,3,3
end
begin
  println$ "rfor loop ------";
  var v = varray[1->int] 4uz;
  rfor i in (0,1,2,3) perform push_back (v,{i});
  for j in 0..3 perform println$ #(v.j);
  // 0,1,2,3
end
\end{minted}
\end{explanation}

\subsection{match/in}
This is an advanced iterator based loop which allows a single
arbitrary pattern match to be applied to the values supplied by
the iterator and present only those which match to the loop body.

\begin{minted}{felix}
  match x,y when x < y in ((1,2),(4,3)) 
    perform println$ x,y;
\end{minted}

\section{Assertions}
\href{http://felix-lang.org/share/lib/grammar/assertions.fsyn}{Library Reference}
\subsection{assert}
Ad hoc assertion throws an assertion exception if its argument
is false. 

\begin{minted}{felix}
assert x > 0;
\end{minted}


\subsection{axiom}\index{axiom}\index{assertion!axiom}
An axiom is a relationship between functions, typically
polymorphic, which is required to hold.

\begin{minted}{felix}
axiom squares (x:double) => x * x >= 0;
class addition[T]
{
  virtual add : T * T -> T;
  virtual == : T * T -> bool;

  axiom assoc (x:T, y:T, z:T) : 
    add (add (x,y),z) == add (x, add (y,z))
  ;
}
\end{minted}

In a class, an axiom is a specification constraining
implementations of virtual function in instances.

Axioms are restricted to first order logic, that is, they
may be polymorphic, but the universal quantification implied
is always at the head.

Existential quantification can be provided in a constructive
logic by actually constructing the requisite variable.

Second order logic, with quantifiers internal to the 
logic term, are not supported.

\subsection{lemma}
A lemma is similar to an axiom, except that is it
easily derivable from axioms; in particular,
a reasonable automatic theorem prover should
be able to derived it.

\subsection{theorem}
A theorem is similar to a lemma, except that it is 
too hard to expect an automatic theorem prover
to be able to derive it without hints or assistance.

There is currently no standard way to prove such hints.

\subsection{reduce}
A reduce statement specifies a term reduction and is logically
equivalent to an axiom, lemma, or theorem, however it acts
as an instruction to the compiler to attempt to actually 
apply the axiom.

The compiler may apply the axiom, but it may miss opportunities
for application.

The set of reductions must be coherent and terminal, 
that is, after a finite number of reductions the final
term must be unique and irreducible. 

Application of reduction is extremely expensive and they
should be used lightly.

\begin{minted}{felix}
reduce revrev[T] (x: list[T]) : rev (rev x) => x;
\end{minted}



\section{invariant}
An invariant is an assertion which must hold on the state variables
of an object, at the point after construction of the state
is completed by the constructor function and just before the
record of method closures is returned, and, at the start and
end of every method invocation.

The invariant need not hold during execution of a method.

Felix inserts the a check on the invariant into the constructor function
and into the post conditions of every procedure or generator
method.


\begin{minted}{felix}
object f(var x:int, var y:int) =
{
   invariant y >= 0;
   method proc set_y (newy: int) => y = newy;
}
\end{minted}


\section{code}
The code statement inserts C++ code literally into the current
Felix code.

The code must be one or more C++ statements.


\begin{minted}{felix}
code 'cout << "hello";';
\end{minted}


\section{noreturn code}
Similar to code, however noreturn code never returns.

\begin{minted}{felix}
noreturn code "throw 1;";
\end{minted}


\section{Service call}
The service call statement calls the Felix system kernel
to perform a specified operation.

It is equivalent to an OS kernel call.

The available operations include:

\begin{minted}{felix}
  union svc_req_t =
  /*0*/ | svc_yield
  /*1*/ | svc_get_fthread         of &fthread    // CHANGED LAYOUT
  /*2*/ | svc_read                of address
  /*3*/ | svc_general             of &address    // CHANGED LAYOUT
  /*4*/ | svc_reserved1
  /*5*/ | svc_spawn_pthread       of fthread
  /*6*/ | svc_spawn_detached      of fthread
  /*7*/ | svc_sread               of _schannel * &gcaddress
  /*8*/ | svc_swrite              of _schannel * &gcaddress
  /*9*/ | svc_kill                of fthread
  /*10*/ | svc_reserved2
  /*11*/ | svc_multi_swrite       of _schannel * &gcaddress 
  /*12*/ | svc_schedule_detached  of fthread
  ;
\end{minted}


These operations are typically related to coroutine or thread scheduling.
However \verb%svc_general% is an unspecified operation, which is typically
used to invoke the asynchronous I/O subsystem.

Service calls can only be issued from flat code, that is,
from procedures, since they call the system by returning
control, the system must reside exactly one return address
up the machine stack at the point a service call is executed.

\section{with/do/done}
The with/do/done statement is use to define temporary variables
which are accessible only in the do/done body of the statement.

It is the statement equivalent of the let expression.

\begin{minted}{felix}
var x = 1;
with var x = 2; do println$ x; done
assert x == 1;
\end{minted}


\section{do/done}
The do/done statement has no semantics and merely acts as a
way to make a sequence of statements appear as a single
statement to the parser.

Jumps into do/done groups are therefore allowed, and
any labels defined in a do/done group are visible in
the enclosing context.

Any variables, functions, or other symbols defined in a do/done
group are visible in the enclosing context.

\begin{minted}{felix}
do something; done
\end{minted}


\section{begin/end}
The begin/end statement creates an anonymous procedure
and then calls it. It therefore appears as a single statement
to the parser, but it simulates a block as would be used in C.
It is exactly equivalent to a brace enclosed procedure called
by a terminating semi-colon.

\begin{minted}{felix}
begin
  var x = 1;
end
// equivalent to
{
  var x = 1;
};
\end{minted}


\chapter{Expressions}
\href{http://felix-lang.org/share/lib/grammar/expressions.fsyn}{Syntax}

Expressions are listed in approximate order of precedence,
starting with the weakest binding.

We will often exhibit expressions in the form

\begin{minted}{felix}
var x = expr;
\end{minted}

so as to present a complete statement. The \verb%x% is of no significance.

\section{Priority levels}
\begin{minted}{felix}
priority 
    let_pri < 
    slambda_pri <
    spipe_apply_pri <
    sdollar_apply_pri < 
    stuple_cons_pri <
    stuple_pri <
    simplies_condition_pri <
    sor_condition_pri <
    sand_condition_pri <
    snot_condition_pri <
    stex_implies_condition_pri <
    stex_or_condition_pri <
    stex_and_condition_pri <
    stex_not_condition_pri <
    scomparison_pri <
    sas_expr_pri <
    ssetunion_pri <
    ssetintersection_pri <
    sarrow_pri <
    scase_literal_pri <
    sbor_pri <
    sbxor_pri <
    sband_pri <
    sshift_pri <
    ssum_pri <
    ssubtraction_pri <
    sproduct_pri <
    s_term_pri <
    sprefixed_pri <
    spower_pri <
    ssuperscript_pri <
    srefr_pri <
    sapplication_pri <
    scoercion_pri <
    sfactor_pri <
    srcompose_pri <
    sthename_pri <
    satomic_pri
  ;
\end{minted}




\section{Chain forms}
\subsection{Pattern let}
The traditional let binding of ML.
The syntax is 

\begin{minted}{felix}
var x = let pattern = expr1 in expr2; // equivalent to
var x = match expr1 with pattern => expr2 endmatch;
\end{minted}

\begin{minted}{felix}
var x = let a = 1 in a + 1; // equivalent to
var x = match 1 with a => a + 1 endmatch
\end{minted}


\subsection{Function let}\index{let fun}
A let form which makes a function available in the
expression

\begin{minted}{felix}
var x = 
  let fun f(y:int)=> y + 1 in 
  f 42
;
\end{minted}

\subsection{Match chain}
A variant on the terminated match which allows a second
match to be chained onto the last branch without any endmatch.

\begin{minted}{felix}
var y = list (1,2);
var x = 
  match y with
  | #Empty => "Empty"
  | _ =>
  match y with 
  | h ! Empty => h.str
  | _ =>
  match y with 
  | h1 ! h2 ! Empty => h1.str + "," + h2.str
;

println$ x;
\end{minted}

\subsection{conditional chain}
A variant on the terminated if/then/elif/else allowing chaining.

\begin{minted}{felix}
var x = 
  if c1 then r1 
  elif c2 then r2 else
  if c3 then r3 else
  r4
;
\end{minted}


\section{Alternate conditional chain}

\begin{minted}{felix}
var x = n / d unless d == 0 then 0;
\end{minted}


\section{Dollar application}
A right associative low precedence forward apply operator taken
from Haskell.

\begin{minted}{felix}
var x =  str$ rev$ list$ 1,2,3;
\end{minted}



\section{Pipe application}
A left associative low precedence reverse apply operator taken
from \verb%C#%.

\begin{minted}{felix}
var x =  1,2,3 |> list |> rev |> str;
\end{minted}


\section{Tuple cons constructor}
A right associative cons operator for tuples. Allows concatenating
an element to the head/front/left end of a tuple. Can also be used
in a pattern match to recursively decode a tuple like a list.

\begin{minted}{felix}
var x =  3,4;
var y = 1 ,, 2 ,, x; // 1,2,3,4
\end{minted}


\section{N-ary tuple constructor}
The is a non-associative n-ary tuple constructor
consists of a sequence of expressions separated by commas.

\begin{minted}{felix}
var x = 1,2,3,4;
\end{minted}


\section{Logical implication}
An operator for function \verb%implies%.

\begin{minted}{felix}
var x = false implies true;
\end{minted}


\section{Logical disjunction}
A chaining operator for function \verb%lor%.

\begin{minted}{felix}
var x = true or false;
\end{minted}

A shortcut enabled disjunction is available too:

\begin{minted}{felix}
fun f(x:int, y:int) =>
  y == 0 or else  x / y > 1 
;
println$ f(1,0), f(10,1), f(1,10);
\end{minted}

This operation is very expensive. The second argument is translated
to a closure and the system \verb%orelse% function is called:

\begin{minted}{felix}
  noinline fun orelse (x: bool, y:1->bool) : bool => 
    if x then true else #y
  ;
\end{minted}

The closure and conditional ensure the second argument
is only evaluated if the first is false. 

 
\section{Logical conjunction}
A chaining operator for function \verb%land%.

\begin{minted}{felix}
var x = true and false;
\end{minted}

A shortcut enable conjunction is available too:

\begin{minted}{felix}
  fun g(x:int, y:int) => 
    y != 0 and then x / y <= 1
  ;
  println$ g(1,0), g(10,1), g(1,10);
\end{minted}

This operation is very expensive. The second argument is translated
to a closure and the system \verb%andthen% function is called:

\begin{minted}{felix}
  noinline fun andthen (x: bool, y:1->bool) : bool => 
    if x then #y else false
  ;
\end{minted}

The closure and conditional ensure the second argument 
is only evaluated if the first is true.

\section{Logical negation}
A bool operator for function \verb%lnot%.

\section{Comparisons}
Non-associative.

\begin{minted}{felix}
var x =
  a < b or 
  a > b or
  a <= b or
  a >= b or
  a == b or
  a != b or
  1 in list(1,2,3)
  1 \in list (1,2,3)
;
\end{minted}


\section{Name temporary}
Allows a subexpression to be named as a \verb%val% by default
or a \verb%var%.

\begin{minted}{felix}
var x = a +  (f y as z) + z; // equivalent to
val z = f y; var x = a + z + z;

var x = a +  (f y as var k) + k; // equivalent to
var k = f y; var x = a + k + k;
\end{minted}

Note that the \verb%var% for ensures the subexpression is eagerly
evaluated, before the containing expression.

\section{Schannel pipe operators}
Used to flow data through schannels from the source on the 
left to the sink on the right via processing units in between.

\begin{minted}{felix}
spawn_fthread$ source |-> filter |-> enhancer |-> sink;
\end{minted}


This variant uses an iterator to stream data out of a data structure:

\begin{minted}{felix}
spawn_fthread$ list (1,2,3) >-> sink;
\end{minted}


\section{Right Arrows}
Right associative arrow operators.

List cons operator.

\begin{minted}{felix}
var x = 1 ! 2 ! list (3,4);
\end{minted}


Function types (type language only):

\begin{minted}{felix}
  D -> C // Felix function
  D --> C // C function pointer
\end{minted}


\section{Case literals}

The case tag is only used in pattern matches.
The sum or union type isn't required because it can be deduced
from the match argument.

\begin{minted}{felix}
var a = match a with Some v  => v | #None => 0;
\end{minted}


The case constructor with integer caseno has two uses. 

It creates a value of
a sum type with no arguments:

\begin{minted}{felix}
var x = case 1 of 2; // aka true
\end{minted}

or it is a function for a sum type variant with an argument:

\begin{minted}{felix}
var x = (case 1 of int + double) 4.2;
\end{minted}


A case literal with a name instead of an integer constructs
a variant instead:

\begin{minted}{felix}
typedef maybe = union { No; Yes of int; };
var x = (case Yes of maybe) 42;
\end{minted}


The tuple projection function names a tuple projection:

\begin{minted}{felix}
typedef triple = int * long * string;
var snd = proj 1 of triple;
var y: int = snd (1, 2L, "3");
\end{minted}


\section{Bitwise or}
Left associative.

\begin{minted}{felix}
var x = q \| b;
\end{minted}


\section{Bitwise exclusive or}
Left associative.

\begin{minted}{felix}
var x = q \^ b;
\end{minted}


\section{Bitwise and}
Left associative.

\begin{minted}{felix}
var x = q \& b;
\end{minted}


\section{Bitwise shifts}
Left associative.

\begin{minted}{felix}
var x = a << b; // left shift
var x = a >> b; // right shift
\end{minted}


\section{Addition}
Chain operator. Non-associative for types. 
Left associative for expressions.

\begin{minted}{felix}
var x = a + b + c;
\end{minted}


\section{Subtraction}
Left associative.

\begin{minted}{felix}
var x = a - b;
\end{minted}


\section{Multiplication}
Chain operator. Non-associative for types.
Left associative for expressions.

\begin{minted}{felix}
var x = a * b;
\end{minted}


\section{Division operators}
Left associative. Not carefully: higher precedence
that multiplication, unlike C!!

\begin{minted}{felix}
var x = a * b / c * d; // means
var x = a * (b / c) * d;

var x = a * b % c * d; // means
var x = a * (b % c) * d;
\end{minted}


\section{Prefix operators}

\begin{minted}{felix}
var x = !a;
var x = -a; // negation
var x = ~a; // bitwise complement
\end{minted}


\section{Fortran exponentiation}
Infix \verb%**} is special syntax for function @{pow%. 
The left operand binds more tightly than \verb%**%
but the right operand binds as for prefixed operators
or more tightly. Observe than that:


\begin{minted}{felix}
var x = -a**-b;  // means
var x = -(a**(-b));
\end{minted}

preserving the usual mathematical syntax. 

\section{Felix exponentiation}
Left associative. The right operand binds as deref operator
or more tightly. Used for array notation in the type language.

\begin{minted}{felix}
var x = a ^ ix;
\end{minted}


\section{Function composition}\index{composition}\index{\(\circ\)}
Standard math notation. Left associative. Same precedence as
exponentiation. Spelled \verb%\}@{circ%.

\begin{minted}{felix}
var x = f \circ g;
\end{minted}


\section{Dereference}\index{dereference}
For function \verb%deref%.

\begin{minted}{felix}
var x = *p;
\end{minted}


For builtin dereference operator:

\begin{minted}{felix}
var x = _deref p;
\end{minted}


Note these usually have the same meaning however
the function \verb%deref% can be overloaded. If the overloaded
definition is not to be circular it may use \verb%_deref%
when dereferencing pointers.

\subsection{Operator new}
Copies a value onto the heap and returns a pointer.

\begin{minted}{felix}
var px = new 42;
\end{minted}


\section{Whitespace application}
Operator whitespace is used for applications.
\subsection{General}

\begin{minted}{felix}
var x = sin y;
\end{minted}


\subsection{Caseno operator}\index{casenp}
Returns the integer tag value of the value of an anonymous sum, 
union, or variant  type.

\begin{minted}{felix}
var x = caseno true; // 1
var x = caseno (Some 43); // 1
\end{minted}


\subsection{Likelyhood}\index{likely}
Indicaties if a bool valued expression is likely
or unlikely to be true. Used to generate the corresponding
gcc optimisation hints, if available.

\begin{minted}{felix}
if likely (c) goto restart;
if unlikely (d) goto loopend;
\end{minted}


\section{Coercion operator}\index{coercion}
left associative. The right operand is a type.

\begin{minted}{felix}
var x = 1L :>> int; // cast
\end{minted}


\section{Suffixed name}\index{name!suffixed}
The most general form of a name:

\begin{minted}{felix}
var x = qualified::name of int;
\end{minted}

Used to name functions, with the right operand specifying
the function argument type.

\section{Factors}
\subsection{Subscript}\index{subscript}
Used for array and string subscripting. Calls function \verb%subscript%.
For strings, returns a character. If the subscript is out of range
after adjustment of negative index, returns \verb%char 0% and thus
cannot fail.

\begin{minted}{felix}
var x = a . [ i ]; // i'th element
\end{minted}


\subsection{Subsstring}\index{substring}
Calls function \verb%substring%. Negative indices may be used to
offset from end, i.e. -1 is the index of the last element.
Out of range indices (past the end or before the start, after
adjustment of negative indices) are clipped back to the end
or start respectively.

\begin{minted}{felix}
var x = a . [ first to past]; 
  // past is one past the last element refered to
\end{minted}


\subsection{Copyfrom}
Calls function \verb%copyfrom%. Copies from designated index to end.
Supports negative indices and range clipping for strings.

\begin{minted}{felix}
var x = a . [to past]; // from the first
\end{minted}


\subsection{Copyto}
Calls function \verb%copyto%. Supports negative indices and
range clipping for strings.

\begin{minted}{felix}
var x = a . [to past]; // from the first
\end{minted}


\subsection{Reverse application}\index{application!reverse}
Left associative.

\begin{minted}{felix}
var x = y .f; // means
var x = f y;
\end{minted}


\subsection{Reverse application with deref}\index{application!reverse with deref}
Left associative

\begin{minted}{felix}
var x = p *. k; // means
var x = (*p) . k;
\end{minted}


\subsection{Reverse application with addressing}\index{application!reverse with address}
Left associative

\begin{minted}{felix}
var x = v &. k; // means
var x = (&v) . k;
\end{minted}


\subsection{Unit application}\index{application!unit}
Prefix operator applies argument to empty tuple.

\begin{minted}{felix}
var x = #f; // means
var x = f ();
\end{minted}


\subsection{Addressing}\index{address!variable}
Finds the pointer address of a variable.
Means pointer to in type language.

\begin{minted}{felix}
var x : int = 1;
var px : &int = &x; 
  // address of x
  // type pointer to int
\end{minted}


\subsection{C pointer}\index{pointer!C}
Used in type language for pointer to type or NULL.

\begin{minted}{felix}
var px : @char =  malloc (42);
\end{minted}


Note that this symbol is also used in fdoc as a markup
indicator. Please keep out of column 1, do not follow
with a left brace.

\subsection{Macro freezer}
Used to disable macro expansion of a symbol.

\begin{minted}{felix}
macro val fred = joe;
var x = fred + noexpand fred; // means
var x = joe + fred;
\end{minted}


\subsection{Pattern variable}\index{pattern!variable}
Notation \verb%v% Used in patterns to designate a val variable
to be bound in the pattern matching.

\begin{minted}{felix}
var x = 
  match y with 
  | Some v => "Some " + v.str 
  | #None => "None";
;
\end{minted}


\subsection{Parser argument}\index{argument!parser}
Notation \verb%n} for some integer @{n%. In user defined syntax designates
the n'th term of a syntax production.

\section{Qualified name}\index{name!qualified}
A name in Felix has the form:

\begin{minted}{felix}
  class1 :: nested1 :: ... :: identifier [ type1, type2, ... ]
\end{minted}

where the qualifiers and/or type list may be elided. 
This is the same as C++ except we use \verb%[]} instead of @{<>%
for template argument types.

\chapter{Atoms}
\section{Record expression}\index{record!expression}

\begin{minted}{felix}
var x = (name="Hello", age=42);
\end{minted}


\section{Alternate record expression}

\begin{minted}{felix}
var x = 
  struct {
    var name = "Hello";
    var age = 42;
  }
;
\end{minted}


\section{Variant type}
Denotes a variant type.

\begin{minted}{felix}
var x : 
  union {
    Cart of double * double;
    Polar of double * double;
  }
;
\end{minted}


\section{Wildcard pattern}\index{pattern!wildcard}
Used in a pattern match, matches anything.

\begin{minted}{felix}
  var x = match a with _ => "anything";
\end{minted}


\section{Ellipsis}\index{ellipsis}\index{varargs}
Used only in C bindings to denote varags.

\begin{minted}{felix}
  fun f: int * ... -> int;
\end{minted}


\section{Truth constants}\index{true}\index{false}\index{bool}

\begin{minted}{felix}
  false // alias for case 0 of 2
  true  // alias for case 1 of 2
\end{minted}


\section{callback expression}
??

\begin{minted}{felix}
callback [ x ]
\end{minted}


\section{Lazy expression}
Function of unit.

\begin{minted}{felix}
var f = { expr };
var x = f ();
\end{minted}


\section{Sequencing}\index{sequencing}
Function dependent on final expression.

\begin{minted}{felix}
var x = ( var y = 1; var z = y + y; z + 1 ); // equivalent to
var x = #{ var y = 1; var z = y + y; return z + 1; };
\end{minted}


\section{Procedure of unit.}\index{procedure!unit}

\begin{minted}{felix}
var p = { println$ "Hello"; } // procedure
p (); 

var f = { var y = 1; return y + y; }; // function
var x = f ();
\end{minted}


\section{Grouping}\index{grouping!parentheses}\index{parentheses!grouping}
Parentheses are used for grouping.

\begin{minted}{felix}
var x = (1 + 2) * 3;
\end{minted}


\section{Object extension}

\begin{minted}{felix}
var x = extend a,b with c end;
\end{minted}


\section{Conditional expression}\index{conditional}

\begin{minted}{felix}
var x =
  if c1 then a elif c2 then b else c endif
;
\end{minted}

\chapter{Pattern Matching}\label{Pattern Matching}\index{pattern!matching}
The patterns
are composed of these forms:

\begin{minted}{felix}
pat | pat | pat ...       // alternate matches
(v1, v2, ... )            // tuple match
h!t                       // list match
h,,t                      // tuple cons
Ctor                      // const union or variant match
Ctor v                    // nonconst union or variant match
(fld1=f1, fld2=f2, ...)   // record match
pat as v                  // assign variable to matched subexpression
pat when expr             // guarded match
pat1 | pat2               // match either pattern
999                       // integer match
"str"                     // string match 
lit1 .. lit2              // range match
_                         // wildcard match
\end{minted}


The guarded match only matches the pattern if the guard expression is true.

\begin{minted}{felix}
match x with
| (x,y) when y != 0 => ...
endmatch
\end{minted}


The tuple as list cons match is a form of row polymorphism\index{polymorphism!row}
where the first element of a tuple and the remaining elements
considered as a tuple are matched.

A good example of this is found in the library \href{http://felix-lang.org/share/lib/std/datatype/tuple.flx}{here}
which allows printing a tuple of arbitrary number of components, indeed,
this facility was implemented precisely to allow this definition in the library.

Record\index{record!matching} matches succeed with any record containing a superset
of the specified fields.

As well as integer and string matches, a literal of any type with an
equality and inequality operator can be matched. In addition,
if there is a less than or equal operator \verb%<=% an inclusive
range match can be specified.


\chapter{Definitions}\index{definitions|textbf}
\href{http://felix-lang.org/share/lib/grammar/functions.fsyn}{Syntax}

\section{Functions}\index{function!definition|textbf}
A felix function definition takes one of three basic forms:

\begin{minted}{felix}
fun f (x:int) = { var y = x + x; return y + 1; }
fun g (x:int) => x + x + 1;
fun h : int -> int = | x => x + x + 1;
\end{minted}

The first form is the most general, the body 
of the function contains executable statements
and the result is returned by a return statement.

The second form is equivalent to a function in the first
form whose body returns the RHS expression.

The third form specifies the function type then the
body of a pattern match. It is equivalent to

\begin{minted}{felix}
fun h (a:int) = { return match a with | x => x + x + 1 endmatch; }
\end{minted}

The first two forms also allow the return type to be
specified:

\begin{minted}{felix}
fun f (x:int) : int = { var y = x + x; return y + 1; }
fun g (x:int) :int => x + x + 1;
\end{minted}


Functions may not have side effects.\index{side-effect!function}

All these function have a type:\index{type!function}\index{function!type}

\begin{minted}{felix}
D -> C
\end{minted}

where D is the domain and C is the codomain: both would
be \verb%int% in the examples.

A function can be applied by the normal forward
notation using juxtaposition or what is whimsically
known as operator whitespace, or in reverse notation
using operator dot:

\begin{minted}{felix}
f x
x.f
\end{minted}

Such applications are equivalent.  Both operators are left
associative. Operator dot binds more
tightly than whitespace so that

\begin{minted}{felix}
f x.g    // means
f (g x)
\end{minted}


A special notation is used for application to the unit tuple:

\begin{minted}{felix}
#zero // means
zero ()
\end{minted}

The intention is intended to suggest a constant since a pure
function with unit argument must always return the
same value. 

This hash operator binds more tightly than operator dot so

\begin{minted}{felix}
#a.b // means
(#a).b
\end{minted}



\section{Pre- and post-conditions}\index{precondition!function}\index{postcondition!function}
\index{function!precondition}\index{function!postcondition}\index{expect!function}\index{function!when}
\index{function!result}
A function using one of the first two forms
may have pre-conditions, post-conditions, or both:

\begin{minted}{felix}
fun f1 (x:int when x > 0) => x + x + 1;
fun f2 (x:int) expect result > 1 => x + x + 1;
fun f3 (x:int when x > 0) expect result > 1 => x + x + 1;
fun f4 (x:int when x > 0) : int expect result > 1 => x + x + 1;
\end{minted}

Pre- and pos-conditions are usually treated as boolean assertions
which are checked at run time. The compiler may occasionally be able
to prove a pre- or post-condition must hold and elide it.

The special identifier \verb%result% is used to indicate the return
value of the function.

\section{Higher order functions}\index{function!higher order}
A function may be written like

\begin{minted}{felix}
fun hof (x:int) (y:int) : int = { return x + y; }
fun hof (x:int) (y:int) => x + y;
\end{minted}

These are called higher order functions of arity 2.\index{function!arity}\index{arity!function}
They have the type

\begin{minted}{felix}
int -> int -> int   // or equivalently
int -> (int -> int) //since -> is right associative.
\end{minted}

They are equivalent to

\begin{minted}{felix}
fun hof (x:int) : int -> int = 
{
  fun inner (y:int) : int => x + y;
  return inner;
}
\end{minted}

that is, a function which returns another function.

Such a function can be applied like

\begin{minted}{felix}
hof 1 2 // or equivalently
(hof 1) 2
\end{minted}

since whitespace application is left associative.

\section{Procedures}\index{procedure!definition}
A function which returns control but no value is called a procedure.
Procedures may have side effects.


\begin{minted}{felix}
fun show (x:int) : 0 = { println x; }
proc show (x:int) { println x; }
proc show (x:int) => println x;
\end{minted}


The second form is a more convenient notation.
The type 0 is also called \verb%void% and denotes
a type with no values.

A procedure may return with a simple return statement:


\begin{minted}{felix}
proc show (x:int) { println x; return; }
\end{minted}


however one is assumed at the end of the procedure
body.

Procedures can also have pre- and post-conditions.

A procedure may be called like an application,
however it must be a whole statement since
expressions of type void may not occur interior
to an expression.


\begin{minted}{felix}
show 1;
1.show;
\end{minted}


If a procedure accepts the unit argument, it may be elided:


\begin{minted}{felix}
proc f () =>  show 1;
f; // equivalent to
f ();
\end{minted}


\section{Generators}
A generator is a special kind of function which is allowed
to have side effects. It is defined similarly to a function,
but using the binder \verb%gen% instead of \verb%fun%:


\begin{minted}{felix}
var seqno = 1;
gen seq () { var result = seqno; ++seqno; return result; }
\end{minted}


When a generator is directly applied in an expression, the application
is replaced by a fresh variable and the generator application
is lifted out and assigned to the variable. For example:


\begin{minted}{felix}
fun twice (x:int) => x + x;
println$ twice #seq;
\end{minted}


will always print an even number because it is equivalent to


\begin{minted}{felix}
var tmp = #seq;
println$ twice tmp;
\end{minted}


Therefore even if twice is inlined we end up with the argument
to \verb%println} being @{tmp+tmp} and not @{#seq+#seq% which would
print an odd number.

\subsection{Yielding Generators}\index{yield}\index{generator!yielding}
A generator may contain a \verb%yield% statement:

\begin{minted}{felix}
gen fresh() {
  var x = 1;
  while x < 10 do
    yield x;
    ++x;
  done
  return x;
}
\end{minted}

In order to use such a yielding generator, a closure of the generator
must be stored in a variable. Then the generator may be called repeatedly.


\begin{minted}{felix}
var g = fresh;
for i in 1 upto 20 do 
  println$ i,#g;
done
\end{minted}


This will print pairs (1,1), (2,2) up to (10,10) then print
(11,10), (12,10) up to (20,10).

The yield statement returns a value such that a subsequent call
to a closure of the generator will resume execution after the yield
statement. Therefore \verb%yield% is a kind of cross between a return
and a subroutine call.

If a generator executes a return statement, that is equivalent
to yielding a value and setting the resume point back to the return
statement, in other words \verb%return expr;% is equivalent to

\begin{minted}{felix}
while true do yield expr; done
\end{minted}


Yielding generators should not be called directly because they
will always start at the beginning with a fresh copy of any
local variables used to maintain state.

Function closures differ from generator closures in that
the closures is cloned before every application to ensure
that the initial state is fresh.

Yielding generators are primarily intended to implement iterators,
that is, to provide lazy lists or streams.

\section{Conversion Operators}\index{conversion}\index{constructor}
Felix provides a special notation which allows an identifier
naming a type to return a value of that type:

\begin{minted}{felix}
typedef cart = dcomplex;
typedef polar = dcomplex;
ctor cart (x:double, y:double) => dcomplex (x,y);
ctor polar (r: double, theta: double) => 
  dcomplex (r * sin theta, r * cos theta)
;
var z = cart (20.0,15.0) + polar (25.8, 0.7 * pi);
\end{minted}

Such conversion operators are sometimes (incorrectly) called
constructors.  The constructions are equivalent to

\begin{minted}{felix}
fun _ctor_cart (x:double, y:double) : cart => dcomplex (x,y);
fun _ctor_polar (r: double, theta: double): polar => 
  dcomplex (r * sin theta, r * cos theta)
;
\end{minted}

When a type with a simple name is applied to a value,
Felix tries to find a function with that name prefixed by  \verb%_ctor_%
instead.

Note that Felix generates a constructor for struct and cstruct types
automatically with argument type the product of the types of the
structure fields.

\section{Special function {\tt apply}}\index{apply!non-function}
When Felix finds and application

\begin{minted}{felix}
  f a
\end{minted}

where \verb%f% is a value of type F which is not
a function (or C function) type, Felix looks instead 
for a function named apply with argument of type:

\begin{minted}{felix}
F * A
\end{minted}

where A is the type of a. For example

\begin{minted}{felix}
fun apply (x:string, y:string) => x + y; // concat
var x = "hello " "world"; // apply a string to a string
\end{minted}


\section{Dynanic Objects}
\label{Dynamic Objects}\index{object!dynamic}
Felix provides an object system with syntax based on Java,
and technology based on CLOS.

An object is a record of function closures, closed over the local
scope of a constructor function that returns the record.


\begin{minted}{felix}
interface person_t {
  get_name: 1 -> string; 
  set_age: int -> 0; 
  set_job : string -> 0;
  get_job : 1 -> string;
}

object person (name:string, var age:int) implements person_t =
{
  var job = "unknown";
  method fun get_name () => name;
  method proc set_age (x:int) { age = x; }
  method fun get_job () => job;
  method proc set_job (x:string) {  job = x; }
}

var john = person ("John", 42);
println$ #(john.name) + " is " + #(john.age).str;
\end{minted}

The entity \verb%person% is a function which when called with
its argument of name and age returns a record of type
\verb%person_t% consisting of closures of the functions and procedures
marked as method in its definition.

Since functions hide their local variables the object state
is hidden and can only be accessed using the methods.

The \verb%implements% clause is optional.

Objects provide an excellent way for a dynamically loaded
shared library to export a set of functions, only the object
function needs to be exported so it has a C name which
can be linked to with \verb%dlopen%.



\part{Gross Structure}

\chapter{Program structure}
A Felix program consists of a nominated root parse unit and
the transitive closure of units with respect to inclusion.

The behaviour of this system consists of the action of the
initialisation code in each unit, performed in sequence
within a given unit, with the order of action between
units unspecified.

\section{Parse Unit}
A parse unit is a file augmented by prefixing specified import
files to the front. These consist of a suite of grammar files
defining the syntax and other files defining macros.

By convention syntax files have the extension \verb%.fsyn%,
and other import files have the extension \verb%.flxh%.

With this augmentation all parse units in a program
are independently parsed to produce an list of statements
represented as abstract syntax, denoted an AST (even
though it is a list of trees, not a single tree).

\section{AST}
The program consists of the concatenation of the ASTs
of each parse unit, resulting in a single AST, which
is then translated to a C++ translation unit by the
compiler.

The ASTs to be concatenated are those parsed from
the parse units specified by the transitive closure
of include directives.

The order of concatenation is unspecified.
Note that include directives do {\em not} cause
an AST to be inserted at the point where the
directive is written.

Parse units can be and are parsed independently.
The contents of a parse unit cannot influence the
parsing of another.

\section{Include Directive}\index{include!directive}
\subsection{Syntax}
An include directive has the syntax:

\begin{minted}{felix}
include "filename";
include "dirname/filename";
include "./filename";
\end{minted}

where the filename is a Unix relative filename,
may not have an extension, and may not begin with or 
contain \verb%..% (two dots).

If the filename begins with \verb%./% then the balance of the name
is relative, a sibling of the including file, otherwise the name
is searched for on an include path. 

In either case, a search succeeds when it finds a file with
the appropriate base path in the search directory with
extension \verb%.flx% or \verb%.fdoc%. If both files exist the
most recently changed one is used. If the time stamps are
the same the choice is unspecified.

\subsection{Effect}
An include directive is included in the AST generated
by the parser. 

An extraction process extracts the set of directives
and locates the files specified and adds them to a set
of files which must be parsed.

Unparsed files are upgraded to parse units and parsed,
and then the extraction process again applied. 

When all the files have been processed the resulting
ASTs are concatenated.

Felix uses a caching scheme to avoid parsing all the
files every time.

\section{Grammar}
The Felix grammar is part of the library.
It is notionally prefixed to each file to be processed
prior to any import files to specify the syntax
with which the file is to be parsed and translated to
an AST.

The grammar uses an augmented BNF like syntax
with parse actions specified in R5RS Scheme.

The resulting S-expressions are translated to
an intermediate form and then into an internal
AST structure.

The parser is a scannerless GLR parser with significant
extensions.

\chapter{Parsing}
Felix has the most advanced parser of any production
programming language. The parser is extensible, and Felix
makes good use of this. The core, hardcoded grammar, called
the {\em bootstrap grammar} does not provide the syntax for
the Felix programming language the average production programmer
is interested in.

Therefore, we recommended this chapter be lightly scanned
or even skipped, until such time as a deeper understanding
is required. Nevertheless we feel obligated to present
some details now, since the ability to define Domain Specific
Sub-Languages is one of the key advantages of the Felix system.

In Felix, the usual programming language is defined by a grammar
in the library in user space. The compiler interprets the 
grammar to produce a parser
for the scope in which the grammar is active. Grammar extension
activations are properly scoped.

As with any modern programming language it is sometimes difficult
to distinguish features present in the core language and those
defined in the standard library. In particular the semantics of some 
features in the library could not be defined in the core language, 
but their syntactic representation can be. This is most evident
in the C programming language where, for example, Posix defines
semantics for multi-threading which are represented by library
functions which could not possibly be defined in C.

However in Felix the situation is even more extreme, since the very
language itself is defined in the library. What is more, most
of the definitions are real Felix code: we are not just
leveraging the library for syntactic modelling of semantics
we could not otherwise define, but are actually presenting
the definitions.

Naturally this is not possible without a core built-in parsing
system and semantics, this called the bootstrap parser
and the first stage abstract machine.

It reads the library syntax specifications to extend itself
to the full blown Felix language described in this document
and maps the input syntax onto terms of the abstract machine.


\section{Bootstrap Parser Grammar}
\href{http://felix-lang.org/share/src/compiler/flx_parse/flx_parse.dyp}{Compiler Reference}

The bootstrap parser is a hard coded parser which translates
Felix EBNF like syntax specification into a parser automaton,
which in turn translates Felix programs.

\subsection{Parser Processing}
Felix use the Dypgen parsing system, which is a run time
extensible GLR parsing engine with extensions. 

When the parsing of the production of a non-terminal is 
complete, the associated action code is executed.

The action code is a string containing a procedure
written in R5RS Scheme. The OCS Scheme language processor
is used for the processing.

Using predefined variables representing the s-expressions
associated with the evaluation of the actions associated
with the symbols of the production, the client action
code constructs a new s-expression representing the
intended AST for parsing that production.

\subsubsection{Example}
Here is a complete, self contained, working
example:

\begin{minted}{felix}
// define the DSSL Ceiling
syntax Ceiling {
  satom := "\lc" sexpr "\rc" =># 
    "`(ast_apply ,_sr (,(noi 'ceiling) (,_2)))"
  ;
}

// define the ceiling operation
fun ceiling (x:double) = "ceil($1)" 
  requires '#include <math.h>'
;

// make the DSSL available for use
open syntax Ceiling;

// Use it
println$ \lc 42.3 \rc;
\end{minted}

In the above example, a production for the \verb%satom%
non-terminal, recognises a the TeX like expression
\verb%\lc e \rc%, where the $e$ in the middle is
any expression. The Scheme code returns an s-expression
consisting of the \verb%ast_apply% symbol, followed by
the value in the current source reference variable 
\verb%_sr%, then a pair consisting of the function 
name \verb%ceiling% to be executed and its argument
\verb%_2% which is the s-expression returned from the
second position in the prodiction.

The Scheme function \verb%noi% converts a symbol
into an s-expression which is the AST representation
of a Felix name. The \verb%ast_apply% s-expression
is the AST representation of a Felix application.

We then define the function \verb%ceiling% 
in Felix by lifting it from the ISO C math library. 

Finally we have to open the DSSL so the new grammar
becomes active, and then when we write
and expression like \verb%\lc 42.3 \rc% the parser
translates it into a function application.

The Felix standard library actually defines 
\verb%\lceil% and \verb%\rceil% and maps
$\lceil e \rceil$ to the function \verb%ceil%
so we couldn't use them in the example.

\subsection{Scheme Actions}\index{scheme!action}\index{action!scheme}
Provided the returned AST is acceptable to subsequent
processing steps, this mechanism is safe provided the
executed Scheme code is purely functional.

If the Scheme actions have side effects, care must be taken.
Dypgen parsers may parse the same text more than once.
At any point in the parsing process the next symbol
is used to determine a possible set of parses, and all
these parses are then explored simultaneously.

Most of these parses will fail at some point in the
future, and the AST constructed at that point is discarded.
Except in special circumstances, if more than one parse
succeeds, the Felix translator will fail with a parsing
ambiguity.

The point, however, is that side effects from exploratory
parses cannot be discarded, nor can the order in which
the parses is performed be predicted. Although the parses
are, in principle, simultaneous, Dypgen actually processes
a single token for each parse in sequence, suspending the
parsing of one thread until all parses have processed
the token, then resuming with the next token, having
discarded failed parses, but also possibly spawning a new
set of parses. In other words, the parsing is performed
using coroutines which ensure that the input is only
consumed in a linear fashion without any backtracking.

If side effects are necessary, then they should usually
be idempotent. For example setting a variable to a fixed
value is an idempotent operation. Incrementing a variable is not.

There is, however, a special exception to this rule.
Top level statements are guaranteed to be parsed exactly once.
Although the start symbol is defined recursively, the recursion
is optimised to a plain loop. This is not necessarily the 
case for nested statements!

\subsection{Spawning Choices}
When presented with a non-terminal to parse, a GLR parser
notionally processes all alternatives simultaneously.
Many of these branches will fail, indeed, if the parse
is to be ambiguous all but one should eventually fail.
This method could be called purely speculative parsing.

However, Dypgen does not actually try all the alternatives.
Some alternatives must start with a particular token
or one of a set of tokens. Instead of starting the parse
and failing after examining the next token, Dypgen
uses an LALR1 predictive parser to narrow down the
set of alternatives. The LALR1 automaton is a finite
state machine which given the next token supplies
a set of possible parse which might continue.
Although LALR1 is not accurate, it is conservative,
and only rejects alternatives that could not possibly
work.

\subsection{No keywords}
In most language, there is a severe lack of usable symbols.
To compensate for this, certain identifiers called keywords
are specified as symbols instead. The tokeniser checks if an
identifier is a keyword and returns a suitable token if it is,
otherwise it returns an identifier.

Unfortunately, because keywords encroach on user space,
many languages such as C++ try to minimise the set used,
and in the case of a language extension, introducing a new
keyword may break existing code. So there is tendancy to
reuse the keyword in a different context, leading to
the nightmare that is evident in C++.

Felix also has keywords, but they are not global.
The Felix parser is scannerless which means it does not
require a separate tokeniser. Instead tokenisation
is local to each parser state, and in particular
the identifiers used as keywords are only recognised
in this state. Felix also provides a consistent way
to unkeyword with a special form of literal to force
the parser to recognise an identifier instead of a keyword.

\begin{minted}{felix}
// var = 1; 
   // Syntax error, variable init expected

var var = 1; 
  // OK, declares identifier "var" as a variable

n"var" = 1; 
  // allows assigning to identifier named "var".
\end{minted}

\subsection{Predefined Variables}
When scheme code is executed certain global variables are
automatically predefined. 

\begin{itemize}
\item The variable \verb%_sr% is set to contain the current
location in the source file. It consists of a scheme list
of the current original source filename as a string, then the 
starting line, starting column, ending line, and ending column
of the text being parsed by the current production.

\item The variables \verb%_1%, \verb%_2% etc refer to the 
scheme s-expressions associated with the parse of the n'th
symbol in a production.
\end{itemize}

Together these variables allow the scheme action code
of a production to construct an scheme s-expression representing
the AST desired for parsing the associated production.

\subsection{{\tt SCHEME} statement}
The keyword \verb%SCHEME% followed by a string literal and semi-colon
causes the embedded R5RS scheme interpreter to execute the text
of the string literal as scheme code during parsing.

It is primarily intended to introduce global procedure definitions
as helper functions to be used in the parser action codes.

However it may contain arbitrary Scheme code, including,
for example diagnostic output, an interpreter, or anything
else. The return value of the execution is discarded.

\begin{minted}{felix}
SCHEME """
  (display "Hello parsing world\n")
""";
\end{minted}

The \verb%SCHEME% statement may be used anywhere a statement is allowed,
it results in a empty statement in the AST. At the top level, the
side-effects will only be seen once. However if nested in a grouping
statement, multiple explorations of the possible parses of that
statement may occur, leading to multiple observations of the side-effects.

\subsection{{\tt STMT} statement}
The keyword \verb%STMT% followed by a string literal and semi-colon
causes the embedded R5RS scheme interpreter to execute the text
of the string literal as scheme code during parsing.

The result of the execution is returned and appended to the current
list of statements being parsed.

It is primarily intended to provide an mechanism whereby
arbitary Felix statements can be generated under program control,
in other words it is a very sophisticated and properly structured
macro processor. The client must have intimated knowledge of the
which s-expressions are accepted by the subsequent compiler
processing phases and what their semantics are.

Care must be taken if the result of the execution is not
invariant. For example if the time, a file from the file system,
or other such variable entity is used in the computation,
the Felix dependency checking system will not recognise
the external dependencies and the cached version of the current
file's parse may be used.

\begin{minted}{felix}
for i in 0..10 do
  println "Stopping now";
  STMT '`(ast_halt ,_sr "STOPPED")';
done
\end{minted}

\subsection{{\tt syntax} statement}
The \verb%syntax% statement consists of the word \verb%syntax%
followed by an identifier and a list of syntax definition statements
enclosed in curly braces \verb"{" and \verb"}". The entity
defined is called a {\em Domain Specific Sub-Language} or just
{\em dssl}.

The given definitions are recorded as a set under the given
name but are not activated. For example here is the
DSSL for the Felix macros:

\begin{minted}{felix}
syntax macros {
  requires expressions, statements, list;
  
  stmt := "macro" "val" snames "=" sexpr ";" =>#
    "`(ast_macro_val ,_sr ,_3 ,_5)";

  stmt := "forall" sname "in" sexpr "do" stmt* "done" =>#
    "`(ast_macro_forall ,_sr (,_2) ,_4 ,_6)"
  ;
}
\end{minted}

\subsection{{\tt requires} statement}\index{requires!grammar}
Within a dssl definition, the requires statement
consists of the word \verb%requires% followed by a comma
separated list of identifiers, and terminated by a semi-colon.

The names in the list identify other dssls on which this
dssl depends, so that subsequently activating the dssl
will also cause the dependent dssls, recursively, to be
activated.

\subsection{Priority statement}\index{priority!grammar}
The \verb%priority% statement introduces a set of priority
names which can be used for indexed non-terminals,
and specifies a total ordering between them,
sometimes called a precedence. Here are the priorities
used for non-terminal \verb%x% representing expressions
in the Felix grammar, lower priority means weaker binding:

\begin{minted}{felix}
  priority 
    let_pri < 
    slambda_pri <
    spipe_apply_pri <
    sdollar_apply_pri < 
    stuple_cons_pri <
    stuple_pri <
    simplies_condition_pri <
    sor_condition_pri <
    sand_condition_pri <
    snot_condition_pri <
    stex_implies_condition_pri <
    stex_or_condition_pri <
    stex_and_condition_pri <
    stex_not_condition_pri <
    scomparison_pri <
    sas_expr_pri <
    ssetunion_pri <
    ssetintersection_pri <
    sarrow_pri <
    scase_literal_pri <
    sbor_pri <
    sbxor_pri <
    sband_pri <
    sshift_pri <
    ssum_pri <
    ssubtraction_pri <
    sproduct_pri <
    s_term_pri <
    sprefixed_pri <
    spower_pri <
    ssuperscript_pri <
    srefr_pri <
    sapplication_pri <
    scoercion_pri <
    sfactor_pri <
    srcompose_pri <
    sthename_pri <
    satomic_pri
  ;
\end{minted}


\subsection{Non-terminal definition}\index{non-terminal!grammar}
A nonterminal definition, in its most basic
form, consists of the non-terminal name,
followed by \verb":=" followed by a list of alternatives
separated by a \verb"|" followed by a semi-colon \verb";".

An alternative is a sequence of grammar symbols 
followed by \verb"=>#" followed by a string.
The sequence of symbols is called the rhs (right hand side)
of the production, and the string is called the action.
The string must encode an R5RS Scheme procedure which returns
an s-expression representing the AST subtree desired to be inserted
in the syntax tree.

The non-terminal name on the left is either an identifier,
or, an identifier followed by \verb"[", an priority identifier,
followed by \verb"]". A priority identifier is an identifier
appearing in a priority specification.

An grammar symbol is either a predefined atom, a string
which is required to match the input stream, a non-terminal
expression, or a grammar expression enclosed in 
parentheses \verb"(" and \verb")", a macro invocation 
or a grammar symbol followed by one of \verb"*" representing 0 or more occurences,
\verb"+" representing one or more occurences, or
\verb"?" representing 0 or 1 occurence.

A non-terminal expression is either a name, or a name followed
by a priority indicator, or a macro invocation. A priority indicator is a priority
name enclosed in either \verb"[" and \verb"]" which indicates
a fixed non terminal, or \verb%[>% and \verb"]" which
indicates all the non-terminals with a greater priority
than the specified one.

A macro invocation is a name optionally followed by \verb"::" and another name,
followed by a sequence of macro arguments. 
A macro argument is a symbol enclosed in \verb"<" and \verb">".

\subsection{Preprocessing productions}
Felix pre-proceses productions to reduce them to
a sequence of basic symbols: either a terminal
or non-terminal symbol. 

\begin{itemize}
\item Expressions contained
in parentheses are replaced by a generated non-terminal
name and the non-terminal is defined by the enclosed
sequence, its action is to return a list of the ASTs
of each symbol.

\item A non-terminal postfixed by \verb%*% or \verb%+% is replaced
by a generated non-terminal defined to recursively generate
a list of the non-terminal, with either 0 or more or 1 or
more entries in the list, respectively.

\item A non-terminal postfixed by \verb%?% returns the a list
containing just the s-expression
returned by the non-terminal if it is matched, otherwise an
empty list.
 
\item Macro calls are expanded. This done by replacing each
occurence of a macro parameter with the correponding argument
of the call.
\end{itemize}

\subsection{Grammar Macros}\index{macros!grammar}
A grammar macro is defined by the same syntax as a non-terminal,
except that a sequence of parameter names is added on the lhs after
the first name. The first name may not be indexed.

On expansion due to an invocation of a macro, each parameter
name in the production is replaced by the non-terminal argument.

As an example, suppose we wish to define a comma separated list
of strings, a comma separated list of integers, and a comma 
separated list of names. By using a macro, the common structure
of the productions can be captured: a macro is basically 
a grammar function. The following is actually found in the
Felix library: 

\begin{minted}{felix}
syntax list 
{
  seplist1 sep a := a (sep a)* =># '(cons _1 (map second _2))'; 
  seplist0 sep a = seplist1<sep><a>;
  seplist0 sep a := sepsilon =># '()';
  commalist1 a = seplist1<","><a>;
  commalist0 a = seplist0<","><a>;

  snames = commalist1<sname>;
  sdeclnames = commalist1<sdeclname>;
}
\end{minted}

Although grammar macros were primarily intended as basic boilerplate
templates, macros can also be passed as arguments to another macro,
thereby allowing higher order macros. Furthermore, since arguments
are presented in a curried form, closures can also be created,
that is, macros with some but not all parameters fixed.

In particular, a macro can actually be passed to itself!
Although this seems weird it is in fact a very high power
technique used in functional programming and type systems,
known as {\em open recursion}.

Open recursion allows a recursive structure to be extended
so that the extensions {\em backport} to the core, in other
words, the extensions are covariant. This solves a significant
problem in grammar extension. Normally in a grammar extension
is easily done by simply adding extra productions for a non-terminal.

The problem is that the nonterminal is then effectively a global 
variable and extensions from different sources can clash. However
the use of a global automatically assures we have covariant extensions,
since existing productions using the non-terminal will automatically
parse the extensions since they're now alternatives of the non-terminal.

The problem with clashing is only a symptom of the deeper problem:
the mechanism does not obey Meyer's Open/Closed principle.

Open recursion solves this problem. The recursions are replaced
by parameters, flattening the term structure and allowing
extenions in a tree like manner. The programmer then closes
the loop at any desired point with a new definition formed by
passing the flat term to itself. 

More generally, a term structure with more details such
as a language with statements, value expressions, type expressions,
and patterns, allows any combination of the extensions to be
used with nested (recursive) structures equal to the top level
or constrained to some lower level as required.

This method obeys the Open/Closed principle because each closure
is specific and cannot be further extended, yet the open part of
the system remaims open for extension.

\subsection{{\tt regdef} statement}\index{regdef!grammar}
The \verb%regdef% statement defines an alias
for a regular expression. A regular expression
may consist of:

\begin{itemize}
\item An atom, being
\begin{itemize} 
\item an alias for a regular expression,
\item a string, representing that string, 
\item a string in
square brackets \verb"[" \verb%]% representing
one of the characters in the string.
\item an underscore \verb%_% representing any character
\item a dot \verb%.% representing any character except newline
\item a decimal integer representing the character with that ordinal
\item a range consisting of pair of decimal integers separated by a dash
and enclosed in square brackets \verb%[% and \verb%]% representing
one of the characters in the range, inclusive
\item a regular expression in parentheses \verb%(% \verb%)%
\end{itemize}
\item a repetition consisting of an atom optionally followed
by 
\begin{itemize}
\item a \verb%*% for zero or more occurences
\item a \verb%+% for one or more occurences
\item a \verb%?% for zero or one
\end{itemize}
\item a sequence of repetitions
\item A list of alternatives separated by a vertical bar \verb%|%
each being a sequence
\end{itemize}

\begin{minted}{felix}
  regdef octdigit = ['01234567'];
  regdef linefeed = 10;
  regdef hichar = [128-255];
  regdef white = space | tab;
  regdef bin_lit  = '0' ('b' | 'B') (underscore? bindigit) +;
\end{minted}

\subsection{{\tt literal} statement}\index{literal!grammar}
The \verb%literal% statement associates a non-terminal with
a regular expression.

The regular expression is formed by replacing all aliases
with their regular definitions, recursively, until no aliases
are left.

For example, from the library:

\begin{minted}{felix}
  regdef Cstring = cstring_literal;
  literal Cstring =>#
  """
    (let*
      (
        (ftype "cstring")
        (iv (decode-string _1))
        (cv (c-quote-string iv))
      )
      `(ast_literal ,_sr ,ftype ,iv ,cv)
    )
  """; 
  sliteral := Cstring =># "_1";
\end{minted}

Assuming \verb%cstring_literal% is an alias for the
regular definition for a C style string, we first
create a final alias with the name of the desired
non-terminal \verb%Cstring%.

Then we define \verb%Cstring% as a non-terminal with the
\verb%literal% statement, giving the required action code.
The layout shown is the standard form for literals in
the AST. the first field is the source reference \verb%_sr%,
the second field \verb%ftype% is the Felix type
of the literal. The third field is the internal representation
of the value. The last field is a string representing
the value in C++.

\subsection{{\tt open syntax} statement}\index{open syntax!grammar}
The \verb%open syntax% statement consists of the words
\verb%open syntax% followed by a comma separated list of names.

It simultaneously activates the transitive closure
with respect to \verb%requires% statements,
of dssls previously recorded by \verb%syntax% statements
identified by the list of names.


\section{Felix Grammar syntax}
\href{http://felix-lang.org/share/src/packages}{Library Package} 
\href{http://felix-lang.org/share/lib/grammar}{Grammar directory} 


% =====================================
\chapter{Modules}
Every Felix program is encapsulated in a module with
the name being a mangle of the basename of the root unit.
The mangling replaces characters in the filename with
other characters so that the module name is a valid
ISO C identifier.

\section{Special procedure {\tt flx\_main}}\index{flx\_main}
A program module may contain at most one top level
procedure with no arguments, exported as \verb%flx_main%. 
After initialisation code suspends or terminates, this procedure is invoked
if it exists. It is the analogue of \verb%main% in C++
however it is rarely used: side-effects of the
root unit initialisation code are typically used instead.

A simple example:

\begin{minted}{felix}
println "Init";
var i,o = mk_ioschannel_pair[int]();
write (o,42);

export proc flx_main()
{
  println$ "main " + (read i).str;
  println$ "done ..";
}

println$ "Init done";
\end{minted}

produces output:

\begin{minted}{text}
Init
main 42
done ..
Init done
\end{minted}

Note that \verb%flx_main% must be exported to ensure that
an \verb%extern "C"% symbol is created by the linker.

\section{Libraries}
In Felix a library is a root unit together with its
transitive closure with respect to inclusion,
which does not contain a top level exported \verb%flx_main%.

A program unit can be augmented by a set of libraries
which are then considered as if included, but without
an include directive being present.


% =====================================
\chapter{Lexicology}\index{utf8}\index{UTF-8}\index{Unicode}
All Felix files are considered to be UTF-8 encoded Unicode.

Felix uses a scannerless parser, there are no keywords.

\section{Comments}\index{comments!grammar}
There are lexical commenting methods for \verb%*.flx% files.
Comments are treated as white space separators. For example

\begin{minted}{felix}
println$ f/**/x; // parsed as f x not fx
\end{minted}

The two forms of lexical comments are exclusive, once the
parser is scanning one kind of comment the other is not
recognised.

\subsection{C++ comments}\index{C++ comments!grammar}
C++ style comments consist of \verb%//% followed by all the characters
up to and including the next newline charcter.

\subsection{Nested C comments}\index{C comments!grammar}\index{comments!nested}
C style comments consist of the lead in sequence
\verb%/*% followed by all the characters
up to and including the balancing exit sequence \verb%*/%. 
These comments can span
multiple lines and can be nested. When scanning comments
lead in and exit sequences are recognised as such even
in strings.

\section{Layout}\index{layout!grammar}
Felix treats code points 0 through 32 (space) as whitespace
which may be used freely between symbols. Whitespace is significant
in strings, however, and newline is a terminator for C++ style comments.

\section{File inclusion}\index{include!grammar}
There is (deliberately) no support in the Felix language 
for lexical (physical)  file inclusion. Inclusions are processed at 
the AST level instead, allowing files to be parsed independently of other files. 
However command line switches can be used to prepend files or sets
of files to the command input file, in particular the grammar
and some standard macros are notionally inserted.

\section{fdoc files}\index{fdoc}
As well as \verb%*.flx% files, the Felix language processor
can directly process \verb%*.fdoc% files using a limited
subset of available \verb%fdoc% commands.

\verb%fdoc% files are processed slighly differently to \verb%*.flx% files.
The translator begins treating the file in comment mode, so all text
is ignored up to a Felix leadin code.

\subsection{Uncomments}\index{uncomment}
A felix uncomment switches to processing lines as Felix program code.
It consists of the line \verb%@felix%, and is terminated by any
line starting with \verb%@%. 

\begin{minted}{text}
@title This is an fdoc.
@h1 Fdocs are documents.
They can contain code:
@felix
var x = 1;
@
Which defines a variable and
@felix
println$ x;
@
which prints it.
\end{minted}

\subsection{{\tt \#line} directive}\index{\#line!directive}
Felix provides support for programs that generate Felix code
by allowing C style \verb%#line% directives. Such a directive
consists of the characters \verb%#line% at the start of a line,
followed by whitespace, a decimal number indicating the line
number in the original source, and optionally whitespace
followed by a filename in double quotes. 

If the filename is present the parser
original source filename is set to it. The line number sets
the line number, 1 origin, so that the next source line
will be taken to be obtained from the original source file
at that line number.

Felix provides two standard programs which make use of this
facility: \verb%flx_tangle% and \verb%flx_iscr% both of
which are literate programming tools which extract source
code from mixed code and comments.

In the event of a compilation error, Felix will specify that
the error occured at a location in the original source
file, as indicated by \verb%#line% directives. An example:

\begin{minted}{felix}
#line 42 "anerror.fdoc"
var x = error;
\end{minted}

If this program is processed by Felix the error on the second
line will be reported as an error on line 42 of the 
file \verb%anerror.fdoc%.

\section{\#! directive}
If the first line of an \verb%*.flx% file starts with \verb%#!% then
the line is ignored. This allows a file on a Unix system marked
executable to specify its natural translator so that the file
may be run directly as a program. On Linux you should use:

\begin{minted}{felix}
#!env /usr/local/lib/felix/felix-latest/host/bin/flx
\end{minted}

assuming you have a standard install and have linked
\verb%felix-latest% to a directory containing an installed
version of Felix such as 

\begin{verbatim}
%/usr/local/lib/felix/felix-2016.05.25%.
\end{verbatim}

This is the standard way to refer to the most recent version of Felix
installed on Unix systems. Note the path name to the translator
on Unix systems must be an absolute path for security reasons.

\section{Identifiers}\index{identifier!grammar}

\href{http://felix-lang.org/share/lib/grammar/grammar_ident_lexer.fsyn}{Library Reference}

Felix has three kinds of basic identifiers, plain identfiers, which
are an enhanced variant of standard C identifiers, TeX identifiers,
which are encodings of mathematical symbols in the style of \TeX,
and some ascii-art character sequences normally use for punctuation
or operators which are also recognised as names.

\subsection{Plain Identifiers}
A plain identifier starts with a letter or underscore,
then consists of a sequence of letters, digits, dash (\verb%-%),
apostrophy (\verb%'%), has no more
than one apostrophy or dash in a row, except at the end no dash is
allowed, and any number of apostrophies.

\begin{minted}{felix}
Ab_cd1  a' b-x
\end{minted}

Identifies starting with underscore are reserved for the implementation.

A letter may be any Unicode character designated for use in an identifier
by the ISO C++ standard. In practice, all high bit set octets are allowed.
Identifiers are uniquely identified by their sequence of 
ISO-10646 (Unicode) code points, alternate encodings of the same 
glyph are distinct.

\subsection{TeX Identifiers}\index{identifier!TeX}
A TeX identifier starts with a slosh and is followed by a sequence
of letters. 

Here is a partial table of 
\href{http://felix-lang.org/share/src/web/ref/texops.fdoc}{TeX Symbols} 
recognised by the grammar as identifiers with undefined semantics
but pre-assigned kind and precedence.

\section{Operator Identifiers}\index{identifier!operator}
Felix allows some operators to be used as an identifier.
For example you can write:

\begin{minted}{felix}
fun +: int * int -> int = "$1+$2";
\end{minted}

to define addition on int in C. Symbols recognised by the parser
such as \verb%+% are usually mapped to functions with the same
name as the operator.

These operators are recognised as identifiers by the parser in
positions where an identifier is expected:

\begin{verbatim}
    +  -  *  /  %  ^ 
    \&  \|  \^  ~ 
    =  +=  -=  *=  /=  %=  ^=  &=  |=  <<=  >>= 
    <  >  ==  !=  <=  >=  <<  >> 
\end{verbatim}

\subsection{Special identifiers}\index{identifier!arbitrary}
The special string literal with a "n" or "N" prefix is a way to encode
an arbitrary sequence of characters as an identifer in a context
where the parser might interpret it otherwise.
It can be used, for example, to define special characters as functions.
For example:

\begin{minted}{felix}
typedef fun n"@" (T:TYPE) : TYPE => cptr[T];
\end{minted}

\section{Boolean Literals}\index{literal!bool}\index{bool!literal}
There are two literals of type \verb%bool%, namely \verb%true% and \verb%false%.

\section{Integer Literals}\index{literal!integer}\index{integer!literal}

\href{http://felix-lang.org/share/lib/grammar/grammar_int_lexer.fsyn}{Library Reference}

An plain integer literal consists of a sequence of digits,
optionally separated by underscores. Each separating
underscore must be between digits.

A prefixed integer literal is a plain integer literal
or a plain integer literal prefixed by a radix specifier.
The radix specifier is a zero followed by one of
the letters \verb%bB% for binary radix, \verb%oO% for octal
radix, optionally on may use \verb%dD% for decimal radix,
although this is the default, and \verb%xX% for hexadecimal radix.

An underscore is permitted after the prefix.

The radix is the one specified by the prefix or decimal
by default.

The digits of an integer consist of those permitted
by the radix: \verb%01% for binary, \verb%01234567%
for octal, \verb%0123456789% for decimal, \verb%0123456789abcdefABCDEF%
for hex.

Note there are no negative integer literals.

A type suffix may be added to the end of a prefixed
integer to designate a literal of a particular integer type,
it has the form of an upper or lower case letter or pair of
letters usually combined with a prefix or suffix \verb%u% or \verb%U%
to designate an unsigned variant of the type.  The allowed lower case
suffices are:

\begin{verbatim}
t  s  l  ll 
ut us u ul ull 
tu su   lu llu
i8 i16 i32 i64
u8 u16 u32 u64
   p  d  j 
zu pu du ju
uz up ud uj
\end{verbatim}

In addition, one or more letters may be upper case, except
that \verb%lL% and \verb%Ll% are not permitted.

There is a table of the types \fullref{Integer Types}.

Note the suffices do not entirely agree with C.

\section{Floating point literals}
\href{http://felix-lang.org/share/lib/grammar/grammar_float_lexer.fsyn}{Library Reference}

Floating point literals follow ISO C89, except that underscores
are allowed between digits, and a a digit is required both before
and after the decimal point if it is present.

The mantissa may be decimal, or hex, a hex mantissa uses a
leading 0x or 0X prefix optionally followed by an underscore.

The exponent may designate a power of 10 using E or e,
or a power of 2, using P or p.

A suffix may be F,f,D,d, L or l, designating floating type,
double precision floating type, or long double precision floating 
type.

\begin{minted}{felix}
123.4
123_456.78
12.6E-5L
0xAf.bE6f
12.7p35
\end{minted}

There is a table of the operators \fullref{Floating Operators}.

\section{String like literals}\index{literal!string}\index{string!literal}
\href{http://felix-lang.org/share/lib/grammar/grammar_string_lexer.fsyn}{Library Reference}

\subsection{Standard string literals}
Generaly we follow Python here.
Felix allows strings to be delimited by;
single quotes ',
double quotes ",
triped single quotes ''' or
tripled double quotes """.

The single quote forms must be on a single line.

The triple quoted forms may span lines, and include embedded newline
characters.

The complete list of special escapes is shown in table \fullref{String Escapes}.

\begin{table}
\caption{String Escapes\label{String Escapes}}\index{string!escape}
\centering
\begin{tabular}[c]{lll}
\multicolumn{3}{c}{Basic}\\
Escape&Name&Decimal Code\\
\hline
\verb%\a%&ASCII Bell& 7\\
\verb%\b%&ASCII Backspace&8\\
\verb%\t%&ASCII Tab&9\\
\verb%\n%&ASCII New Line&10\\
\verb%\r%&ASCII Vertical Tab&11\\
\verb%\f%&ASCII Form Feed&12\\
\verb%\r%&ASCII Carriage Return&13\\
\verb%\'%&ASCII Single Quote&39\\
\verb%\"%&ASCII Double Quote&34\\
\verb%\\%&ASCII Backslash&92\\
\multicolumn{3}{c}{Numeric}\\
\hline
\verb%\d%999&Decimal encoding&\\
\verb%\o%777&Octal encoding&\\
\verb%\x%FF&Hex encoding&\\
\verb%\u%FFFF&UTF-8 encoding&\\
\verb%\U%FFFFFFFF&UTF-8 encoding&\\
\end{tabular}
\end{table}

\subsection{Raw strings}
A prefix "r" or "R" on a double quoted string
or triple double quoted string suppresses escape processing,
This is called a raw string literal.

NOTE: single quoted string cannot be used, because this would
clash with the use of single quotes/apostrophies in identifiers.

\subsection{Null terminated strings}
A prefix of "c" or "C" specifies a C NTBS (Nul terminated
byte string) be generated instead of a C++ string.
Such a string has type +char rather than string.

\subsection{Perl interpolation strings}
A literal prefixed by "q" or "Q" is a Perl interpolation
string. Such strings are actually functions.
Each occurrence of \verb%$(varname)% in the string is replaced
at run time by the value "str varname". The type of the
variable must provide an overload of "str" which returns
a C++ string for this to work.

\begin{minted}{felix}
var x = 1;
var y = 3.2;
println$ q"x=$(x), y=$(y)";
\end{minted}

\subsection{C format strings}
A literal prefixed by a "f" or "F" is a C format string.

Such strings are actually functions.

The string contains code such as "%d" or other supported
C format specifiers. 

\begin{minted}{felix}
var x = 1;
var y = 3.2;
println$ f"x=%03d, y=%4.1f, s=%S" (x,y,"Hello");
\end{minted}

Variable field width specifiers "*" are not permitted. 

The additional format specification %S
is supported and requires a Felix string argument.

If \verb%vsnprintf% is available on the local platform it is used
to provide an implementation which cannot overrun.
If it is not, \verb%vsprintf% is used instead with a 1000 character
buffer.

The argument types and code types are fully checked for type safety.
There are some tables of accepted codes: 
\fullref{C format codes: integer},
\fullref{C format codes: special integer},
\fullref{C format codes: floating},
\fullref{C format codes: other}.

Please see a suitable reference to learn how to use C format codes.

\begin{table}
\caption{C format codes: integer\label{C format codes: integer}}
\centering
\begin{tabular}[c]{rll}
Code&Type&Radix\\
\hline
hhd&tiny&decimal\\
hhi&tiny&decimal\\
hho&utiny&octal\\
hhx&utiny&hex\\
hhX&utiny&HEX\\
\hline
hd&short&decimal\\
hi&short&decimal\\
hu&ushort&decimal\\
ho&ushort&octal\\
hx&ushort&hex\\
hX&ushort&HEX\\
\hline
d&int&decimal\\
i&int&decimal\\
u&uint&decimal\\
o&uint&octal\\
x&uint&hex\\
X&uint&HEX\\
\hline
ld&long&decimal\\
li&long&decimal\\
lu&ulong&decimal\\
lo&ulong&octal\\
lx&ulong&hex\\
lX&ulong&HEX\\
\hline
lld&vlong&decimal\\
lli&vlong&decimal\\
llu&uvlong&decimal\\
llo&uvlong&octal\\
llx&uvlong&hex\\
llX&uvlong&HEX\\
\end{tabular}
\end{table}


\begin{table}
\caption{C format codes: special integer\label{C format codes: special integer}}
\centering
\begin{tabular}[c]{rll}
Code&Type&Radix\\
\hline
zd&ssize&decimal\\
zi&ssize&decimal\\
zu&size&decimal\\
zo&size&octal\\
zx&size&hex\\
zX&size&HEX\\
\hline
jd&intmax&decimal\\
ji&intmax&decimal\\
ju&uintmax&decimal\\
jo&uintmax&octal\\
jx&uintmax&hex\\
jX&uintmax&HEX\\
\hline
td&ptrdiff&decimal\\
ti&ptrdiff&decimal\\
tu&uptrdiff&decimal\\
to&uptrdiff&octal\\
tx&uptrdiff&hex\\
tX&uptrdiff&HEX\\
\end{tabular}
\end{table}




\begin{table}
\caption{C format codes: floating\label{C format codes: floating}}
\centering
\begin{tabular}[c]{rll}
Code&Type&format\\
\hline
e&double&scientific\\
E&double&SCIENTIFIC\\
f&double&fixed\\
F&double&FIXED\\
g&double&general\\
G&double&GENERAL\\
a&double&hex\\
A&double&HEX\\
\hline
Le&ldouble&scientific\\
LE&ldouble&SCIENTIFIC\\
Lf&ldouble&fixed\\
LF&ldouble&FIXED\\
Lg&ldouble&general\\
LG&ldouble&GENERAL\\
La&ldouble&hex\\
LA&ldouble&HEX\\
\end{tabular}
\end{table}

\begin{table}
\caption{C format codes: other\label{C format codes: other}}
\centering
\begin{tabular}[c]{rll}
Code&Type\\
\hline

c&int (prints char)\\

S&string\\
s&\&char\\
p&address&hex\\
P&address&HEX\\
\end{tabular}
\end{table}


\chapter{Macro processing}
\href{http://felix-lang.org/share/lib/grammar/macros.fsyn}{Library Syntax Reference}

\href{http://felix-lang.org/share/src/compiler/flx_desugar/flx_macro.ml}{Compiler Semantics Reference}

\section{Macro val}
The macro val statement is used to specify an identifier should
be replaced by the defining expression wherever it occurs in an
expression, type expression, or pattern.

\begin{minted}{felix}
macro val WIN32 = true;
macro val hitchhiker;
macro val a,b,c = 1,2,3;
\end{minted}


\section{Macro for}
This statement allows a list of statements to be repeated
with a sequence of replacements.

\begin{minted}{felix}
forall name in 1,2,3 do
  println$ name;
done
\end{minted}


\section{Constant folding and conditional compilation}
\href{http://felix-lang.org/share/src/compiler/flx_desugar/flx_constfld.ml}{Compiler Semantics Reference}

Felix provides two core kinds of constant folding:
folding of arithmetic, boolean, and string values, and 
deletion of code, either statements or expressions,
which would become unreachable due to certain
value of conditionals.

Basic operations on integer literals, namely 
addition, subtraction, negation, multiplication,
division, and remainder are folded.

Strings are concatenated.

Boolean and, or, exclusive or, and negation,
are evaluated.

False branches of if/then/else/endif expression
and match expressions are eliminated.

False branches of if/do/elif/else/done 
are also eliminated.

By this mechanism of constant folding and
elimination, Felix provides conditional
compilation without the need for special
constructions.

\part{Lookup}

\chapter{Names}
\section{Kinds of Names}
Felix has several kinds of names which can find two kinds of entities.

\subsection{Simple Names}\index{name!simple}
A {\em simple name} is just an identifier.

\begin{minted}{felix}
  x
  joe
\end{minted}

\subsection{Simple Indexed Names}\index{name!indexed}
An {\em simple indexed name} is an identifier followed by an
open square bracket \verb%[%, a possibly empty comma separated list of type
expressions, followed by a close square bracket \verb%]%. If the list
is empty, the name is equivalent to a simple name. 

\begin{minted}{felix}
  x[] // equivalent to just x
  joe[int]
  fred[int,string]
\end{minted}

\subsection{Unqualified Names}\index{name!unqualified}
An {\em unqualified name} is a simple name or simple
indexed name. Because indexed names with an empty index list
are equivalent to simple unindexed names, all unqualified names
are notionally simple indexed names.

\subsection{Qualified Names}\index{name!qualified}
A {\em qualified name} is sequence of at least two unqualified names
separated by two colons \verb%::%.

\begin{minted}{felix}
  A::B::f
  A[int]::f[long]
  root::B
\end{minted}

\subsection{Special name {\tt root}}\index{root}
The special name \verb%root% is an alias for the top level module
name and may be used to force qualified name lookup  starting with the top
level module.

\subsection{Unsuffixed Name}\index{name!unsuffixed}
An {\em unsuffixed name} is a qualified or unqualified name.

\subsection{Suffixed Name}\index{name!suffixed}
A {\em suffixed name} is an unsuffixed name followed by a suffix consisting
of the word \verb%of% followed either by a simple name, or a possibly empty
comma separated list of type expressions enclosed in parentheses \verb%(% and
\verb%)%.

\begin{minted}{felix}
fun f (x:int) => x;
fun f (x:double, s:string) => x.str + s;
var closure = f of (double * string); // suffixed name
\end{minted}

Suffixed names are used to disambiguate references to functions
from an overloaded set.

\section{Symbol Tables}
Felix represents scopes using symbol tables. There are two kinds of symbol table.

\subsection{Symbol definition table}\index{table!symbol definition}
Every symbol defined is allocated a unique integer representative and 
the definition is stored in a hash table keyed by this integer, 
this is the {\em symbol definition table}. 
The definition includes a reference to the parent if any, and, a name lookup table if the
defined entity contains a scope.

\subsection{Name lookup table}\index{symbol table!lookup}
For each scope, another kind of hash table called a {\em name lookup table}
is used which maps a string name to a single entry which is one of two
kinds: either the entry is a non-function entry or a function entry.
Non-function entries contain a single view of a symbol, whilst 
function entries contain a list of views.

Each scope is associated with two name lookup tables.

\subsubsection{Private name lookup table}\index{symbol table!private}
This table maps all the names defined in the scope
to all the definitions in the scope. It is used for
unqualified lookup within the scope or any children of the scope.

\subsubsection{Public name lookup table}\index{symbol table!public}
This table contains mappings of the names of all
definitions in the scope which are not marked private
to those definitions. It is used for qualified lookup
into the scope from outside the scope.

\subsection{Views}\index{symbol!view}\index{view!symbol}
A {\em view} of a symbol consists of the index of the definition in the
symbol definition table together with a list of type variables and a list
of type expressions using these type variables, this list of
type expression must agree in number with the type variables 
in the entry in the symbol definition table.

The interpretation of a definition refered to by a view is as a polymorphic
definition indexed by the view type variables, with each use of the
definition's type variables in the body of the definition replaced
by the corresponding type in the view's list of type expressions.

For example:

\begin{minted}{felix}
class A[T] { fun f:T * T -> T; }
open[U, V] A[U * V];
var x = f ( (1,2.0), (3, 4.0) );
\end{minted}

This matches because the view of f available in
the open has type

\begin{minted}{felix}
(U * V) * (U * V)
\end{minted}

via the assigment \verb%T <- U * V%,
and the assignents
\verb%U <- int% and \verb%V <- long% mean the function call
is a specialisation of the view, which in turn is a specialisation
of the original function. The view provided by the open may be
described by a triple:

\begin{minted}{felix}
(index=A::f; vs=U,V; ts=U*V)
\end{minted}

where index is the integer index of f in the definition
table.

Multiple views of a function, even if overlapping, never create
an ambiguity: views are logically merged with non-discriminated
setwise union.

This is effected as follows by overload resolution.

\subsection{Name lookup environments}\index{environment!lookup}
A name lookup environment consists of a stack of pairs
of name lookup tables. The top of the stack represents
the current, innermost scope. This is where searching
for an unqualified name or first component of a qualified name begins. 
Each layer represents an including
or parent scope, up to the bottom of the stack,
which represents the outermost or global scope,
which has no parent.

The two name lookup tables at each level are the primary
name lookup table and the shadow name lookup table.

\subsubsection{Primary name lookup table}\index{symbol table!primary}
The primary name lookup table contains mappings for
all the symbols defined in the scope or injected
into the scope with a \verb%inherit% or \verb%rename% directives.

\subsubsection{Shadow name lookup table}\index{symbol table!shadow}\index{shadow!symbol table}
The shadow name lookup table contains mappings for
all the symbols made available by an \verb%open%\index{open!directive} directive
or by the \verb%with%\index{with!clause} clause of the type variable specification
of a definition.

\verb%open% directives make the public names defined
in, or inherited into, a class available via a view.

The name lookup algorithms\index{algorithm!lookup} always search the primary
lookup table first, if that fails they try the shadow
table. This ensures any definitions in a scope can be used
to hide names provided by an open, which ensures it is
possible to resolve ambiguities or modify semantics
for a particular name, whilst retaining the other names.

\section{Overload Resolution}\index{overload!resolution}
Overload resolution proceeeds as follows. We start with the result
of a function lookup which returns a set of views called candidates.\index{overload!candidate}
Our aim is to select a unique function, together with assignments to
the type variables of the definition, which cause the signature
of the call to equal the signature of the function, and any constraints
specified for the function to be satisfied.

\begin{itemize}
\item During overload resolution, a subset of the
candidates views is selected which match the supplied name and signatures.
\index{overload!signature}

\item Constraints are now applied to eliminate candidates additional candidates.
See \fullref{Type Constraints} for more details.\index{overload!type constraint}

\item The subset is then reduced by eliminating views which are strictly less
specialised than another. \index{overload!specialisation}

\item If the result of this reduction consists of a set
of more than one view\index{overload!view} of the same function, then they must map to the same
specialisation of the function, and overload resolution has succeeded. 
\end{itemize}

The algorithm is non trivial. Here is another example:

\begin{minted}{felix}
class A[T] { fun f (x:T) => x; }
open[U] A[U * int];
open[V] A[int * V];
println$ f (1,1);
\end{minted}

Here, there are two views of \verb$f$, {\em both} of which 
have \verb%T <-int% as a specialisation. The first views is accepted
because \verb%U <- int% shows the call signature \verb%int * int%
is a specialisation of signature \verb%U * int%, 
and the second because \verb%V <- int% shows the call signatures is
a specialisation of the signature \verb%int * V%.

Now, the first view provides the substitution \verb%T <- U * int%
and the second \verb%T <- int * V%, and substituting the respective
specialisations of the views by the call results in the same specialisation
of the original function \verb%T <- int * int%. Therefore this specialisation,
being unique, allows the function \verb%f% to be called directly with signature
\verb%int * int%.

Not all calls match all views. For example a call on signature
\verb%long * int% only matches the first view, and with \verb%int * long%
only matches the second: the point is that if more than one view
of the same function matches, it is never ambiguous because the views
always lead to the same specialisation of the function.


{\small This is not true if the views are views of distinct functions.
Something like this can and does happen when the inverse of view
lookup is performed when instantiating class virtual functions
with distinct instances. In this case, if there are two instances
which match, neither of which is most specialised, then Felix
will report an ambiguity, even though semantically the behaviour
is required to be the same: the semantic requirement is too sloppy
to ensure this is meaninful.}

Our conclusion is that the programmer is free to open overlapping
views of the same function, and in particular duplicated opens
do not cause a problem (the same consideration applies to non-functions).
{\em This is {\em not} the case when specifying type class instances,
where overlaps are associated with distinct definitions, so that
a single most specialised definition is required. This issue,
however, does not occur in ordinary (phase 1) lookup of names.
It occurs only during instantiation by monomorphisation when
a type class virtual must select from available instances.
In this case, there is no overloading to consider since the
particular function in each instance is already selected:
we only need consider the specialisation lattice. }

\section{Lookup}\index{lookup!name|textbf}
Names refer either to a function set or single non-function entity.
Finding the entity a particular use of a name refers to is called
{\em name lookup} or {\em name binding}.

Note that unlike C++, Felix has no automatic type conversions.\index{conversion!automatic}
Matching of signatures in Felix must be exact.\index{matching!exact}
(However see \fullref{Row Polymorphism}).

\subsection{Unqualified name lookup}\index{lookup!unqualified name|textbf}
Unqualified name lookup is performed using one of 
two base general algorithms, depending on the kind
of name required.

\subsection{Basic unqualified non-function lookup}
When seeking a non-function, Felix first looks for the
simple (non-indexed) part of the unqualified name in the
current scope.

If the name is found but it is a function symbol
the compiler aborts with a diagnostic message.

If the name is not found, it then looks in the special
shadow scope of the current context which contains symbols
inserted by \verb%open%\index{open!directive} or \verb%rename% directives.

If an \verb%open% or \verb%rename%\index{rename!directive} directive introduces
the same non-function name into the same shadow scope
and it refers to the same non-function, with the equivalent
indices, one entry is discarded to prevent an ambiguous
name error, when in fact both insertions are equivalent
and refer to the same entity.

The equivalent indices check works by testing for
type equality after alpha conversion. For example

\begin{minted}{felix}
class Unique[T] { typedef A = B; }
open[U] Unique[U * U];
open[V] Unique[V * V];
var x: A[int]; // OK
\end{minted}

If there are two occurences of the name in the shadow scope,
it aborts the compile with a diagnostic message.
Note: this error occurs whilst creating the shadow scope,
even if the symbol is not used.

If the name is still not found, Felix repeats the process in
the parent scope, if any, or aborts the compile with
a diagnostic if there is no parent.

\begin{minted}{felix}
var x = 1;
class A {
  var x = "hello";
  class C {
     fun f() => x; // A::x = hello
   }
}

begin
  fun f() => x; // root::x = 1
end

open A;
fun f() => x; // root::x = 1

begin
  open A;
  fun f () => x; // A::x = hello
end
\end{minted}

If the name is indexed, it then checks that the number 
and kind of indices matches the definition of the symbol
discovered, if not it aborts the compile with a 
diagnostic.

\subsection{Basic unqualified function lookup}
This is the same as non-function lookup, except that
the result is expected to be a set of functions
with exactly one member.

\begin{minted}{felix}
fun f() => 1;
var closure = f; // only one f, OK
\end{minted}

\subsection{Lookup with signatures}\index{lookup!signature}
Felix can find a list of signatures for a function
lookup in two contexts.

If the name is a suffixed name\index{lookup!suffixed name}, the suffix supplies
the sole signature:

\begin{minted}{felix}
fun f (x:int) => x;
fun f (x:double) => x;
var closure = f of (double);
\end{minted}

In an application or call the argument or arguments supply
the signatures:

\begin{minted}{felix}
fun f (x:int) => x;
fun f (x:double) => x;
var xi = f 1; // f of (double);
var xd = f 1.0; // f of (double);

fun g(x:int) (y:int) => x + y;
fun g(x:int) (y:double) => x.double + y;

var yi = g 1 2; // signatures int,int
var ydi = g 1 2.0; // signatures int,double
// var closure = g 1; // ambiguous, error
\end{minted}

Care must be taken designing functions to reduce
possible ambiguities. Using curried arguments has advantages
but one disadvantage is that eta-expansion may be required
to select the right overload. A famous example from the 
Felix library:

\begin{minted}{felix}
class List {
  fun map[T,U] (f:T->U) (ls: list[T]) : list[U];
}
class Varray {
  fun map[T,U] (f:T->U) (ls: varray[T]) : varray[U];
}

open List;
open Array;
fun itod (x:int): double => x.double;
// var closure = map itod; // Error, which map?
\end{minted}


\subsection{Unqualified function lookup with signatures}\index{lookup!signature}
\index{lookup!unqualified!function}
When seeking a function with signatures, 
felix follows similar rules to non-functions, except as follows. 

When a function symbol is found, it consists of a set of functions,
not a single function. Felix then performs overload
resolution to determine if any of the candidates matches.

If there is no match at all, it proceeds with the next step.
If there is more than one match, and no match is most specialised,
it aborts with a diagnostic message.

Lookup in the shadow scope is performed similarly.\index{lookup!shadow}
Function names introduced from multiple opens are merged
prior to the shadow scope lookup. This merger ensures that
if the same name is present in two opens, refers to the same
original symbol, and has the same type indicies, one is discarded.
Note: this merge of equivalent names is not done when constructing
the shadow scope, but during overload resolution. Therefore it
is not an error to introduce the same name with the same signature
for distinct functions, it is only an error if there is an attempt
to use such a name. This is different from non-function lookup.

\begin{minted}{felix}
fun f(x:int) => x; // signature int
class A {
  fun f(x:double) => x; // signature double
  fun g() {
    return f 1; // signature int: finds root::f of (int)
  }
}
begin 
  open A;
  fun f(x:string) => x; // signature string;
  var x = f 1.1; // signature int: finds A::f of (double)
end
\end{minted}

For the first lookup, there is no f in g.
There is no shadow scope.
The f in A has signature double which doesn't match.
The f in root has signature int which matches.

For the second lookup we want an f with signature double.
The f in the begin/end scope does not match.
The f in the begin/end shadow scope introduced by the open
does match so A::f of (double) is found.

Note that this algorithms is not the same as C++.
Felix has fixed the design fault.

   
\subsection{Qualified non-funtion name lookup}\index{lookup!qualified!non-function}
Qualified name lookup for a non-function proceeds by
first performing an unqualified name lookup on the first
component of the qualified name, ignoring indices.

If the name is not found the compile aborts and a diagnostic
is issued. If the name is not a class name, the compiler aborts
and a diagnostic is issued.

Now, the next component is searched for in the
specified class scope. If the name is not found, the compile
aborts and a diagnostic is issued. The search ignores shadow
scopes, and it does not proceed into the context of the
classes parent. However it does take inherited names into account.
All names except the final one must resolve to class names.

The search a non-ultimate name requires a class to be found
in the previously found class. Qualified name lookup, therefore,
drills down into a class heirarchy from the root.

The final name is sought similarly and must be of the
kind sought.

Finally the complete sequence of type index parameters is checked
for length against the entity's definition.

\subsection{Qualified function name lookup}\index{lookup!qualified!function}
Searching for a qualified name of a function is the same
as for a non-function, except that a set of overloaded
functions may be returned. 

If a unique symbol is required, then the set must contain
only one function name.

\subsection{Qualified lookup for function with signatures}
A standard qualified name lookup is performed so that the
last component returns a function set. 

Overload resolution is then performed on the function set.
The result is final.

\chapter{Libraries}\index{library!statement}
The \verb%library% statement may be used to form extensible
groups of named definitions.

\begin{minted}{felix}
library A {
  proc f(x:int) => println$ "A::f " + x.str;
}
library B {
  proc f(x:int) => println$ "B::f " + x.str;
}
library A {
  proc g(x:int) => println$ "A::g ";
}

open A;
f 1;
B::f 1;
\end{minted}

Unlike classes and modules, libraries are extensible
and so may span multiple files (however the library 
statement may not!).

Libraries cannot be polymorphic, and may not contain
virtual functions or procedures directly.

Libraries must be at the top level or nested in
other libraries.

Statements within a library statement are executed in
order. Statements in multiple library statements for
the same library in the same file are executed in order.

Statements in a library statement nested in a library statement
are executed before any statements in the enclosing library statement.

Top level code not enclosed in a library statement is treated
as if enclosed in an anonymous library.

Otherwise the order of execution is unspecified.

\chapter{Abstract Types}\index{type!abstract|textbf}
An abstract type is a type for which a specific implementation
is given, but which can only be accessed by specified methods.

\subsection{Syntax}
An abstract type is created like this, taken
from the standard library. First define the type
in terms of another type:

\begin{minted}{felix}
open class Darray {
  private struct darray_ctl[T]
  {
    a: varray[T];
    resize: size * size --> size;
  }
  type darray[T] = new &darray_ctl[T];
\end{minted}

Now, we provide a constructor for the abstract type:

\begin{minted}{felix}
  //$ Make an empty darray, give it 20 slots 
  //$ for no particular reason.
  ctor[T] darray[T] () => 
    _make_darray[T]$ new darray_ctl[T](
      varray[T] 20uz , dflt_resize
  );
\end{minted}

We use the special name \verb%_make_darray% for the constructor.
It constructs a value of the abstraction implementation secretly
and casts it to the abstract type.

Now we need some methods defined on the abstract type:

\begin{minted}{felix}
  // resize
  proc do_resize[T] (pd: darray[T], new_size: size)
  {
    var old = (_repr_ pd)*.a;
    (_repr_ pd).a <- 
      varray[T] (new_size, (len old), (fun(i:size)=>old.i));
  }

  // pop head
  proc pop[t](a:darray[t]) {
    pop (_repr_ a)*.a;
    newsize := (_repr_ a)*.resize 
      (maxlen (_repr_ a)*.a, len (_repr_ a)*.a)
    ;
    if newsize != maxlen (_repr_ a)*.a 
      call do_resize (a,newsize)
    ;
  }
} // end Darray class
\end{minted}

These methods use the special name \verb%_repr_% to cast the abstract
type to its implementation.

So, the statement

\begin{minted}{felix}
  type darray[T] = new &darray_ctl[T];
\end{minted}

defines a new, abstract, polymorphic type, \verb%darray[T]%
implemented as a pointer to a \verb%struct% containing an 
bounded array \verb%varray% and a \verb%resize% function.

In order to construct a \verb%darray%, we first create pointer
to a \verb%darray_ctl% objects, and then cast it to the
abstract type using special name \verb%_make_darray%.
The cast from the representation to the abstract type is the
name of the abstract type constructor prefixed by \verb%_make_%.

To access the implementation, which is needed to define methods
on the abstract type, we use the cast \verb%_repr_%, which 
converts the abstract type back to its implementation.

Both these casts are private to the containing class \verb%Darray%,
so they can only be used in the class defining the abstract type,
because private names of a class are only accessible within
the class (and it descendants of course).

Therefore, the mechanism provides a way to rigorously protect
the implementation of the abstract type. The privacy of
the representation type in the example is not necessary, but is used
to prevent namespace pollution.



\chapter{Classes}
\href{http://felix-lang.org/share/lib/grammar/namespaces.fsyn}{Syntax}

\section{Class Statement}\index{class!statement|textbf}
The top level Felix module can contain submodules 
which are specified by a non-polymorphic class
statement:

\begin{minted}{felix}
class classname { ... }
\end{minted}

\subsection{Qualified Names}\index{name!qualified}
The effect is to produce a qualified name to be used
outside the class:

\begin{minted}{felix}
class classname { proc f () {} }
classname::f (); 
\end{minted}

\subsection{Nested Classes}\index{class!nested}
Classes may be nested.

\begin{minted}{felix}
class A { 
  class B {
    proc f () {}
    f(); 
  }
  B::f();
}
A::B::f (); 
\end{minted}

\subsection{Private definitions}\index{definition!private}
A class may contain private definitions,
if a symbol is marked private it is only visible
in the scope class in which it is defined,
including any nested classes:


\begin{minted}{felix}

class A { 
  private proc f () {}
  class B {
    f(); // visible inside A 
  }
  f(); // visible inside A
}
// A::f () fails, f is not visible outside A
\end{minted}

\subsection{Single File Rule}
A class must be specified within a single file.

\subsection{Completeness Rule}
Classes are not extensible, a definition of a class
with the same name in the same scope is not permitted.

\subsection{Setwise Lookup}\index{lookup!setwise}
The body of a class forms a nested scope. Within
a class all symbols defined in the class are visible,
along with all those visible in the enclosing context.
This is called \gls{setwise lookup}.

\begin{minted}{felix}
var x = "Hello";
class A {
   proc f() { g(); } // g visible before defined
   proc g() { x = x + 1; } 
   var y = f(); // calls g which uses x uninitialised!
   var x = 1;
}
\end{minted}

Beware using uninitialised variables! The procedure \verb%g%
above uses \verb%A::x% before it is initialised.

\subsection{Top level module alias}
The reserved name \verb%root% may be used as a prefix
for the top level module:

\begin{minted}{felix}
var x = 1;
class A { var x = root::x; }
\end{minted}

\section{Polymorphic Classes}

\subsection{Parametric Polymorphism}\index{polymorphism!parametric|textbf}
A class may be specified with one or more type variables.
Such a class is said to be a polymorphic class.

\begin{minted}{felix}
class MakePair[T,U] {
  fun fwd_pair (x:T) (y:U) => x,y;
  fun rev_pair (x:T) (y:U) => x,y;
}
\end{minted}

\subsection{No Variables}
Polymorphic classes may not directly contain variables,
all members must be function or type definitions.

\subsection{Indexed Names}

Entities from a class can be specified from
outside using a qualified name in which the
class name is followed a square bracketted list
of types instantiating the corresponding type
variables, this is an {\em indexed name}:

\begin{minted}{felix}
var f = MakePair[int,string]::fwd_pair 42 "Hello";
var r = MakePair[int,string]::rev_pair 42 "Hello";
\end{minted}

\subsection{Deduced indices}
However one or more or even all the types can be omitted
from the right end of the list if they can be deduced by 
overload resolution. 

\subsection{Elision of indices}
The square brackets are optional if they enclose an
empty list.

\begin{minted}{felix}
f = MakePair[int,string]::fwd_pair 42 "Hello";
f = MakePair[int]::fwd_pair 42 "Hello";
f = MakePair[]::fwd_pair 42 "Hello";
f = MakePair::fwd_pair 42 "Hello";
\end{minted}

Note that all symbols in Felix are considered to be
indexed. So for example this is correct:

\begin{minted}{felix}
var x = 1;
println$ x[];
\end{minted}

\subsection{Sloppy indexing}
It is also permitted to move the instance types across
qualification boundaries:

\begin{minted}{felix}
f = MakePair[int]::fwd_pair[string] 42 "Hello";
\end{minted}

although this practice is not recommended.

From the outside, class type variables are
viewed as \href{http://en.wikipedia.org/wiki/Universal_quantification}{univeral quantification}.
From inside however, the variable names are
fixed types which happen to be unknown, that
is they're to be considered as \href{http://en.wikipedia.org/wiki/Existential_quantification}{existentials}.

This means they cannot be specified explicitly
or implicitly inside the class with unqualified access:

\begin{minted}{felix}
class A[T] {
  fun f(x:T) => x,x;
   fun g(y:T) => f y; // correct
   fun h(y:T) => A[T]::f y; // correct
   // fun h(y:T) => f[T] y; // wrong
}
\end{minted}

From inside the class the function \verb%f% above is
monomorphic, not polymorphic. Although \verb%T% is not
known, it is a fixed, invariant type.

\subsection{Polymorphic Members}
Functional members of a class may be polymorphic.
This is independent of the class type parameters.

\begin{minted}{felix}
class A[T] {
  fun f[U] (x:T, y:U) => x,y;
  fun g(a:T) => f[T * T] (a,a);
\end{minted}

In this case the function \verb%f% has its own personal
type variable \verb%U% which is set in the call in 
\verb%g% to \verb%T * T% because its argument is a pair
of \verb%a% which has that type.

\section{Virtuals and instances}
A polymorphic class may contain functions or
procedures marked \verb%virtual%. In this
case a definition is optional.

\begin{minted}{felix}
class A[T] {
  // non virtual function
  fun f(x:T) => x,x; 

  // virtual function with definition
  virtual fun g (x:T) => f x;

  // virtual function without definition
  virtual fun h : T -> T * T;
}
\end{minted}

If a virtual function without a definition from a class is used, it must be
defined for the types for which it is used. This is done in an
instance statement:

\begin{minted}{felix}
var a = 1; // int
var x = A[int]::h a; // h used for T=int

// instance defining h for int
instance A[int] {
  fun h(a:int) => a + 1, a - 1;
}
\end{minted}

An instance can be defined anywhere in a non-functional scope
provided the class being instantiated is visible.

In this case symbols in the instance definition are bound
the context of the instance scope and its enclosing scope
in the usual way.

It is not possible to directly access the definitions
in an instance from outside the instance. Instead all
access is diverted through the top level class being
instantiated. Symbols are bound to the top level class
member first to fix the type variables, then a matching
instance is sought. This second lookup is called {\em phase 2 lookup},
it occurs during monormophisation.

If there is no matching instance, the program
fails with a diagnostic.

If there is more than than one matching instance,
the most specialised instance is used. If there
is no most specialised instance, the use of the class
member is ambiguous and the program fails with
a diagnostic.

An instance must be not less specialised than the class
it instantiates (that is, equally specialised or more specialised).

Instance specialisation is judged by the polymorphic subtyping
rule which is implemented using a variant of the
unification algorithm in which dependent and independent types
are distinguished.

A polymorphic type S[T] parametrised by a type variable T is more
specialised than a polymorphic type A[U] if U may be
replaced by a type expression containing T the result of which
equals S[T].

If S has more than one type variable, substitutions for each
of them must be found. If A has more than one type variable,
the substitutions involve all of these variables.

A specialisation S1 of A is strictly more specialised than a specialisation
S2 of A if S1 is also strictly a specialisation of S2.

\subsection{Default Definition}
If a virtual function has a definition, it is called
a {\em default definition}. On a use of a function with
a default definition, if no matching instance is found, 
the default definition is specialised and used instead.
Note that default definitions do not resolve ambiguity
from overlapping instances.

\subsection{Matching polymorphic functions}
If a virtual function is polymorphic, the personal
type variables of the function must not be specialised
in an instance. The class type variables are resolved
independently of the function type variables.

You may think of this as if the class type variables
are resolved first by finding a uniquely matching
instance, and then the function type variables are
resolved in a second step (although in fact there
is no coupling between the resolutions).

\section{Classes as Categories}
A class may thought of in a category theoretic sense.

A base category constists of objects which consists of a 
finite set of base types and all combinations thereof formed
by the available set of type combinators such as tuples,
records, function types, etc. 

The arrows of this category
consist of all the functions explicity defined on these
types, and all compositions thereof. This includes the
identity functions implicitly defined by copying.

This is the category generated by the graph consisting
of the defined functions, where paths in the graph
are (reverse) function compositions. This is called
the free category generated by the graph.

A class in the context of a base category generates
an extension to that category by throwing in some
additional types and arrows. The type variables are added 
to the set of base types, so that the set of combinations
is extended. The functions are added to the set of digraph
edges of the base generating graph, to add additional
arrows.

This view of the role of a class is vital to understand
semantic rules which correspond to the notion of a
category generated by a graph with relations. The relations
form a constraint. In category theory, the relations are 
generally equations relating the arrows which specify that
certain function compositions are equal to others.

The freely generated category is sometimes called an
initial algebra. When constraints are added a new
category is formed by collapsing some sequences of
arrows to a single arrow. There is a mapping called
an epi-morphism which takes each arrow from the freely
generated category to the target so that two distinct
composites may map to the same arrow in the target. 

Conversely, there is a reverse mapping from the arrows
of the target category to sets of arrows of the freely
generated one, forming another category called a quotient
category. These two mappings are functions and together
form an adjunction.

The purpose of this explanation is not merely for your
entertainment. In most programming classes are intended
to be constrained, that is, the instances are intended
to have a certain behaviour. 

Consider for example:

\begin{minted}{felix}
class Addable [T] {
  virtual fun add: T * T -> T;
}
\end{minted}

Now consider the following instances:

\begin{minted}{felix}
instance Addable[uint] {
  fun add (x:uint, y:uint) => x + y;
}

instance Addable[int] {
  fun add (x:int, y:int) => x + y;
}

instance Addable[double] {
  fun add (x:double, y:double) => x + y;
}
\end{minted}

These instances are good according to the specification,
however the addition of \verb%int% and \verb%double%
is unspecifed because we may have overflow. Furthermore
one may get a big surprise adding floating point representations
of reals: if you add a big float to a small one the small one
may be so small that the result is equal to the big one.

Basic laws of arithmetic are not obeyed by
floating representations, in particular addition of
floats is not associative!

It is usual to specify requirements in comments and
the programmer must carefully check that instances
obey the constraints specified for the class in these
comments.

Felix, however, can do a bit better.

\section{Axioms, Lemmas and Theorems}
Felix allows some semantic constraints on class instances
to be specified with explicit axioms. For example:

\begin{minted}{felix}
class Addable [T] {
  virtual fun add: T * T -> T;
  axiom associative (x:T, y:T, z:T):
    add (add (x,y), z) == add (x, add (y, z))
  ;
}
\end{minted}

The axiom documents in a formal language the requirement
that the addition operator defined in an instance
must be associative. An axiom is similar to an ordinary
boolean function which tests whether some condition is
met or not. By applying an axiom to particular values
we can check for non-conformance of instances.

However, in a categorical sense, an equational axiom
plays precisely the role of a relation in the formation
of a category with generators and relations. It is a vital
part of practical specification of semantics.

We should note that not all constraints can be expressed
as axioms in Felix. In particular since Felix only provides
first order polymorphism, only equational predicates
in Horn form can be used, that is, all the quantifiers
must be (implicitly or explicitly) on the left of the formula.
Interior quantifiers require second order polymorphism.

Note also that like most programming languages we use
constructive mathematics which constrains the role
of existentials to calculations. That is, it is not
acceptable to assert that something exists because
if it did not a contradiction would be derived.
Indirect proof is not allowed in constructive mathematics.

Instead, one must prove something exists by actually 
constructing one! So for example given $a < b$ we
know there exists a number $x$ such that $a < x < b$.
A proof of this by asserting that if no such $x$
existed, then we would have a contradiction, is
not allowed in constructive mathematics. Instead
the proof is simply given by $x = (a + b) / 2$.
And of course .. the assertion is false for many types
such as integers!

\subsection{Lemmas and Theorems}
Lemmas and theorems play the same role as axioms.
The idea, however, is that lemmas and theorems
can be derived from the axioms. A lemma is a simple
rule which a good automatic theorem prover could 
derive, and which seems obvious to a human.

A theorem is a more difficult rule which would
require a proof assistant with hints to derive a proof.

\subsection{Reductions}
A reduction is also an axiom, but it has an additional
property. Instead of an equation, reductions use a directed
equality:

\begin{minted}{felix}
  reduce revrev[T] (x:list[T]):
    rev (rev x) => x
  ;
\end{minted}

This reduction asserts that if you reverse a list twice
you end up back where you started. But it does more:
it give the compiler a hint that it is allowed to replace
an expression which reverse a list twice with the list,
removing the two reversing operations.

This is a particularly interesting example of the general
rule that all theorems are nothing more than optimisations,
in this case literally allowing the compiler to optimise
a program. Note that it is only a hint!

\section{Lookup control directives}

\subsection{Open directive}
The simple \verb%open% directive may be used to make the symbols
defined in a class visible in the scope containing the \verb%open% directive.

\begin{minted}{felix}
class X { var x = 1; }
open X;
println$ x;
\end{minted}


Names made visible by an open directive
live in a weak scope under the current scope.
Names in the weak scope may be hidden by definitions
in the current scope without error.

\begin{minted}{felix}
class X { var x = 1; }
open X;
var x = 2;
println$ x; // prints 2
\end{minted}


The open directive is not transitive.
The names it makes visible are only visible
in the scope in which the open directive is written.

\subsection{Inherit directive}
The inherit directive allows all of the public symbols
of a class to be included in another scope as if they
were defined in that scope. This means such names
inherited into a class can be accessed by qualification
with the inheriting class name, and will be visible
if that class is opened. 

Inheriting is transtitive.

If a name is inherited it will clash with a local definition.

\begin{minted}{felix}
class A { var a = 1; }
class B { inherit A; }
println$ B::a;
\end{minted}


\subsection{Rename directive}
This directive is can be used to inherit a single
symbol into a scope, possibly with a new name,
and also to add an alias for a name in the current
scope.

When applied to a function name all functions with
that name are renamed.


\begin{minted}{felix}
class A { 
  var a = 1; 
  proc f() {} 
  proc f(x:int) {} 
}

class B { 
  rename a = A::a;
  rename fun f = A::f;
}
\end{minted}

The new name injected by a rename may be polymorphic:

\begin{minted}{felix}
class A { proc f[T] () {} }
class B { rename g[T] = A::f[T]; } 
\end{minted}

\subsection{Use directive}
This is a short form of the rename directive:

\begin{minted}{felix}
class A { var a = 1; }
class B { use A::a; use b = A::a; }
\end{minted}

It cannot be applied to functions. The first
form is equivalent to

\begin{minted}{felix}
use a = A::a;
\end{minted}

Unlike the rename directive the new name cannot be polymorphic
and is limited to a simple identifier.

\part{Type System}
\chapter{{\tt typedef} statement}\index{typedef}
\href{http://felix-lang.org/share/lib/grammar/type_decls.fsyn}{Syntax}

\section{Type aliases}
The plain \verb%typedef% statement is used to define an alias for a type.
It does not create a new type.

\subsection{Basic typedef}

\begin{minted}{felix}
typedef Int = int;
\end{minted}

\subsection{Polymorphic typedef}
Typedef can also be polymorphic:

\begin{minted}{felix}
typedef Cplx[T] = {x:T,y:T};
\end{minted}

\section{Type Sets}
A type set is a component of a type constraint expression
which restricts a type variable one of a specifed list of entities. 
Each entity may be either a type, a type predicate, or another
type set.  Syntax like:

\begin{minted}{felix}
typesetof (short, int, long, vlong)
\end{minted}

is used for a typeset expression.

The typedef statement may be used to define an alias for
a typeset:

\begin{minted}{felix}
typedef Integers = typesetof (short, int, long, vlong);
\end{minted}

A typeset can be used with a membership test or kinding specification
to form a constraint. For example:

\begin{minted}{felix}
fun f[T where T in Integers] (x:T) => x;
\end{minted}



See also \fullref{Integer Typesets}, \fullref{Typeset membership constraints}.

\subsection{basic type functions (functors)}
Typedef can also be used to create a type functions, or {\em functors}.
A basic definition is like this:

\begin{minted}{felix}
typedef fun pair0 (T:TYPE, U:TYPE) : TYPE => T * U;

var x0 : pair0 (int, long) = 1,2L;
println$ x0;
\end{minted}

The kind \verb%TYPE% is the meta-type or kind of all types.
The return kind must be specified for functors.

\subsection{type functions with typematch}
A more sophisticated defintion allows a typematch:

\begin{minted}{felix}
typedef fun pair1: TYPE * TYPE -> TYPE =
| int,_ => int
| _,int => int
| ?T, ?U => T * U
;

var x1a : pair1 (int,double) = 1;
var x1b : pair1 (long, double) = 1L,2.0;
println$ x1a,x1b;
\end{minted}

See also \fullref{typematch}.

\subsection{Higher order functors}
You can also have curried type functions:

\begin{minted}{felix}
typedef fun pair2 (T:TYPE) (U:TYPE) : TYPE => T * U;

var x1 : pair2 int long = 1,2L;
println$ x1;

typedef pair3[T] = pair2 int T;
var x3 : pair3 [double] = 1,2.0;

println$ x3;
\end{minted}

Type functions can be overloaded just like ordinary functions.

\subsection{Functors in classes}
Note carefully there is a difference between an {\em indexed type}
and a {\em type function}. In the above example, \verb%pair3%
is not a functor, its an indexed type. Functors can be passed 
to type classes, for example:

\begin{minted}{felix}
class Monad [M: TYPE->TYPE] {
  virtual fun bind[a,b]: M a * (a -> M b) -> M b;
  virtual fun ret[a]: a -> M a;
}

typedef fun Maybe (t:TYPE):TYPE=>opt[t];

instance Monad [Maybe] {
  fun bind[a,b] (x:Maybe a, f:a -> Maybe b) =>
    match x with
    | #None[a] => None[b]
    | Some x => f x
    endmatch
  ;

  fun ret[a](x:a):Maybe a => Some x;
}
\end{minted}

The \verb%M% above accepts a functor, \verb%Maybe% in the instance.
We can't use \verb%opt% directly because it is an indexed type,
not a functor.

\chapter{Products}
\section{Tuples}
Tuple types are well known: a tuple is just a Cartesian Product
with components identified by position, starting at 0. 
The n-ary type combinator is infix \verb%*% and the n-ary value
constructor is infix \verb%,%:

\begin{minted}{felix}
val tup : int * string * double = 1, "Hello", 4.2;
\end{minted}

\subsection{Unit Tuple}
The 0-ary tuple type is denoted \verb%1% or \verb%unit%
with sole value \verb%()%:

\begin{minted}{felix}
val u : unit = ();
\end{minted}


\subsection{No 1-ary Tuple}
The 1-ary tuple of type \verb%T% component value \verb%v% is identified
with the type \verb%T% and has value \verb%v%.

\subsection{Projections}
The individual components of a tuple may be accessed by a projection
function. Felix uses an integer literal to denote this function.

\begin{minted}{felix}
var x = 1,"Hello";
assert 0 x == 1; assert x.0 == 1;
assert 1 x == "Hello"; assert x.1 == "Hello";
\end{minted}

A pointer to a tuple is also in itself a tuple, namely the
tuple of pointers to the individual components. This means
if a tuple is addressable, so are the components.

\begin{minted}{felix}
var x = 1, "Hello";
val px = &x;
val pi = px.0; pi <-42;
val ps = px.1; ps <-"World";
assert x.0 == 42;
assert x.1 == "World";
\end{minted}


In particular note:

\begin{minted}{felix}
var x = 1, "Hello";
&x.0 <- 42;
\end{minted}


because the precedences make the grouping \verb%(&x).0%.

You cannot take the address of a tuple component because
a projection of a value is a value.

Assignment to components of tuples stored in variables is supported
but only to one level, for general access you must take a pointer
and use the store-at-address operator \verb%<-%.

\subsection{Standalone Tuple projections}
The projections of a tuple can also be written in an expanded form
so that they may stand alone as functions:

\begin{minted}{felix}
var first = proj 0 of (int * string);
var a = 1, "Hello";
var one = a . first; 
var two = a . proj 1 of (int * string);
\end{minted}

\subsection{Alternate Tuple Representation}
Felix provides a second representation of tuples.\
Given a tuple of at least two components, another
value may be prefixed to the head of the tuple
with the \verb%,,% operator:

\begin{minted}{felix}
var tail = 1,2L; // binary tuple
var triple = "head",,tail; // ternary tuple
assert triple == "head",1,2L;
\end{minted}

Similarly, given a tuple type of at least two components,
another type may be prefixed to the head of the type
with the \verb%**% combinator:

\begin{minted}{felix}
typedef pair = int * long;
typedef triple = string ** pair;
// same type as string * int * long
\end{minted}

\subsection{Pattern matching tuples}
The real value of the alternate notation lies in pattern matching.
For example:

\begin{minted}{felix}
match "head",1,2L with
| head,,tail =>  println$ tail;
endmatch;
\end{minted}

The real power comes with classes:

\begin{minted}{felix}
instance [T,U with Eq[T], Eq[U]] Eq[T ** U] {
  fun == : (T ** U) * (T ** U) -> bool =
  | (ah ,, at) , (bh ,, bt) => ah == bh and at == bt;
  ;
}

instance[t,u with Eq[t],Eq[u]] Eq[t*u] {
  fun == : (t * u) * (t * u) -> bool =
  | (x1,y1),(x2,y2) => x1==x2 and y1 == y2
  ;
}

instance[t with Eq[t]] Eq[t*t] {
  fun == : (t * t) * (t * t) -> bool =
  | (x1,y1),(x2,y2) => x1==x2 and y1 == y2
  ;
}
\end{minted}

This mechanism allows recursive processing of heterogenous lists,
that is, tuples. In the example we compute equality.
The first instance provides heterogenous list recursion, 
the second case terminates the recursion. The third case
is also a terminal which allows the recursion to work
for arrays as well. Without this the library array recursion
would be more specialised.


\section{Records}
A record is similar to a tuple except the components are 
named and considered unordered up to duplication.

\subsection{Plain Record}
A plain record is one without duplicate fields. A plain record
type is one without duplicate fields or a row variable.
A record is constructed using a parenthesis enclosed list
of comma separated field assignments. An empty record is
equivalent to an empty tuple.

\begin{minted}{felix}
typedef xy = (x:int, y:int);
var r : xy = (x=1,y=2);
\end{minted}

\subsection{Record projections}
A component of a record may be accessed with a 
function called a record value projection, it is denoted
by the name of the field.

\begin{minted}{felix}
var r (x=1,y=2);
println$ x r, r.y;
\end{minted}

Record value projections can also be used as stand-alone
functions. For example:

\begin{minted}{felix}
var rl = list ((x=1,y=11),(x=2,y=22));
var xs =  map (x of (x:int, y:int)) rl;
println$ xs; // list (1,2)
\end{minted}

Records also have pointer projections, overloaded with
the value projections: if the name of
a field is applied to a pointer to a record,
a pointer to the named component field is returned.
This allows assignment and other mutators to be
applied to record components.

\begin{minted}{felix}
var r =(x=1,y=2);
var px = &r.x; // means (&r).x
px <- 42;
r&.y <- 23;
println$ r.x, r.y; // (42,23)
\end{minted}

Record pointer projects can also be used as stand-alone functions:

\begin{minted}{felix}
var prjx = x of (xy);
var prjpx = x of (&xy);
\end{minted}

\subsection{General record}
Records may have duplicate fields. In this case, reading from left
to right in a record literal, a duplicate field is hidden by a previous 
field of the same name, in a push down stack like fashion.
 
\begin{minted}{felix}
var r = (x=1,y=2,x="Hello");
println$ r._strr, r.x;
  // ((x=1,x='Hello',y=2), 1)
\end{minted}

Note that the generic function \verb%_strr% displays the whole of the
record including duplicate fields. However projections only find
the left-most field.

\subsection{Adding fields}
Fields can be added to an existing record to construct 
a new record:

\begin{minted}{felix}
var r = (x=1,y=2);
var r2 = (a="one",b="two",x="newx" | r);
println$ r2._strr,r2,x;
  // ((a='one',b='two',x='newx',x=1,y=2), newx)
\end{minted}

Again, leftmost fields hide rightmost ones. 

\subsection{Removing Fields}
The \verb%without% operator can be used to return a record
with some fields removed. It works on values of record and
polyrecord type. Note that because Felix uses scoped fields
once a field is removed, the previous value of that field is exposed
if it exists, and can also be removed. This means it is correct
and sometimes necessary to list a field more than once when
using the \verb%without% operator.

\begin{minted}{felix}
var x = (a=1,b=2,c=3,c=4);
var y = (x without b c); // (a=1,c=4)
\end{minted}

\subsection{Adding records}
You can also add two 
records with infix \verb%+%:

\begin{minted}{felix}
var r = (a=1) + (b=2) + (a="hello");
println$ r._strr, r.a.str; // ((a=1,a='hello',b=2), 1)
\end{minted}

The leftmost field with a given name dominates. Record addition
by \verb%+% is only applied if a user defined addition is not
found for the argument types.

Currently, addition of fields of two records of the same type
is not supported: it is likely the user intended to add corresponding
field values rather than hide the fields in the right argument
with those on the left.


\subsection{Row Polymorphism}
\label{Row Polymorphism}
Felix provides a special record type called a {\em polyrecord} which
supports row polymorphism with scoped labels in the style of
Daan Leijen. The article
is \href{http://research.microsoft.com/pubs/65409/scopedlabels.pdf}{here}.

This allows a generic function to be written which accepts an argument
which is or contains a value of a record type with more fields than
required. Unlike subtyping, the extra fields, whilst inaccessible,
are not lost and can be returned by the function. For example:

\begin{minted}{felix}
val circle = (x=0.0,y=0.0,r=1.0);
val square = (x=0.0,y=0.0,w=1.0,h=1.0);

fun move[T] 
  (dx:double, dy:double) 
  (shape: (x:double, y:double | T)) 
=>
  (x=shape.x+dx, y=shape.y+dx | (shape without x y))
;

var inc = 1.0,1.0;
println$ (move inc circle)._strr, (move inc square)._strr;
\end{minted}


\subsection{Interfaces}
An interface is a special notation for a record type
all of whose fields are functions or procedures.

\begin{minted}{felix}
interface fred {
  f: int -> int;
  g: int -> 0; // procedure
}

// equivalent to
typedef fred = (f: int -> int, g: int -> 0);
\end{minted}

The primary use is for specifying the type of a Java like
objects. See \fullref{Dynamic Objects}

\section{Structs}
A struct is a a nominally typed record, that is, it must be defined,
and each definition specifies a distinct type.

\begin{minted}{felix}
struct S { x:int; y:int; };
var s1 : S = S (1,2);
var s2 : S = S (y=42,x=32);
println$ s1.x,s1.y,s2.x,s2.y;
\end{minted}

\subsection{Alternate Definition Syntax}
As a special case, a \verb%struct% can be defined
with the field type before the field name:

\begin{minted}{felix}
struct S { int x; int y; };
\end{minted}

Furthermore there is a special syntax:

\begin{minted}{felix}
typedef struct {
  int x;
  int y;
} S;
\end{minted}

for defining structs. The purpose of this syntax is to allow
a C struct to be copied directly into Felix code with
minimal editing. Note the above form does {\em not} define
an alias, it defines a struct called \verb%S%.

\subsection{Struct constructor}
A struct value can be constructued using the structure name as a function
and passing a tuple of values corresponding by position to the fields
of the struct, as shown above. A record with matching fields can also be used.
Of course in both cases the types must also agree.

\subsection{Constructor as function}
A struct constructor can be used a first class function.
In this case the argument type is a tuple, not a record.

\begin{minted}{felix}
struct S { x:int; y:int; };

proc p (factory: int * int -> S) {
  var a = factory (1,2);
  println$ a.x,a.y;
}
p S;
\end{minted}

\subsection{Projections}
The field names are projection functions and can be applied
to a struct value to extract the nominated component.

They can also be applied to a pointer to a struct to find a pointer to the
nominated component.

\begin{minted}{felix}
struct S {x:int, y:int; };
var a = S (1,2);
var px = &a.x;
px <- 11;
&a.y <- 22;
\end{minted}

Projections can be used as first class functions with the notation:

\begin{minted}{felix}
var prjx = x of (S);
var prjpx = x of (&S);
\end{minted}

can be used to refer to a projection in isolation, and a pointer
projection in isolation, that is, as unapplied first class functions.

\subsection{Nested methods}
A struct may also contain function and procedure definitions:

\begin{minted}{felix}
struct A { 
  x:int;
  y:int;
  fun get2x => 2 * self.x;
  fun get2y () => 2 * self.y;
  proc diag (d:int) { self.x <- d; self.y <-d; }
};
\end{minted}

These functions are precisely equivalent to:

\begin{minted}{felix}
fun get2x (self:A) => 2 * self.x;
proc diag (self: &A) (d:int) { self.x <-d; self.y <-d; }
\end{minted}

Note that for a function \verb%self% is a value, 
for a procedure \verb%self% is a pointer.

\subsection{Method closures}
Because of these definitions, we can form object closures over a struct:

\begin{minted}{felix}
var a = A(1,2);
var g2y = a. get2y;
var di = a . diag;
\end{minted}

Note we can't form a closure for \verb%get2x% without an explicit wrapper,
i.e. eta-expansion.

\chapter{Sums}
\section{Anonymous Sum Types}
Sum types are the dual of tuples. They represent a sequence
of possible cases, potentially with arguments.  For example
the sum type:

\subsection{Sum type}
\begin{minted}{felix}
typedef num = int + long + double;
  // dual to int * long * double
\end{minted}

represents three cases and is dual to the tuple type shown.

\subsection{Sum values}
The value of a sum represents one of the cases 
of the type.  Case indicies are 0 origin as with tuples. 

\begin{minted}{felix}
typedef num = int + long + double;
var c0 = (case 0 of num) 42;
var c1 = (case 1 of num) 53L;
var c2 = (case 2 of num) 66.0;
\end{minted}

The parenthesised expressions above are called 
{\em type constructors}.

\subsection{{\tt caseno} operator}
The case number of a sum value can be extracted
as an \verb%int%:

\begin{minted}{felix}
println$ caseno c0, caseno c1, caseno c2;
  // 0,1,2
\end{minted}

\subsection{Switching on cases}
Sum variables are decoded with a match which
may also extract an argument value:

\begin{minted}{felix}
typedef num = int + long + double;
var x = (case 1 of num) 53L;
println$ 
  match x with
  | case 0 (i) => "int " + i.str
  | case 1 (l) => "long" + l.str
  | case 2 (d) => "double " + d.str
  endmatch
;
\end{minted}

The match tags do not include the type of the sum, since
it is known from the argument.

\subsection{Unit arguments}
A sum component may have a unit argument.

\begin{minted}{felix}
typedef maybe_int = 1 + int;
var x = case 0 of maybe_int;
match x with
| case 0 () => println$ "Nothing";
| case 1 i => println$ "Integer " + i.str;
endmatch;
\end{minted}

For the first case the \verb%()% may be omitted:

\begin{minted}{felix}
typedef maybe_int = 1 + int;
var x = case 0 of maybe_int;
match x with
| case 0 => println$ "Nothing";
| case 1 i => println$ "Integer " + i.str;
endmatch;
\end{minted}

or the unit application syntax can be used:

\begin{minted}{felix}
typedef maybe_int = 1 + int;
var x = case 0 of maybe_int;
match x with
| #case 0 => println$ "Nothing";
| case 1 i => println$ "Integer " + i.str;
endmatch;
\end{minted}

If the argument is a unit, the constructor is 
sometimes called a {\em constant constructor}.

\subsection{Unit sum}
There is a family of special sum types equivalent to:

\begin{minted}{felix}
  2 = 1 + 1
  3 = 1 + 1 + 1
  4 = 1 + 1 + 1
\end{minted}

Recall type 1, or \verb%unit%, is the type of the empty tuple.

The type 2 is also known as \verb%bool%, and represents two cases
where \verb%false% is an alias for \verb%case 0 of 2% and 
\verb%true%
is an alias for \verb%case 1 of 2%.

The type 0 or \verb%void%, is the type of no values.

These types are called {\em unit sums} because they're a sum
of a certain number of units.

Note carefully that:

\begin{minted}{felix}
  x + (y + z), (x + y) + z, x + y + z
\end{minted}

are three distinct types because operator + is not associative.

\section{Unions}
A union is the dual of a struct. It is a nominally typed version
of a sum. Here for example is a list of integers:

\begin{minted}{felix}
union intlist {
  iEmpty ;
  iCons of int * intlist;
};
\end{minted}


This alternative syntax is more commonly used
and comes from ML family:

\begin{minted}{felix}
union intlist = 
  | iEmpty 
  | iCons of int * intlist
;
\end{minted}


The fields of a union type are injections or type constructors.
In effect they cast their argument to the type of the union, thus
unifying heterogenous types into a single type.

Pattern matches are used to decode unions.

\begin{minted}{felix}
var x = iEmpty;
x = iCons (1, x);
x = iCons (2, x); // list of two integers

fun istr (x:intlist) =>
  match x with
  | #iEmpty => "end"
  | iCons (i, tail) => i.str + "," + istr tail
  endmatch
;
\end{minted}


The first variant represents an empty list. The second variant
says that a pair consisting of an int and a list can be considered
as a list by applying the type constructor iCons to it.

Unlike product types, a sum may directly contain itself.
This is because sum types are represented by pointers.

\subsection{enum}
A restricted kind of union, being a nominally typed version
of a unit sum.

\begin{minted}{felix}
enum colour { red, green, blue }; // same as
enum colour = red, green, blue; // same as
union colour = red | green | blue;
\end{minted}


The tag value of an enum can be set:

\begin{minted}{felix}
enum wsize = w8=8, w16=16, w32=32, w64=64;
\end{minted}


\subsection{caseno operator}
The caseno operator can find the tag value of any sum type,
the anonymous sum, union, enum or variant as an integer.

\begin{minted}{felix}
assert caseno w16 == 16;
assert caseno (case 1 of 2) == 1;
\end{minted}


\section{Variants}
Variants are the sum type which are the dual of records.
They used named injections like unions but are structurally typed.


\subsection{Variant coercion}
Variant values always have a singleton type, so to be used
with a set of variants, a coercion must be used.

\begin{minted}{felix}
typedef vars = union { Int of int ; Float of float; };
var x = Int 32; // type union {Int of int}
var y = x :>> vars; // coerce to type vars
\end{minted}

The type being coerced must be super type of the coercion type,
this means it must have subset of the cases of the coercion type.


\chapter{Exponentials}
\section{Functions}
Felix provides four core function types.

\subsection{Function}
Functions use infix operator \verb%->% to denote a function,
where the LHS is called the domain, and the RHS the codomain:

\begin{minted}{felix}
var f : int -> long = fun (x:int)=>x.long;
\end{minted}

The operator is right associative so that

\begin{minted}{felix}
var f : int -> long -> string = 
  fun (x:int) => 
    fun (y:long)=> 
      (x.long + y).str
;
// equivalent to
var f : int -> (long -> string) = 
  fun (x:int) (y:long) => 
      (x.long + y).str
;
\end{minted}

If a function returns a function, it is a higher order functions.
The number of \verb%->% symbols in the type is known as the
{\em arity} of the function.

Function defined with \verb%fun% binder may not have side effects.

Functions are slaves which accept data and control, perform
a calculation, and return a result and control to the 
caller, which is the master. The master/slave relationship may
also be expressed as a client/server relationship.

Felix uses the machine stack to save and retrieve the
master return address for a function.

At run time a function closure is represented by a single machine
address which is nominally a pointer to a heap allocated C++
object which retains the function local variables and
provides an \verb%apply()% method.

Functions which depend only on their arguments and
any constants in their context are said to be {\em pure},
those which depend on external variables are {\em impure}.
The kind can be noted with an adjectival descriptor on
the definition:

\begin{minted}{felix}
pure fun sin: double -> double = "sin($1)";
var x = 1;
impure fun get_x() => x;
\end{minted}

Impure functions are sometimes called observers or accessors.

\section{Generators}
A generator is a function which is permitted to have side effects.
It is defined with the \verb%gen% binder instead of \verb%fun%.

Generators have the same type as functions.

The exemplar generator is of course \verb%rand()% which
returns a different value each call, depending on some
external state.

\section{Yielding Generators}
Generators may contain a yield statement which returns a value
and control, but also records the current location in the local
state, to allow the generator to be resumed where it left off.

To use a yielding generator a closure must be stored in a variable.

Yielding generators are a special case of coroutines, implemented
in a special way.

\begin{minted}{felix}
gen integers () = {
  var i = 0;
next:>
  yield i;
  ++i;
  goto next;
}

var counter = integers; 
var zero = counter();
var one = counter();
\end{minted}

\subsection{Procedure}
A procedure is given the type of a function with the codomain
set to 0 (aka void). Procedures are not functions, they return
control but not a value. Technically a procedure is a delimited
continuation.

Although procedures look like functions, in Felix they are utterly
different because they do not use the machine stack to store the
return address. Instead, procedures use a heap allocated data frame,
with pointers to ancestral context and the caller continuation
return address, and save the current location in a special variable
in the data frame.

Thus procedures may cooperatively swap themselves out in favour
of another procedure. 

A procedure with its ancestral context, call chain, and current
location is called a fibre. It represents an exchangeable locus
of contol.

\subsection{C function}
Felix can deal with C functions too. The infix operator 
\verb%-->% is used to denote a C function type. C functions
can be defined directly in Felix like:

\begin{minted}{felix}
cfun f(x:int)=>x + x;
\end{minted}

Felix will represent this function like:

\begin{minted}{c}
 int f(int x) { return x + x; }
\end{minted}

which has a C function type. The Felix C function type is
a C function closure, in C a pointer to the C function.

If a \verb%cfun% uses the thread frame or garbage collector,
it will fail to compile at C++ compile time because Felix
always passes a pointer to the thread frame explicitly.

We provide \verb%cfun% primarily to allow construction of C data
types used by drivers as callback functions.

However they can also be used to define an interface to a C callable
library written in Felix.

Note that Felix knows that C functions accept multiple arguments
rather than tuples and automatically unpacks a tuple passed
to a C function.

\subsection{C procedure}
Felix can deal with C procedures too, the type is the same
as a C function with the codomain set to void.

\begin{minted}{felix}
var y = 1;
cproc f(x:int) {y = x}
\end{minted}

Felix will represent this like:

\begin{minted}{c}
 void f(int x) { y = x; }
\end{minted}

which has a C function type. The Felix C function type is
a C function closure, in C a pointer to the C function.

If a \verb%cfun% uses the thread frame or garbage collector,
it will fail to compile at C++ compile time because Felix
always passes a pointer to the thread frame explicitly.

\subsection{Type indexed functions}
Felix has an extension which allows function types to
be indexed by types. For example:

\begin{minted}{felix}
struct IO {};
fun f(x:int) :[IO] int = {
  println$ x; return x;
}
var clos : int ->[IO] int = f;
\end{minted}

The indexing type is known as the {\em effects} type
of the function. For functions without effects annotations,
the effects type defaults to \verb%unit%.

By combining effects types, row polymorphism, and typematches,
Felix provides tools to manage various effects. For example:

\begin{minted}{felix}
fun map[T,U] (f:T->[G] U) (list[T]) : [G] list[U] => ..
\end{minted}

says that the higher order function map, given a function f
with effect G, will itself have effect G.

This is an experimental feature that is not yet utilised by
the library.

\section{Array}
Felix has various kinds of array. The term is abused and sometimes
refers to the abstract concept, and sometimes the statically
typed fixed length array described here.

An array is nothing but a tuple all of whose elements have the
same type. It is convenient to use an exponential operator with
a  unit sum index to provide a compact notation:

\begin{minted}{felix}
int ^ 3 // array of 3 integers equivalent to
int * int * int
\end{minted}

Therefore the value:

\begin{minted}{felix}
var a3 = 1,2,3;
\end{minted}

is, in fact, an array. As for tuples an integer literal applied
to an array value returns a component, however for arrays,
an expression may be used as well:

\begin{minted}{felix}
var i = 1;
var y = a3 . i;
var z = a3 . proj i of (int^3);
\end{minted}

The last form is equivalent to

\begin{minted}{felix}
var z = a . 
  match i with
  | 0 => proj 0 of (int^3)
  | 1 => proj 1 of (int^3)
  | 2 => proj 2 of (int^3)
  | _ => throw error
  endmatch
;
\end{minted}

in other words there is a run time array bounds check equivalent to a match failure.
Note that of course the actual generated code is optimised!

A run time check can be avoided by using the correct type of index:

\begin{minted}{felix}
var i = case 1 of 3;
var z = a . i; // no run time check
\end{minted}


\subsection{Multi-arrays}
Whilst we introduced the exponential notation

\begin{minted}{felix}
  B ^ J
\end{minted}

as a mere shorthand, where J is a unit sum, in fact Felix allows
the index to be any compact linear type.

A compact linear type is any combination of sums, products,
and exponentials of unit sums. The type

\begin{minted}{felix}
 3 * 4 * 5
\end{minted}

for example is compact linear, and therefore Felix allows the array type

\begin{minted}{felix}
typedef matrix = double ^ (3 * 4 * 5)
\end{minted}

Although this looks like the type

\begin{minted}{felix}
typedef array3 = double  ^ 3 ^ 4 ^ 5
\end{minted}

as suggested by the usual index laws, the latter is an array size 5 of
arrays size 4 of arrays size 3 which can be used like:

\begin{minted}{felix}
var a : array3;
var z = a . case 1 of 5 . case 1 of 4 . case 1 of 3;
\end{minted}

where you will note that the projections are applied in the reverse
order to the indices. On the other hand to use the first form we have
instead:

\begin{minted}{felix}
var m : matrix;
var z = a . (case 1 of 3, case 1 of 4, case 1 of 5);
\end{minted}

The exponent here is a value of a compact linear type.
It is a single tuple value! It is called a multi-index
when applied to an array.

The advantage of this type is that there is an obvious
encoding of the values shown in this psuedo code:

\begin{minted}{felix}
  i * 3 * 4 + j * 3 + k
\end{minted}

which is nothing more than a positional number notation where the
base varies with position. That encoding clearly associates
with the compact linear value as integer in the range 0 to 59,
or, alternatively, an value of type 60. In other words the type
is linear and compact.

Since clearly, given an integer in range 0 through 59 and this type
we can decode the integer into a tuple, being the positional representation
of the integer in this weird coding scheme, the type is clearly
isomorphic to the subrange of integer.

Therefore Felix allows you to coerce an integer to a compact linear
type with a run time check, and convert a compact linear type
to an integer or a unitsum:

\begin{minted}{felix}
var i : int = ((case 1 of 3, case 1 of 4, case 1 of 5) :>> int);
var j : 60 = ((case 1 of 3, case 1 of 4, case 1 of 5) :>> 60);
var clt : 3 * 4 * 5 = (16 :>> 3 * 4 * 5);
\end{minted}

Because we can do this we can now write a loop over a matrix
with a single iterator:

\begin{minted}{felix}
for i in 60 do
  println$ m . (i :>> 3 * 4 * 5);
done
\end{minted}

This is an advanced topic which will require an extensive explanation
beyond the scope of this summary. However we will not that this
facility provides a very high level feature known as polyadic array
programming. In short this means that one may write routines which
work on matrices of arbitrary dimension. You can of course do this
in C by doing your own index calculations at run time and using
casts, however Felix does these calculations for you based on the
type so they're always correct.

\chapter{Type Recursion}
Felix provides two mechanisms to allow type recursion:
implicit recursion by self reference, and explicit recursion
using pointers.

\section{The {\tt as} combinator}
We can express the type of a list without a definition using
the \verb%as% combinator:

\begin{minted}{felix}
typedef intlist = 1 + int * X as X;
\end{minted}

The \verb%as% combinator has low precedence and introduces
a binding variable similar to a $\mu$ binder in type theory,
however the variable is visible outside the scope it controls
as well. The inner use of the variable is known as a {\em fixoint}
and is a place where whole term controlled by the binder
may be substituted, resulting, if repeated, in an infinnite
recursion. The use of the binder therefore allows infinite
data types to be represented finitely.

In the list case above, a list is either a unit, or a pair
consisting of an int and another list.

Type recursion in Felix may be written anywhere but will not
work correctly unless a box intervenes. A box is either any
sum type or a pointer.

\section{Pointers}\index{pointers|textbf}
Pointers provide another mechanism for implementing type recursion
explicitly. Pointers are more powerful than fixpoints, which is
not always an advantage.

The notation for a pointer to an object of type T is
either

\begin{minted}{felix}
&T
ptr[T] // alias for &T
\end{minted}

Values of pointer type can originate in two primary ways:
from the address of a variable:

\begin{minted}{felix}
var x = 1;
var px : &int = &x;
\end{minted}

or by constructing an object on the heap. The \verb%new%
operator copies a value to a heap object and returns its
address as a pointer:

\begin{minted}{felix}
var px : &int = new 1;
\end{minted}

The basic Felix pointer type cannot be NULL, cannot be
incremented, and always points to a single object.
It is therefore always safe to dereference to fetch
the value stored at the pointer with the prefix
dereference operator \verb%*%:

\begin{minted}{felix}
var px : &int = new 1;
var x = *px;
\end{minted}

Objects located by pointers are intrinsically mutable.
A new value can be stored with the store at operator:

\begin{minted}{felix}
var px : &int = new 1;
px <- 2;
\end{minted}

Pointers to products have special properties. More precisely,
for each value projection $p$ which extracts a the value of
a product component, there is a corresponding pointer projection
which accepts a pointer to a product object and returns a pointer
to the component;

\begin{minted}{felix}
var pair = 1,"Hello";
var s = pair . 1; // value projection
var ps = &pair . 1; // pointer projection
ps <- "Bye";  // pair is now 1,"Bye"
\end{minted}

This machinery eliminates the need for the concept of addressable
lvalues. Thus you cannot take the address of a product component:

\begin{minted}{felix}
var pair = 1,"Hello";
// var ps = &(pair.1); // ERROR
\end{minted}

because \verb%pair.1% is a value, and values do not have
public addresses. Instead Felix provides some sugar:

\begin{minted}{felix}
var pair = 1,"Hello";
var ps = pair&.1; // means (&pair).1
var &p = &pair;
var s = p*.1; // means (*p).1
\end{minted}

\section{Maybe NULL pointers}
Felix also has a pointer type that could perhaps be NULL.
It is defined in the library by:

\begin{minted}{felix}
  union cptr[T] = | nullptr | Ptr of &T;
\end{minted}

and the alias

\begin{minted}{felix}
@T
\end{minted}

may also be used. Due to implementation magic, this type
is compatible with C/C++ pointers, that is, Felix \verb%nullptr%
is C \verb%NULL%. In particular, one should write, for example:

\begin{minted}{felix}
gen malloc: size -> @T = "malloc($1)";
\end{minted}

because \verb%malloc% could return NULL. Strict adherence to 
correctly typing C functions is necessary to ensure the
guarrantee dereferencing Felix pointers is safe.

Note that dereferencing C pointers, even if not NULL, need
not be safe because the pointed at object may not be live.
Felix guarrantees to keep heap allocated objects to which
Felix pointers exist live.

Note that a \verb%cptr[T]% obtained from a \verb%ptr[T]%
has this guarrantee as well.

\section{Incrementable Pointers}
Felix also has C style incrementable pointers given the
type \verb%carray[T]% or alias \verb%+T%. Pointers of this
type are not intrinsically safe to use. They can be NULL.

The \verb%carray% is provided as a model of C arrays so that
they can be manipulated in Felix, even if such operations
are risky.

\chapter{Meta-typing}
Felix provides some facilities for meta-typing.

\subsection{typedef fun}
The notation

\begin{minted}{felix}
typedef fun diag (T:TYPE):TYPE=> T * T;
var x: diag int = 1,2;
\end{minted}

defines a type function (or functor). Given a type T,
this function returns the type for a pair of T's.
The identifier \verb%TYPE% denotes the kind which is a category 
of all types.

Applications of type functions must be resolved during binding, since
the result may influence overloading.

\section{{\tt typematch} expression}
\label{typematch}
Felix has a facility to inspect and decode types at compile time.

\begin{minted}{felix}
typedef T = int * long;
var x: 
  typematch T with
  | A * B => A
  | _ => int
  endmatch
  = 1
;
\end{minted}

As with type functions, type matches must be resolved during binding.
If a type match fails, an error is issued a compilation halted.
The wildcard type pattern \verb%_% matches any type.

The real power of the type match comes when combined with a type function:

\begin{minted}{felix}
typedef fun promote (T:TYPE): TYPE =>
  typematch T with
  | #tiny => int
  | #short => int
  | #int => int
  | #long => long
  | #vlong => vlong
  endmatch
;
\end{minted}

This functor does integral promotions of signed integer types 
corresponding to ISO C rules.

\section{Type Sets}
A type set is a component of a type constraint expression
which restricts a type variable one of a specifed list of entities. 
Each entity may be either a type, a type predicate, or another
type set.  Syntax like:

\begin{minted}{felix}
typesetof (short, int, long, vlong)
\end{minted}

is used for a typeset expression.

The typedef statement may be used to define an alias for
a typeset:

\begin{minted}{felix}
typedef Integers = typesetof (short, int, long, vlong);
\end{minted}

\subsection{Union and Intersection}
Two operators are provided for typesets, union and intersection
which are spelled \verb%\cup% and \verb%\cap% respectively and
are printed as set theoretic union and intersection symbols $\cup$ and $\cap$.

\begin{minted}[escapeinside=||]{felix}
typedef sints = typesetof (int,long);
typedef uints = typesetof (uint,ulong);
typedef ints = sints |$\cup$| uints;
typedef shorts = typesetof (short,int,long);
typedef ssints =  ints |$\cap$| shorts;
\end{minted}


\subsection{Membership}
The membership operator tests if a type is a member
of a typeset, it is spelled \verb%in% or \verb%\in%, the latter
being typeset as the set theoretic symbol $\in$. It serves to define the semantics
of a typeset as a typematch.

\begin{minted}[escapeinside=||]{felix}
typedef ints = typesetof(short,int,long);
typedef result = int |$\in$| ints;
\end{minted}

The membership test above is precisely equivalent to

\begin{minted}{felix}
typedef result = typematch int with
  | short => 1
  | int => 1
  | long => 1
  | _ => 0
endmatch;
\end{minted}



\subsection{Typeset membership constraints}
\label{Typeset membership constraints}
Typesets were introduced to Felix to simplify bindings to C functions.
For example:

\begin{minted}{felix}
typedef someints = typesetof (int, long);
typedef someuints = typesetof (uint, ulong);
fun someadd[T in someints] : T * T -> T = "$1+$2";
fun someadd[T in someuints] : T * T -> T = "$1+$2";
println$ someadd (1,2); // OK, first function
println$ someadd (1u,2u); // OK, second function
// println$ someadd (1,2u); // fails
\end{minted}

In this example, a call to \verb%someadd% will only work if
the arguments types are the same and both either \verb%int% or \verb%long%.
Overload resolution first treats the functions as if there were no constraints.
Then, any candidate which fails the constraint it thrown out of the set.

This method saves writing out all the cases individually,
which may be exponential in the number of individual types involved,
whilst at the same time preventing invalid cases which an unconstrained
binding would allow.

If two candidates exist with the same signatures, and one has
a constraint and the other does not (or has the constant constraint true),
the constrained candidate is considered more specialised than the
unconstrained one.

However in general, constraints cannot be compared to see which is
most specialised. In particular if for all values of type variables
constraint A implies constraint B, A is more specialised, however
there is no operational method for measuring this in general
(halting problem!).



\chapter{Polymorphism}
\section{Type Constraints}\index{constraints!type}
\label{Type Constraints}
Felix provides 4 kinds of type constraints.
\subsection{Type class constraints}
Class constraints have no impact on phase 1 overload resolution
when selecting a binding. 

However, when binding the definition
of a symbol, specified class views are introduced in a shadow
scope as if the class were opened.

During monomorphisation, any virtuals must be mapped to instance
functions and this is done by a process similar to that used for
overload resolution. 

\subsection{Typeset membership constraints}
\label{Typeset membership constraints}
Typesets were introduced to Felix to simplify bindings to C functions.
For example:

\begin{minted}{felix}
typedef someints = typesetof (int, long);
typedef someuints = typesetof (uint, ulong);
fun someadd[T in someints] : T * T -> T = "$1+$2";
fun someadd[T in someuints] : T * T -> T = "$1+$2";
println$ someadd (1,2); // OK, first function
println$ someadd (1u,2u); // OK, second function
// println$ someadd (1,2u); // fails
\end{minted}

In this example, a call to \verb%someadd% will only work if
the arguments types are the same and both either \verb%int% or \verb%long%.
Overload resolution first treats the functions as if there were no constraints.
Then, any candidate which fails the constraint it thrown out of the set.

This method saves writing out all the cases individually,
which may be exponential in the number of individual types involved,
whilst at the same time preventing invalid cases which an unconstrained
binding would allow.

If two candidates exist with the same signatures, and one has
a constraint and the other does not (or has the constant constraint true),
the constrained candidate is considered more specialised than the
unconstrained one.

However in general, constraints cannot be compared to see which is
most specialised. In particular if for all values of type variables
constraint A implies constraint B, A is more specialised, however
there is no operational method for measuring this in general
(halting problem!).

\subsection{Equational Constraints}
Felix allows a constraint to be
For example:

\begin{minted}{felix}
fun f[T,U where T * int == int * U] (x:T * U) => x;
println$ f (1,2);
\end{minted}

This example uses an equational constraint.

Here is another example using a typematch:

\begin{minted}{felix}
typedef fun ispair (x:TYPE) : TYPE => 
  typematch x with
  | _ * _ => 1
  | _ => 0
  endmatch
;

fun f[T where ispair T] (x:T)=>x;

println$ f (42,33L); // OK, signature ispair (int * long) true
\end{minted}
\chapter{Generics}
\section{Generic Functions}
Generic provide a way to do {\em cut and paste} polymorphism,
similar to what C++ does. It is useful for one-off scripts,
or times when parametric polymorphism and type classes are
not enough.

\subsection{Generic Syntax}
The syntax is to just put the kind of a type
variable as \verb%GENERIC%:

\begin{minted}{felix}
fun accumulate [T:GENERIC] (container:T) = {
  var sum = 0;
  for elt in container do
    sum = sum + elt;
  done 
  return sum;
}
\end{minted}

\subsection{Using a generic}
With the above example you can write:

\begin{minted}{felix}
println$ accumulate (1,2,3,4);
println$ accumulate (1,2,3,4).list;
println$ accumulate (1,2,3,4).varray;
\end{minted}

\subsection{Operational Semantics}
Felix handles generics as follows. The generic function is
treated first, as if it were polymorphic, that is, the
\verb%GENERIC% kind replaced by the default \verb%TYPE%.
After overload resolution is complete, if a generic
function is selected, Felix clones the function and all
its descendants, creating a new function. In these clones,
however, the generic type variable is removed and in the
definition of the function and its descenants it is replaced
by the actual type deduced by the overload resolution algorithm.
The system then replaces the overload result with a reference
to this new function.

Therefore, the call is dispatched to a non-generic function,
specialised for the arguments.

This kind of operation is also done by the type class specialiser
during monomorphisation, and inlining routines during optimisation,
however with generics the specialisation is done {\em before} binding.

Unlike C++, the specialisation is done in the context of the definition.
This is more reliable, but it means functions in the context of use
cannot participate in overload resolution.

\begin{minted}{felix}
class X {
  proc f(x:int) => println$ "X::f " + x;
  proc g[T:GENERIC] (x:T) => f x;
  g 25;
}

class Y {
  proc f(x:int) => println$ "Y::f " + x;
  X::g 33;
}

X::g 99;
\end{minted}

The output is:

\begin{minted}{text}
X::f 25
X::f 33
X::f 99
\end{minted}

\section{The {\tt typecase} expression}
The \verb%typecase% expression allows you to write
expressions dependent on a type.  Typecase expressions must be
resolved before binding is complete.

\begin{minted}{felix}
fun f[T:GENERIC] (x:T) =>
  typecase T with
  | int => x + 1
  | double => x + 1.0
  endmatch
;

println$ f 1, f 4.2;
\end{minted}

 Although primarily of use in generic functions, 
typecases can also be useful for conditional compilation.
For example:

\begin{minted}{felix}
if PLAT_WIN32
  type file = "HANDLE";
else
  type file = "int";
endif

fun read (f:file) =>
  typecase typeof f with
  | int => unix::read f
  | HANDLE => win32::read f
  endmatch
;
\end{minted}

\section{Builtin Generics}
Felix has a number of builtin generic functions.

\subsection{The {\tt \_strr} function}
The \verb%_strr% function converts any value into a string,
with a syntax roughly approximating the syntax used in the program.

It was primarily intended to simplify output of records, the 
final \verb%r% from to record.

\begin{minted}{felix}
println$ (a=1,b="hello", c=(1,2,3), a=(1,2,3).list);
struct A {
  a:int;
  b:int;
}
println$ (A(1,2))._strr;
\end{minted}

The output is:

\begin{minted}{felix}
(a=1,a=Snoc (Snoc (Snoc (Empty,3),2),1),b='hello',c=(1, 2, 3))
A {a=1,b=2}
\end{minted}

The \verb%Snoc(tail,head)% terms there are the native list constructors
used by Felix, which are reversed \verb%Cons(head,tail)% terms. Felix
uses these because they generate a pointer followed by data in C++,
which means list scanning is independent of the data type.

\subsubsection{Operational Semantics}
The converter works by calling the \verb%str% function for primitives
such as \verb%int%, \verb%string%, or \verb%float%, so these must
be in scope. The type class based \verb%str% is injected into the
global shadow scope so this will always be bound, however if
there is no instance for a type an error will be generated during
monomorphisation.

For structured types, \verb%_strr% exhibits the structure in some
nice way, wrapping the child members of the structures, which are
converted to strings recursively.

There is a limit on the recursion depth, and Felix will print
\verb%...% when it hits that depth.

\begin{minted}{felix}
println$ (1,2,3,4,5,6,7,8,9).list._strr;
\end{minted}

produces

\begin{minted}{text}
Snoc (Snoc (Snoc (Snoc (...,4),3),2),1)
\end{minted}

\subsection{The {\tt \_eq} function}
Felix provides a builtin generic function \verb%_eq% which can
be used to compare small acyclic values of arbitrary types:

\begin{minted}{felix}
var x = (1,(a=2,b=3,c=(1,2).list));
var y = x;
println$ _eq (x,y);
\end{minted}

\subsubsection{Operational Semantics}
The comparator requires arguments of the same type.
It operates by a recursive descent into the structure
of the types. If the type of a subexpression is primitve, 
the function \verb%==% is used. This function is defined
in the typeclass \verb%Eq%, and may also be defined in the
context of the comparison.

Do not use this generic on a cyclic value, it will loop 
forever at run time.

The current implementation only expands tuples and records.
It delegates to \verb%==% for all other types. Therefore,
lists will be handled correctly since \verb%==% is already
defined for lists in the library.

Technically \verb%_eq% is polyadic, not generic. However the
implementation is generic.

\subsection{The {\tt \_map} function}
The \verb%_map% operator maps an overloaded function set over a tuple. 
For example:

\begin{minted}{felix}
fun f(x:int)=> x + 1;
fun f(x:double)=> (x+42.0).str+ "!!";
fun f[T] (x:T)=>x;
println$ _map f (1,2.2,"Hello");
\end{minted}

produces:

\begin{minted}{text}
(2, 44.2!!, Hello)
\end{minted}

Although many generic operations can better be defined using type classes,
it is not possible to do this kind of operation other than with a generic.

All folds, for example, can be defined with type classes because
folding mandates the return type be invariant, and we can use
polymorphism on the argument type. However this is not possible
with mapping because the return type is not dependent on the
argument type in a predictable manner.

Technically, Felix can indeed express the return type of
the individual functions. Consider:

\begin{minted}{felix}
typedef fun fret(argtype:TYPE):TYPE =>
  typematch argtype with
  | int => int
  | double => string
  | ?other => other
  endmatch
;

class U[T] {
  virtual fun ff: T -> fret T;
}

instance U[int] {
fun ff(x:int)=> x + 1;
}

instance U[double] {
fun ff(x:double)=> (x+42.0).str+ "!!";
}

instance[T] U[T] {
fun ff[T] (x:T)=>x;
}

open[T] U[T];

println$ ff 22; 
println$ ff 3.0; 
println$ ff "Hello";
\end{minted}

The first two prints actually work, but the last one fails:

\begin{minted}{text}
CLIENT ERROR
[Cannotmatch] Cannot instantiate virtual ff<271188>[string]
In /Users/skaller/felix/b.flx: line 57, cols 1 to 20
56: println$ ff 3.0;
57: println$ ff "Hello";
    ********************
58:
\end{minted}

The reason is, that the ff[T] in instance U[T] does not have the
correct type to specialise the virtual. The virtual has type

\begin{minted}{felix}
T ->  
  typematch argtype with
  | int => int
  | double => string
  | ?other => other
  endmatch
\end{minted}

whereas the instance has type

\begin{minted}{felix}
T -> T
\end{minted}

however matching T against \verb%int% or %double% in the typematch stalls
because whether the match succeeds depends on the not yet performed
replacement of T with a specific type.

Note that in theory if the \verb%double% branch were removed,
the typematch would always return \verb%T%, since if it is \verb%int% it returns
\verb%int%, and otherwise returns that. However the compiler is not
able to reduce the typematch like that because it looks only
at the top branch to begin, and continues only on definite failure.


\part{Interfacing}
\chapter{C bindings}\index{binding!C|textbf}
Felix is specifically designed to provide almost seamless integration
with C and C++.

In particular, Felix and C++ can share types and functions,
typically without executable glue.

However Felix has a stronger and stricter type system than C++
and a much better syntax, so binding specifications which lift
C++ entities into Felix typically require some static glue.

\section{Type bindings}\index{binding!type|textbf}
In general, Felix requires all primitive types to be first class,
that is, they must be default initialisable, copy constructible,
assignable, and destructible. Assignment to a default initialised
variable must have the same semantics as copy construction.

It is recommended C++ objects provide move constructors as
Felix generated code uses pass by value extensively.

The Felix type system does not support C++ references in general,
you should use pointers instead. 

However, there is a special lvalue annotation for C++ functions
returning lvalues that allows them to appear on the LHS of
an assignment. Only primitives can be marked lvalue.

The Felix type system does not support either const or volatile.
This has no impact when passing arguments to C++ functions.
However it may be necessary to cast a pointer returned from
a primitive function in order for the generated code to type check.

\section{Basic {\tt type} statement}\index{type!statement}\index{type!statement}
The type statement is used to lift a C++ type into Felix.

The simplest example:

\begin{minted}{felix}
type metres = "double";
type Z = "long";
\end{minted}

creates a Felix type \verb%metres% based on C++ type
{\tt double}.

Polymorphic types can be created too:

\begin{minted}{felix}
type pair[T,U] = "::std::pair<?1,?2>";
\end{minted}

The names of the type variables $T$ and $U$ are not used
in the binding, but serve as placeholders. Instead the
encoding \verb%?1% and \verb%?2% is used to refer to the first
and second type variables. The Felix type \verb%pair[metres,Z]%
will become C++ type \verb%::std::pair<double,long>% in the
generated code.

Please be sure to always fully qualify C++ names including
leading \verb%::% since you do not know the namespace context
in which the generated code will reside.

\subsection{Semantics}
Unless otherwise noted all types lifted into Felix with
the type binding statement must be first class. This means
that they must have a publically accessible 
default initialiser, copy constructor, copy assignment operator, 
and destructor.

To lift a type which is immobile, lift a pointer to the type
instead. Please see below.

\section{Adjectives}
Felix needs to know some things about types to handle
them properly, it uses a system of adjectives to provide
some basic information.

\subsection{{\tt pod} types}\index{type!pod}
A \verb%pod% type is a {\em plain old data type} of the kind
common in C. In Felix the annotation is used for optimisation:
calls to the destructor are not made by the Felix garbage collector
when an object of this type is finalised. This will have no impact
provided the type has a trivial destuctor in C++.

Do not specify this adjective unless you are sure the destructor
does no useful work.

\begin{minted}{felix}
pod type metres = "double";
\end{minted}

\subsection{{\tt \_gc\_pointer} types}\index{type!\_gc\_pointer}
The \verb%_gc_pointer_% adjective tells Felix that the
type is a pointer which could be a pointer to an heap
allocated object managed by the GC.

When a value of this type is included in a larger objects,
the run time type description will include the offset
of this subobject in the array of offsets which might
contain managed pointers.

It is always safe to use this annotation if the C++ types
is a pointer.

\subsection{{\tt incomplete} types}\index{type!incomplete}
The adjective \verb%incomplete% may be used to specify the
type denoted may not be constructed. If the C++ type being
bound is incomplete, for example \verb%void%, or an
incomplete C struct, this annotation will cause a type error
in Felix if an attempt is made to construct values of the type.

The primary reason for introducing incomplete types is to
permit a Felix pointer to be formed to it, in Felix.
For example

\begin{minted}{felix}
incomplete type MyVoid = "void";
typedef MyVoidPointer = &MyVoid;
\end{minted}

\subsection{{\tt uncopyable} types}\index{type!uncopyable}
This annotation may be used if a value of the type can
be constructed but not copied. Since Felix always initialised
variables with assignments, uncopyable type are useful only
for heap construction yielding a pointer.

\subsection{{\tt \_gc\_type T} types}\index{type!\_gc\_type}
This adjective is for gurus only! In conjunction
with \verb%_gc_pointer% it is used to permit
contruction of an immobile C++ type whilst at the same
time ensuring that the type garbage collected.

The method is fraught with peril and use of C++ smart pointers
is recommended instead.

\begin{minted}{felix}
  private uncopyable type bQ_ = 
    "::flx::pthread::bound_queue_t"
  ;
  _gc_pointer _gc_type bQ_ type ts_bound_queue_t[T] = 
    "::flx::pthread::bound_queue_t*" 
    requires 
     package "flx_bound_queue",
     scanner "::flx::pthread::bound_queue_scanner"
  ;
  ctor[T] ts_bound_queue_t[T]: !ints = 
    """
      new (*PTF gcp, @0, false) ::flx::pthread::bound_queue_t(
      PTF gcp->collector->get_thread_control(), (size_t)$1)
    """ requires property "needs_ptf";
\end{minted}

In this example from the library, we define the thread safe bound
queue as type \verb%bQ_% and mark it private and uncopyable 
to protect it. The C++ type is defined the Felix run time library.

The type the user gets is a ponter, which is defined by a binding
to C++ again, declared to be a pointer, and declared to point 
to \verb%bQ_%.

What this actually does is seen in the constructor which,
in C++, constructs the object on the heap, passing the magic
encoding \verb%@0% to the second argument of the new expression.

This encoding {\em normally} denotes the pointer to the shape (RTTI) for the
return type of the function. However that type, \verb%ts_bound_queue%
has the attribute \verb%_gc_type bQ_% which tells Felix to use
the shape object for that type instead. We need to do this because the
heap value is represented by a pointer to the type, but we want the
garbage collector to be able to map such a pointer to the type of the
heap object at run time.

\subsection{{\tt finaliser} clause}\index{finaliser!clause}
A primitive type may have a \verb%finaliser% clause with
string argument which must specify a C++ function with type

\begin{minted}{c}
typedef void finaliser_t (collector_t*, void*); 
\end{minted}

which is used to destroy objects of the type. If not specified
the C++ destructor is used, unless the type is a \verb%pod%
in which case no finaliser is deployed. The finaliser
is called by the collector after an object is determined
to be unreachable, before the underlying storage is reclaimed.

\subsection{{\tt scanner} clause}\index{scanner!clause}
A primitive type may have a \verb%scanner% clause with  string
argument which must specify the name of a C++ function with type
\verb%::flx::gc::generic::scanner_t% given by the typedef

\begin{minted}{c}
typedef void *scanner_t(
  collector_t*, gc_shape_t *, void *, size_t, int
);
\end{minted}

This is a custom function respondible for finding all the 
pointers in a data type.

\subsection{{\tt encoder} clause}\index{encoder!clause}\index{serialisation!encoder}
A primitive type may have an \verb%encoder% clause with  string
argument which must specify the name of a C++ function with type
\verb%::flx::gc::generic::encoder_t_t% given by the typedef

\begin{minted}{c}
typedef ::std::string encoder_t (void *);
\end{minted}

This is a custom function respondible for encoding the type
in a string for serialisation.

\subsection{{\tt decoder} clause}\index{decoder!clause}\index{serialisation!decoder}
A primitive type may have an \verb%decoder% clause with  string
argument which must specify the name of a C++ function with type
\verb%::flx::gc::generic::decoder_t% given by the typedef

\begin{minted}{c}
typedef ::std::size_t decoder_t(void *, char *, ::std::size_t);
\end{minted}

This is a custom function respondible for decoding a string encoded
by the corresponding encoder and building an object of the specified
type on the heap.

\section{Expression bindings}\index{const!statement}
Felix has a simple method for introducing constants and expressions
from C++ using the \verb%const% statement:

\begin{minted}{felix}
type metres = "double";
const zero_metres: metres = "0.0";
const one_metre: metres = "1.0";
\end{minted}

Const statements cannot be polymorphic.

\section{Function bindings}\index{function!binding}
In order to use types lifted from C++, Felix has a special
form of function definition:

\begin{minted}{felix}
type metres = "double";
const zero_metres: metres = "0.0";
const one_metre: metres = "1.0";
fun +: metres * metres -> metres = "$1+$2";
fun *: metres * double-> metres = "$1*$2";
fun *: double * metres -> metres = "$1*$2";
fun todouble: metres -> double = "$1";
var len = one_metre * 3.7;
var len2 = len + len;
println$ "length " + len2.todouble.str;
\end{minted}

Here the encodings \verb%$1% and \verb%$2% refer to the first
and second components of the argument tuple type.

Felix instantiates such functions by substituting the C++
encoding of the argument component into the text string of the function
at the position nominated location, possibly adding parentheses
to ensure precedence rules do not interfere with the intended
semantics.

There is a special set of strings for which Felix knows
the precedence. You must use exactly these strings to trigger
the rules. It is also possible to manually specify the precedence.
If Felix does not understand the precedence then parentheses
will be inserted. The purpose of the precedence controls is 
entirely to reduce the need for the generated C++ to be 
be littered with parentheses.

[Put table of special strings here]
[Put list of precedence names here]

It is sometimes necessary to refer to other entities
than the argument components.

[Put table of binding encodings here]

\section{Conversion bindings}\index{conversion!binding}\index{conversion!operator}
A special form is allowed for conversion function:

\begin{minted}{felix}
type metres = "double";
ctor metres: double = "$1";
ctor double: metres = "$1";
\end{minted}

This is exactly the same as:

\begin{minted}{felix}
fun _ctor_metres: double -> metres = "$1";
fun _ctor_double: metres -> double = "$1";
\end{minted}

so that the syntax makes it easier to see a conversion
is being defined and what is being converted.

Conversion statements can be polymorphic, but the type
variables must be introduced like this:

\begin{minted}{felix}
type vect[T] = "::std::vector<?1>" 
  requires header '#include <vector>';
fun len[T]: vect[T]->size = "$1.size()";
ctor[T] vect[T]: size = "::std::vector<?1>($1)";
var x = vect[int] 42uz;
println$ x.len;
\end{minted}

When Felix is seeking a function, if it finds a type name
instead, it retries the lookup with \verb%_ctor_% prefixed
to the type name. This only works with actual names! 
However the name may be the defined name of a type or
a typedef alias.

Despite the spelling, a \verb%ctor% is a conversion operator
and not a type constructor.

\section{Procedure bindings}\index{procedure!binding}
Procedure bindings work the same as function bindings,
by emitting the C++ string specified with substitutions.

The primary difference is that procedure bindings must
generate one or more C++ statements.

\section{{\tt cstruct} bindings}\index{struct!binding}
A \verb%cstruct% definition has the same syntax as
a \verb%struct% and provides the same facilities,
except that the compiler does not actually emit a C++
structure definition.

Instead the statement is used to lift an existing C
type into Felix. It can be used for structs or unions.

The compiler provides projection and pointer projections
and a memberwise constructor.

{\em Be warned against using the memberwise constructor
if the definition is not complete, for example for 
certain standard C and Posix types which are incompletely
specified.}

[It is not currently possible to turn this off]
[It currently is only possible to bind to names in the global namespace]

\section{{\tt cenum} bindings}\index{enum!binding}
Blah.

\section{{\tt cflags} bindings}\index{cflags!binding}
Blah.

\section{Floating insertions}\index{insertion!floating}\index{floating insertion!textbf}
Felix has a sophisticated system of {\em floating insertions}
which are primarily designed to ensure that if a C++ binding
is used, the appropriate C++ header files are included so
the binding code is well defined. For example:

\begin{minted}{felix}
type vect[T] = "::std::vector<?1>" 
  requires header '#include <vector>';
\end{minted}

Floating insertions can be directly introduced as
indicated for inclusion in either the C++ \verb%header%
file or \verb%body% file Felix generates.\index{header!floating insertion}
\index{body!floating insertion}

Floating insertions {\em float} to near the top of the
relevant file. Insertions specified in a single file
appear in order of writing, if they appear at all,
and at most once. The actual insertion strings are
checked to prevent duplication.

Insertions can be named:\index{floating insertion!named}

\begin{minted}{felix}
header vector_h ='#include <vector>';
type vect[T] = "::std::vector<?1>" 
  requires vector_h
;
\end{minted}

so that a requires clause\index{requires!clause} may specify the name
of an insertion. Duplication of names is permitted:

\begin{minted}{felix}
header vector_h ='#include <vector>';
header vector_h ='#include <stdlib.h>';
type vect[T] = "::std::vector<?1>" 
  requires vector_h
;
\end{minted}

A floating insertion is included if, and only if,
a C binding is actually used.

Floating insertions themselves may have dependencies
expressed with requires clause:

\begin{minted}{felix}
header algo_h ='#include <algorithm>';
header vector_h ='#include <vector'
  requires algo_h
;
\end{minted}

Insertions can also be polymorphic.\index{floating insertion!polymorphic}

\begin{minted}{felix}
body sho[T] ='show<?1>();';
\end{minted}

A module or non-polymorphic class may contain
a naked requires statement:

\begin{minted}{felix}
class Vect { 
  requires algo_h;
  type vect[T] = 'vector<?1>';
\end{minted}

The effect is to apply the requirement to all C++ bindings
in the class or module. Note: the specified insertion
if only emitted if a C++ binding is used!

\section{Property requirements}\index{requirement!property!textbf}
A requires clause may also specify properties:

\begin{minted}{felix}
  fun f: int -> int = "f($1)" 
    requires property "needs_ptf"
  ;
\end{minted}

This provide a flexible tagging system in addition
to the use of adjectives.

[List of properties here]

\section{Package requirements}\index{requirement!package}\index{package!requirement}
A requires clause may specify a package requirement:

\begin{minted}{felix}
  fun f: int -> int = "f($1)"
    requires packahe "mylib"
  ; 
\end{minted}

Packages are a collection of platform
dependent resources which are configured so
that the abstract requirement is used to locate
the resource descriptors in a database managed by
the \verb%flx_pkgconfig% system,  mapping the abstract
platform independent name to platform dependent names.

Packages can be used to add header files to compiler
C++ output, and to find and link support libraries.
In some cases, they can also help build the package.

The package system requires the user to supply a search
path consisting of a list of directories which are searched
to find the file with the package name and extension \verb%.fpc%.

A typical file looks like this:

\begin{minted}{text}
Name: SDL2
Description: Simple Direct Media Layer 2.0
cflags: -I/usr/local/include/SDL2
includes: '"SDL.h"'
provides_dlib: -L/usr/local/lib -lSDL2
provides_slib: -L/usr/local/lib -lSDL2
\end{minted}

This description allows Felix to compile and link programs
using the SDL2 library without any special switches on the
flx command line.

The standard location for an installed Felix configuration
is in the subdirectory \verb%host/config% on Unix or \verb%host\config%
on Windows.

\chapter{Embedding}
\section{C Callable libraries}
\section{Export directives}\index{export!directive}
The \verb%export% directives make the exported symbol a root
of the symbol graph. 

The functional export and forces it to be place in the
generated code as an \verb%extern "C"% symbol with the
given name:


\begin{minted}{felix}
export fun f of (int) as "myf";
export cfun f of (int) as "myf";
export proc f of (int) as "myf";
export cproc f of (int) as "myf";
\end{minted}


Functions are exported by generating a wrapper around
the Felix function. If the function is exported as \verb%fun%
or \verb%proc% the C function generated requires a pointer
to the thread frame as the first argument,
if the \verb%cfun% or \verb%cproc% forms are used, the wrapper
will not require the thread frame. 

In the latter case, the Felix function must not
require the thread frame.

A type may also be exported:

\begin{minted}{felix}
export type ( mystruct ) as "MyStruct";
\end{minted}

This causes a C typedef to be emitted making 
the name \verb%MyStruct% an alias to the Felix type.
This is useful because Felix types can have unpredictable
mangled names.

The word \verb%export% optionally followed by a string
may also be used as a prefix for any Felix function,
generator, or procedure definition. If the string
is omitted is taken as the symbol name. The effect
is the same as if an export statement has been written.


% =====================================
\part{Library}
\chapter{Core Primitive Types}

\section{Boolean type}\index{bool!type}
The type \verb%bool% which is also called \verb%2% provides the
usual boolean logic values \verb%false% and \verb%true%. 
The name \verb%bool% is actually an alias for type %2% the
unitsum of two cases. The name \verb%false% is an alternate
name of the value \verb%case 0 of 2% and the name \verb%true%
is an alternate name for the value \verb%case 1 of 2%.

See \fullref{Unit sum} for more information.

\section{Integer types}\index{integer!type}
\href{http://felix-lang.org/share/lib/std/scalar/ctypedefs.flx}{Library Reference}

There is a table of the types \fullref{Integer Types}.

\begin{table}
\caption{Felix Integer Types\label{Integer Types}}
\centering
\begin{tabular}[c]{lll}
\hline
Felix&C&Suffix\\
\hline
\multicolumn{3}{c}{Standard signed integers}\\
\hline
\verb%tiny% &\verb%char% & t \\
\verb%short% &\verb%short% & s \\
\verb%int% &\verb%int% &  \\
\verb%long% &\verb%long% & l \\
\verb%vlong% &\verb%long long% & ll \\
\multicolumn{3}{c}{Standard unsigned integers}\\
\hline
\verb%utiny% &\verb%unsigned char% & ut \\
\verb%ushort% &\verb%unsigned short% & us \\
\verb%uint% &\verb%unsigned int% & u \\
\verb%ulong% &\verb%unsigned long% & ul \\
\verb%uvlong% &\verb%unsigned long long% & ull \\
\multicolumn{3}{c}{Exact signed integers}\\
\hline
\verb%int8% &\verb%int8_t% & i8 \\
\verb%int16% &\verb%int16_t% & i16 \\
\verb%int32% &\verb%int32_t% & i32 \\
\verb%int64% &\verb%int64_t% & i64 \\
\multicolumn{3}{c}{Exact unsigned integers}\\
\hline
\verb%uint8% &\verb%uint8_t% & u8 \\
\verb%uint16% &\verb%uint16_t% & u16 \\
\verb%uint32% &\verb%uint32_t% & u32 \\
\verb%uint64% &\verb%uint64_t% & u64 \\
\multicolumn{3}{c}{Weird ones}\\
\hline
\verb%size% &\verb%size_t% & uz \\
\verb%intptr% &\verb%uintptr_t% & p \\
\verb%uintptr% &\verb%uintptr_t% & up \\
\verb%ptrdiff% &\verb%ptrdiff_t% & d \\
\verb%uptrdiff% &\verb%ptrdiff_t% & ud \\
\verb%intmax% &\verb%intmax_t% & j\\
\verb%uintmax% &\verb%uintmax_t% & uj \\
\multicolumn{3}{c}{Addressing}\\
\hline
\verb%address% &\verb%void*% &\\
\verb%byte% &\verb%unsigned char% &\\
\end{tabular}
\end{table}


Note that all these types are distinct unlike C and C++.
The types designated are not the complete set of available
integer like types since not all have literal representations.

Signed integers are expected to be two's complement with one
more negative value that positive value. Bitwise and,
or, exclusive or, and complement operations do not apply
with signed types.

The effect of overflow on signed types is unspecified.

Unsigned types use the standard representation. 
Bitwise operations may be applied to unsigned types.
Basic arithmetic operations on unsigned types are
all well defined as the result of the operation
mathematically modulo the maximum value of the type
plus one.

The maximum value of an unsigned type is one less than
two raised to the power of the number of bits in the type.
The number of bits is 8, 16, 32, or 64 or 128 for all unsigned types.

There is a table of the operators \fullref{Integer Operators}.

\begin{table}
\caption{Integer Operators\label{Integer Operators}}\index{integer!operator}
\centering
\begin{tabular}[c]{llll}
\hline
symbol&kind&type&semantics\\
\hline
\multicolumn{4}{c}{All Integers}\\
\hline
\verb%==%&infix-nassoc&\verb%T * T -> bool%&equality\\
\verb%!=%&infix-nassoc&\verb%T * T -> bool%&inequality\\
\verb%<%&infix-nassoc&\verb%T * T -> bool%&less\\
\verb%<=%&infix-nassoc&\verb%T * T -> bool%&less or equal\\
\verb%>%&infix-nassoc&\verb%T * T -> bool%&greater\\
\verb%>=%&infix-nassoc&\verb%T * T -> bool%&greater or equal\\
\verb%+%&infix-lassoc&\verb%T * T -> T%&addition\\
\verb%-%&infix-lassoc&\verb%T * T -> T%&subtraction\\
\verb%*%&infix-lassoc&\verb%T * T -> T%&multiplication\\
\verb%/%&infix-lassoc&\verb%T * T -> T%&quotient\\
\verb+%+&infix-lassoc&\verb%T * T -> T%&remainder\\
\verb%<<%&infix-lassoc&\verb%T * T -> T%&multiplication by power of 2\\
\verb%>>%&infix-lassoc&\verb%T * T -> T%&division by power of 2\\
\verb%-%&prefix&\verb%T -> T%&negation\\
\verb%+%&prefix&\verb%T -> T%&no op\\
\hline
\verb%succ%&func&\verb%T -> T%&successor\\
\verb%pred%&func&\verb%T -> T%&predecessor\\
\multicolumn{4}{c}{Signed Integers}\\
\hline
\verb%sgn%&func&\verb%T -> T%&sign\\
\verb%abs%&func&\verb%T -> T%&absolute value\\
\multicolumn{4}{c}{Unsigned Integers}\\
\hline
\verb%\&%&infix-lassoc&\verb%T * T -> T%&bitwise and\\
\verb%\|%&infix-lassoc&\verb%T * T -> T%&bitwise or\\
\verb%\^%&infix-lassoc&\verb%T * T -> T%&bitwise exclusive or\\
\verb%~%&prefix&\verb%T * T -> T%&bitwise complement\\
\hline
\multicolumn{4}{l}{nassoc: non-associative}\\
\multicolumn{4}{l}{lassoc: left associative}\\
\multicolumn{4}{l}{func: function}\\
\multicolumn{4}{l}{note prefix {\tt -} maps to function {\tt neg}}\\
\hline
\end{tabular}
\end{table}

\subsection{Classification of integers}
\href{http://felix-lang.org/share/lib/std/scalar/ctypedefs.flx}{Library Reference}

Integer types can be grouped into sets. In Felix this can be
done with a \verb%typeset% construction. The defined typesets
are show in \fullref{Integer Typesets}. 

Fast integers correspond to the usual C distinct integer types.

Exact integers are aliases to fast integers of specified sizes,
in Felix these are distinct types.

The weird integers are special purposes aliases to fast integers
which have special uses, again in Felix these are distinct types.
\begin{itemize}
\item The type \verb%ptrdiff% is a signed type intended to represent
the difference between two pointers. 
\item The \verb%size% type is
for measuring the sizes of objects and number of elements
in an array.
\item  The \verb%intmax% and \verb%uintmax% types are
the largest available signed and unsigned integer types,
respectively. 
\item Finally the \verb%intptr% and \verb%uintptr%
types are the same size as a pointer and can be used for
low level numerical operations on pointers either by 
a conversion or reinterpret cast. They're typically used
for to pack extra data into the low bits of a pointer
to a type with an alignment which ensures the low bits
of the pointer are zero.
\end{itemize}

\begin{table}
\caption{Integer Typesets\label{Integer Typesets}}\index{integer!typeset}
\centering
\begin{tabular}[c]{ll}
\hline
fast\_sints &(tiny, short, int, long, vlong)\\
exact\_sints &(int8,int16,int32,int64)\\
fast\_uints &(utiny, ushort, uint, ulong,uvlong)\\
exact\_uints &(uint8,uint16,uint32,uint64)\\
weird\_sints &(ptrdiff, ssize, intmax, intptr)\\
weird\_uints &(size, uintmax, uintptr)\\
sints &fast\_sints $\cup$ exact\_sints $\cup$ weird\_sints\\
uints & fast\_uints $\cup$ exact\_uints $\cup$ weird\_uints\\
fast\_ints & fast\_sints $\cup$ fast\_uints\\
exact\_ints & exact\_sints $\cup$ exact\_uints\\
ints & sints $\cup$ uints\\
\hline
\end{tabular}
\end{table}

Typesets can be used to simplify specification of C function bindings
by leveraging constrained polymorphism.

\begin{minted}{felix}
fun add[T in ints]: T * T -> T = "$1+$2";

// or equivalently
fun add[T : ints]: T * T -> T = "$1+$2";
\end{minted}

A special form which implies a quantifer for every use
also exists,  by specifying a \verb%!% followed by
a typeset name. For example,

\begin{minted}{felix}
fun intsum: !ints * !ints -> long = "(long)($1+$2)";

// or equivalently
fun intsum[T:ints, U:ints] : T * U -> long = "(long)($1+$2)";
\end{minted}

Note \verb%intsum% can add any different kinds of integers,
returning a \verb%long%.

Polymorphism with typeset constraints is {\em only} 
useful for C bindings because the constraints are not
propagated. It simply saves writing out all the finite
combinations when genericity in the C or C++ target
code means the C code would be the same in each case.

\section{Floating point types}\index{floating point!type}
\href{http://felix-lang.org/share/lib/std/scalar/float\_math.flx}{Library Reference}

There is a table of the operators \fullref{Floating Operators},
we also have \fullref{Trig functions}.

Felix also provides \verb%FINFINITY%, \verb%DINFINITY% and \verb%LINFINITY%
for positive infinity representation for \verb%float%, \verb%double%
and \verb%ldouble% types, respectively.

\begin{table}
\caption{Trig functions\label{Trig functions}}
\centering
\begin{tabular}[c]{lll}
\hline
symbol&type&semantics\\
\hline
\multicolumn{3}{c}{circular}\\
\verb%sin%&\verb%T -> T%&sine\\
\verb%cos%&\verb%T -> T%&cosine \\
\verb%tan%&\verb%T -> T%&tangent\\
\verb%asin%&\verb%T -> T%&arc (inverse) sine\\
\verb%acos%&\verb%T -> T%&arc (inverse) cosine\\
\verb%atan%&\verb%T -> T%&arc (inverse) tangent\\
\multicolumn{3}{c}{hyperbolic}\\
\hline
\verb%sinh%&\verb%T -> T%&hyperbolic sine\\
\verb%cosh%&\verb%T -> T%&hyperbolc cosine\\
\verb%tanh%&\verb%T -> T%&hyperbolic tangent\\
\verb%asinh%&\verb%T -> T%&arc (inverse) hyperbolic sine\\
\verb%acosh%&\verb%T -> T%&arc (inverse) hyperbolic cosine\\
\verb%atanh%&\verb%T -> T%&arc (inverse) hyperbolic tangent\\
\hline
\multicolumn{3}{l}{Note: Inverses return primary branch}\\
\hline
\end{tabular}
\end{table}


\begin{table}
\caption{Floating Point Operators\label{Floating Operators}}
\centering
\begin{tabular}[c]{llll}
\hline
symbol&kind&type&semantics\\
\hline
\hline
\verb%==%&infix-nassoc&\verb%T * T -> bool%&equality\\
\verb%!=%&infix-nassoc&\verb%T * T -> bool%&inequality\\
\hline
\verb%<%&infix-nassoc&\verb%T * T -> bool%&less\\
\verb%<=%&infix-nassoc&\verb%T * T -> bool%&less or equal\\
\verb%>%&infix-nassoc&\verb%T * T -> bool%&greater\\
\verb%>=%&infix-nassoc&\verb%T * T -> bool%&greater or equal\\
\hline
\verb%+%&infix-nassoc&\verb%T * T -> T%&addition\\
\hline
\verb%-%&infix-nassoc&\verb%T * T -> T%&subtraction\\
\hline
\verb%*%&infix-nassoc&\verb%T * T -> T%&multiplication\\
\hline
\verb%/%&infix-nassoc&\verb%T * T -> T%&quotient\\
\hline
\verb%-%&prefix&\verb%T -> T%&negation\\
\hline
\verb%abs%&func&\verb%T->T%&absolute value\\
\verb%log10%&func&\verb%T->T%&base 10 logarithm\\
\verb%sqrt%&func&\verb%T->T%&square root\\
\verb%ceil%&func&\verb%T->T%&ceiling\\
\verb%floor%&func&\verb%T->T%&floor\\
\verb%trunc%&func&\verb%T->T%&truncate\\
\verb%embed%&func&\verb%int->T%&embedding\\
\hline
%\verb%exp&func&\verb%T -> T%&natural exponential\\
%\verb%log&func&\verb%T -> T%&natural logarithm\\
%\verb%pow&func&\verb%T * T -> T%&power\\
\hline
\multicolumn{4}{l}{nassoc: non-associative}\\
\multicolumn{4}{l}{lassoc: left associative}\\
\multicolumn{4}{l}{func: function}\\
\multicolumn{4}{l}{note prefix {\tt -} maps to function {\tt neg}}\\
\hline
\end{tabular}
\end{table}

\section{Complex types}\index{complex!type}\index{fcomplex}\index{dcomplex}\index{lcomplex}

There are three complex types, \verb%fcomplex%, \verb%dcomplex%
and \verb%lcomplex% corresponding to cartesian representation
using a pair of \verb%float%, \verb%double% and \verb%ldouble%
values, respectively.

There is a table of the operators \fullref{Complex Operators},
we also have \fullref{Trig functions}.

\begin{table}
\caption{Complex Operators}
\label{Complex Operators}
\centering
\begin{tabular}[c]{llll}
\hline
symbol&kind&type&semantics\\
\hline
\hline
\verb%fcomplex%&ctor&\verb%float * float -> fcomplex%&cartesian\\
\verb%fcomplex%&ctor&\verb%float -> fcomplex%&from real\\
\verb%dcomplex%&ctor&\verb%double * double -> dcomplex%&cartesian\\
\verb%dcomplex%&ctor&\verb%double -> dcomplex%&from real\\
\verb%lcomplex%&ctor&\verb%ldouble * ldouble -> lcomplex%&cartesian\\
\verb%lcomplex%&ctor&\verb%ldouble -> lcomplex%&from real\\
\verb%==%&infix-nassoc&\verb%T * T -> bool%&equality\\
\verb%!=%&infix-nassoc&\verb%T * T -> bool%&inequality\\
\hline
\verb%+%&infix-lassoc&\verb%T * T -> T%&addition\\
\hline
\verb%-%&infix-lassoc&\verb%T * T -> T%&subtraction\\
\hline
\verb%*%&infix-lassoc&\verb%T * T -> T%&multiplication\\
\hline
\verb%/%&infix-lassoc&\verb%T * T -> T%&quotient\\
\hline
\verb%-%&prefix&\verb%T -> T%&negation\\
\hline
\verb%real%&func&\verb%T->R%&real part\\
\verb%imag%&func&\verb%T->R%&imaginary part\\
\verb%abs%&func&\verb%T->R%&norm\\
\verb%arg%&func&\verb%T->R%&argument\\
\hline
%\verb%exp&func&\verb%T -> T%&natural exponential\\
%\verb%log&func&\verb%T -> T%&natural logarithm\\
%\verb%pow&func&\verb%T * T -> T%&power\\
\hline
\multicolumn{4}{l}{nassoc: non-associative}\\
\multicolumn{4}{l}{lassoc: left associative}\\
\multicolumn{4}{l}{func: function}\\
\multicolumn{4}{l}{ctor: constructor}\\
\multicolumn{4}{l}{note prefix {\tt -} maps to function {\tt neg}}\\
\hline
\end{tabular}
\end{table}


\section{Quaternion type}\index{quaternion!type}
There is a table of the operators \fullref{Quaternion Operators}.

\begin{table}
\caption{Quaternion Operators}
\label{Quaternion Operators}
\centering
\begin{tabular}[c]{llll}
\hline
symbol&kind&type&semantics\\
\hline
\hline
\verb%==%&infix-nassoc&\verb%T * T -> bool%&equality\\
\verb%!=%&infix-nassoc&\verb%T * T -> bool%&inequality\\
\hline
\verb%+%&infix-lassoc&\verb%T * T -> T%&addition\\
\hline
\verb%-%&infix-lassoc&\verb%T * T -> T%&subtraction\\
\hline
\verb%*%&infix-lassoc&\verb%T * T -> T%&multiplication\\
\hline
\verb%/%&infix-lassoc&\verb%T * T -> T%&quotient\\
\hline
\verb%-%&prefix&\verb%T -> T%&negation\\
\hline
\multicolumn{4}{l}{nassoc: non-associative}\\
\multicolumn{4}{l}{lassoc: left associative}\\
\multicolumn{4}{l}{func: function}\\
\multicolumn{4}{l}{note prefix {\tt -} maps to function {\tt neg}}\\
\hline
\end{tabular}
\end{table}



\section{Char Type}\index{char!type}
Felix has a type \verb%char% corresponding to C's \verb%char%.
There are no literals of type \verb%char%, instead there
is a function, also called \verb%char% to construct a \verb%char%
from a string literal. If the string at least length 1, the first
character is returned, if the string is empty the \verb%NUL% char
\verb%\0x00% is returned.

An overload of the \verb%char% function also accepts an \verb%int%
argument and constructs a \verb%char% with the designated C code value
which must be in the range 0 through 255.

The function \verb%ord% extracts the C code value of a \verb%char%
as an \verb%int% in the range 0 through 255.

We note that C code values were originally used for the ASCII
character set. Strings of \verb%char% may be used to represent
ISO-10646 (Unicode) code points via the UTF-8 encoding method.
ASCII characters in the range 0 through 127 agree with Unicode
interpretation.

\subsection{ASCII classification functions}\index{ascii!classification}
There is a set of \fullref{ASCII classification functions}
used to test the ASCII class of char.

The library also provides character sets represents as
strings, show in \fullref{ASCII charsets}

\begin{table}
\caption{ASCII charsets}
\label{ASCII charsets}
\centering
\begin{tabular}[c]{ll}
\hline
symbol&definition\\
\hline
\verb%upper%&\verb%ABCDEFGHIJKLMNOPQRSTUVWXYZ%\\
\verb%lower%&\verb%abcdefghijklmnopqrstuvwxyz%\\
\verb%letters%&\verb%upper + lower%\\
\verb%digits%&\verb%0123456789%\\
\verb%alphanum%&\verb%letters + digits%\\
\verb%cidcont%&\verb%alphanum+"_"%\\
\verb%flxidcont%&\verb%alphanum+"_-'"%\\
\verb%camlidcont%&\verb%alphanum+"_'"%\\
\verb%numeric%&\verb%digits + ".eEdD_"%\\
\end{tabular}
\end{table}


\begin{table}
\caption{ASCII classification functions}
\label{ASCII classification functions}
\centering
\begin{tabular}[c]{lll}
\hline
symbol&type&semantics\\
\hline
\multicolumn{3}{c}{ISO C functions}\\
\verb%isupper%&\verb%char -> bool%&A-Z\\
\verb%islower%&\verb%char -> bool%&a-z\\
\verb%isalnum%&\verb%char -> bool%&A-Za-z0-9\\
\verb%isalpha%&\verb%char -> bool%&A-Za-z\\
\verb%isdigit%&\verb%char -> bool%&0-9\\
\verb%isxdigit%&\verb%char -> bool%&0-9A-Za-z\\
\verb%iscntrl%&\verb%char -> bool%&0x0-0x1F\\
\verb%isspace%&\verb%char -> bool%&0x20\\
\verb%isblank%&\verb%char -> bool%&0x20,0x09\\
\verb%isprint%&\verb%char -> bool%&0x20-0x7e\\
\verb%ispunct%&\verb%char -> bool%&punctuation\\
\multicolumn{3}{c}{Lexing functions}\\
\hline
\verb%isidstart%&\verb%char -> bool%&First char identifier\\
\verb%iscamlidcont%&\verb%char -> bool%&Ocaml, subsequent chars\\
\verb%iscidcont%&\verb%char -> bool%&C, subsequent chars\\
\verb%isflxidcont%&\verb%char -> bool%&Felix, subsequent chars\\
\verb%isnumeric%&\verb%char -> bool%&\verb%0-9+-.Ee%\\
\verb%isalphanum%&\verb%char -> bool%&A-Za-z0-9\\
\verb%isletter%&\verb%char -> bool%&A-Za-z\\
\verb%issq%&\verb%char -> bool%&\verb%'%\\
\verb%isdq%&\verb%char -> bool%&\verb%"%\\
\verb%isslosh%&\verb%char -> bool%&\verb%\%\\
\verb%isnull%&\verb%char -> bool%&0x0\\
\verb%iseol%&\verb%char -> bool%&\verb%\n% LF: Unix, CR: Windows\\
\hline
\end{tabular}
\end{table}

\chapter{Algebraic Structure of numeric types}\index{algebra!numeric types}
\href{http://felix-lang.org/share/src/packages/algebra.fdoc}{Basic Algebra Package}
\href{http://felix-lang.org/share/src/packages/numbers.fdoc}{Number Package}

The functionality identified in the previous chapter is not
provided on a type by type basis, but instead leverages the
mathematic structure of the types. This is modelled using
Felix type classes.

In this section we display a subset of the definitions from
the library. The real code also defines some other classes
and also a rich set of alternate names for operators utilising
the presence of LaTeX symbols: this allows for pleasant typesetting
but usually doesn't provide essential semantic operations.

\section{Equality}\index{equality!numeric}
We start with class \verb$Eq$ which models equality based
on the notion it is an equivalent relation:

\begin{minted}{felix}
class Eq[t] {
  virtual fun == : t * t -> bool;
  virtual fun != (x:t,y:t):bool => not (x == y);

  axiom reflex(x:t): x == x;
  axiom sym(x:t, y:t): (x == y) == (y == x);
  axiom trans(x:t, y:t, z:t): x == y and y == z implies x == z;
}
\end{minted}

By providing an instance with \verb%==%, inequality \verb%!=% is
automatically provided by the defaulty definition. An instance
can also provide the inequality, as this is sometimes more efficient.

Conformance to the axioms in instances is required, these axioms define the
required semantics.

\section{Total Ordering}\index{ordering!total}
Now we can define a total order:

\begin{minted}{felix}
class Tord[t]{
  inherit Eq[t];
  virtual fun < : t * t -> bool;
  fun >(x:t,y:t):bool => y < x;
  fun <= (x:t,y:t):bool => not (y < x);
  fun >= (x:t,y:t):bool => not (x < y);

  fun max(x:t,y:t):t=> if x < y then y else x endif;
  fun min(x:t,y:t):t => if x < y then x else y endif;

  axiom trans(x:t, y:t, z:t): x < y and y < z implies x < z;
  axiom antisym(x:t, y:t): x < y or y < x or x == y;
  axiom reflex(x:t, y:t): x < y and y <= x implies x == y;
  axiom totality(x:t, y:t): x <= y or y <= x;
}
\end{minted}

To have a total order, we first have to have equality,
this is effected by use of the \verb%inherit% statement.

Next we define the less than function as virtual,
instances must define this. The other functions are
provided automatically.

The semantics are specified by the axioms which define
a total ordering: reflexivity, anti-symmetry, transitivity,
and finally totallity.

\section{Addition}\index{addition!numeric}
Now we define addition in two steps. First we define approximate
addition

\begin{minted}{felix}
class FloatAddgrp[t] {
  inherit Eq[t];
  virtual fun zero: unit -> t;
  virtual fun + : t * t -> t;
  virtual fun neg : t -> t;
  virtual fun prefix_plus : t -> t = "$1";
  virtual fun - (x:t,y:t):t => x + -y;
  virtual proc += (px:&t,y:t) { px <- *px + y; }
  virtual proc -= (px:&t,y:t) { px <- *px - y; }

  reduce id (x:t): x+zero() => x;
  reduce id (x:t): zero()+x => x;
  reduce inv(x:t): x - x => zero();
  reduce inv(x:t): - (-x) => x;
  axiom sym (x:t,y:t): x+y == y+x;

  fun add(x:t,y:t)=> x + y;
  fun plus(x:t)=> +x;
  fun sub(x:t,y:t)=> x - y;
  proc pluseq(px:&t, y:t) {  += (px,y); }
  proc  minuseq(px:&t, y:t) { -= (px,y); }
}
\end{minted}

Most the functions here are virtual without defaults.
For the procedures we provide either virtual defaulted
definitions or just the definitions.

The key requirement here is simply that some number added
to zero returns the original number, as does negating a number
twice. We also require subtracting a number from itself yields
zero.

Normally addition has stronger requirements such as 
associativity, but this class is designed to support
floating point numbers, and floating addition is not associative.

However since integer addition is, we provide another class:

\begin{minted}{felix}
class Addgrp[t] {
  inherit FloatAddgrp[t];
  axiom assoc (x:t,y:t,z:t): (x + y) + z == x + (y + z);
  reduce inv(x:t,y:t): x + y - y => x;
}
\end{minted}

for an additive group, which just throws in the associative law.
The semantic axiom requires cancellation.

Although technically signed integers in Felix do not obey these
semantics due to overflow, the real axiom required cannot be
modelled in Felix. The true axiom for computer signed integers
is that the axioms are obeyed locally, that is, in a compact
set close to zero.

This axiom would say, for example, that there exists some $d$
such that if $|a|<d$ and $|b|<d$ and $|c|<d$ then 
$$(a + b) + c = a + (b + c)$$

where the restriction is there to ensure there is no overflow.
However constructive mathematics cannot model existentials,
and instead must provide an actual formula for calculating $d$.
In this case, given the range of the integers, we can actually
do that, however in general, even if it can be done, it is all
too hard to bother.

Instead we just rely on the programmer understanding that
the laws only apply locally and using calculations which they
believe will not overflow. In practive many calculations
with integers do not overflow, however there are a few very
nasty cases.

\chapter{More core types}
\section{Slices}\index{slice!type}
A basic slice is just a compact subset of the integers.

\subsection{Slice Definition}
\begin{minted}{felix}
union slice[T] =
  | Slice_all
  | Slice_from of T
  | Slice_from_counted of T * T /* second arg is count */
  | Slice_to_incl of T
  | Slice_to_excl of T
  | Slice_range_incl of T * T
  | Slice_range_excl of T * T
  | Slice_one of T
;
\end{minted}

The \verb%_incl% suffix implies the second argument is included,
whilst the \verb%_excl% suffix implies it is excluded.

All slices have a lower bound, if the initial value is not
specified, the slice starts at 0.

\subsection{Syntax}
We provide a special syntax for slices as illustrated.

\begin{minted}{felix}
var i7up = 7..;       // from 7 up
var i7to11 = 7..11;   // 7 upto 11 inclusive
var i7lt12 = 7..<12;  // 7..11
var iall = ..;        // from 0 up
var ito11 = ..11;     // 0..11
var ilt12 = ..<12;    // 0..11
\end{minted}

The syntax with \verb%<99% was stolen from Apple's Swift language.

\subsection{Iterators}\index{iterator!slice}\index{slice!iterator}
The slice type has an iterator method which allows 
you to enumerate the slice from its lower bound
in increasing order. The iterator returns an option
type and can be implicitly used in a for loop.
The following program:

\begin{minted}{felix}

var it = iterator$ 7..8;
for i in 0..3 do
  println$ i,it ();
done\end{minted}

outputs:

\begin{minted}{text}
(0, Some 7)
(1, Some 8)
(2, None)
(3, None)
\end{minted}

\subsection{String slices}\index{slice!string}\index{string!slice}
Slices can be used with strings to extract substrings.

\begin{minted}{felix}
var x = "Hello World";
println$ x.[2..4];
\end{minted}

Before the slice is applied it is clipped to the string
dimensions by a special translation. First, negative values
are considered to be relative to one past the end of the string
going backwards, so the value -1 for example is the last character 
in the string, and -2 is the second last character.

After translation by adding the string length to negative values,
the resulting start and end locations are clipped to the range
\verb%0..<n% where $n$ is the string length: the lower bound
is the maximum of the calculated value and 0, the upper bound,
exclusive, is the minimum of the calculated value and the string
length. If the lower bound is greater than or equal to the upper
bound exclusive the result is an empty string, otherwise the
substring from the lower bound to the upper bound exclusive
is returned.

This sounds complicated but it is what Python does.
Note that the substring slicing operation cannot fail.

[TODO: slices can be applied to arrays as well but this is not
yet implemented]

\subsection{Generalised Slices}\index{slice!generalised}
A generalised slice is an arbitrary sequence of integers
built by adding slices together. They also support
mapping of slices.

\begin{minted}{felix}
union gslice[T] =
  | GSlice of slice[T]
  | GSSList of list[gslice[T]]
  | GSIList of list[T]
  | GSIter of 1 -> opt[T]
  | GSMap of (T -> T) * gslice[T]
;
\end{minted}

A generalised slice can be built from an ordinary slice
using \verb%GSlice%.

It can also be built from a list of integers using \verb%GSIList%,
or a list of generalised slices using \verb%GSSList%.

When an iterator of a gslice is used, it simply iterates
over the underlying slices in order. The resulting stream
of integers need not be either compact or ascending, and
values can be repeated. 

The form GSIter is a high power generalisation which allows
any iterator over integers to form a gslice, without 
calling the iterator yet. So for example using a GSSList
of two GSIter forms, when iterated over, will use the first
iterator until it runs out of values, then move on to the second one.

The GSMap provides a way to take any sequence of integers from
a gslice an translate them using a function.

Generalised slices are not supported by any high level mathematical
concept, they're just a grab bag of convenient tools for generating
sequences of integers.

\subsection{Slice addition}\index{slice!addition}
Slices and generalised slices can be added to return a generalised slice.

\subsection{Generalised String slicing}\index{slice!string}
Strings can be mapped by generalised slices:

\begin{minted}{felix}
var yank = "11/23/16";
var oz = yank.[(3..5)+(0..2)+(6..7)];
var nip = yank.[(6..7)+(2..2)+(0..4)];
println$ yank,oz,nip;
\end{minted}

[TODO: negative indices don't appear to work in
generalised slices]


\section{String Type}\index{string!type|textbf}
\subsection{Equality and total ordering}
\begin{minted}{felix}
  fun == (s:string, c:string): bool
  fun < (s:string, c:string): bool
  fun <= (s:string, c:string): bool
  fun > (s:string, c:string): bool
  fun >= (s:string, c:string): bool
\end{minted}

\subsection{Equality of {\tt string} and {\tt char}}\index{string!equality}
\begin{minted}{felix}
  fun == (s:string, c:char): bool
  fun == (c:char, s:string): bool
  fun != (s:string, c:char): bool
  fun != (c:char, s:string): bool
\end{minted}

\subsection{Append to {\tt string} object}\index{string!append}\index{append!string}
\begin{minted}{felix}
  proc  += : &string * string 
  proc  += : &string * +char
  proc  += : &string * char
\end{minted}

\subsection{Length of {\\ string}}\index{string!length}\index{length!string}
\begin{minted}{felix}
  fun len: string -> size
\end{minted}

\subsection{String concatenation}
\begin{minted}{felix}
  fun + : string * string -> string
  fun + : string * carray[char] -> string
  fun + : string * char -> string
  fun + : char * string -> string
  fun + ( x:string,  y: int):string
\end{minted}

\subsection{Repetition of {\tt string} or {\tt char}}
\begin{minted}{felix}
  fun * : string * int -> string
  fun * : char * int -> string
\end{minted}

\subsection{Application of {\tt string} to {\tt string} or {\tt int} is concatenation}
\begin{minted}{felix}
  fun apply (x:string, y:string):string
  fun apply (x:string, y:int):string
\end{minted}

\subsection{Construct a char from first byte of a {\tt string}}
Returns nul char (code 0) if the string is empty.

\begin{minted}{felix}
  ctor char (x:string)
\end{minted}

\subsection{Constructors for {\tt string}}
\begin{minted}{felix}
  ctor string (c:char)
  ctor string: +char 
  ctor string: +char  * !ints
  fun utf8: int -> string
\end{minted}

\subsection{Substrings}
\begin{minted}{felix}
  fun subscript: string * !ints -> char
  fun copyfrom: string * !ints -> string
  fun copyto: string * !ints -> string
  fun substring: string * !ints * !ints -> string
  fun subscript (x:string, gs:gslice[int]):string
  proc store: &string * !ints * char
\end{minted}

\subsection{Map a string {\tt char} by {\tt char}}\index{string!map}
\begin{minted}{felix}
  fun map (f:char->char) (var x:string): string = {
\end{minted}

\subsection{STL string functions}\index{string!stl}
These come in two flavours: the standard C++ operations
which return \verb%stl_npos% on failure, and a more Felix
like variant which uses an @{option} type.

\begin{minted}{felix}
  const stl_npos: size

  fun stl_find: string * string -> size
  fun stl_find: string * string * size -> size
  fun stl_find: string * +char -> size
  fun stl_find: string * +char * size -> size
  fun stl_find: string * char -> size
  fun stl_find: string * char * size -> size

  fun find (s:string, e:string) : opt[size] 
  fun find (s:string, e:string, i:size) : opt[size]
  fun find (s:string, e:+char) : opt[size]
  fun find (s:string, e:+char, i:size) : opt[size]
  fun find (s:string, e:char) : opt[size]
  fun find (s:string, e:char, i:size) : opt[size]

  fun stl_rfind: string * string -> size
  fun stl_rfind: string * string * size -> size
  fun stl_rfind: string * +char-> size
  fun stl_rfind: string * +char * size
  fun stl_rfind: string * char -> size
  fun stl_rfind: string * char * size -> size

  fun rfind (s:string, e:string) : opt[size]
  fun rfind (s:string, e:string, i:size) : opt[size]
  fun rfind (s:string, e:+char) : opt[size]
  fun rfind (s:string, e:+char, i:size) : opt[size]
  fun rfind (s:string, e:char) : opt[size]
  fun rfind (s:string, e:char, i:size) : opt[size]

  fun stl_find_first_of: string * string -> size
  fun stl_find_first_of: string * string * size -> size
  fun stl_find_first_of: string * +char -> size
  fun stl_find_first_of: string * +char * size -> size
  fun stl_find_first_of: string * char -> size
  fun stl_find_first_of: string * char * size -> size

  fun find_first_of (s:string, e:string) : opt[size]
  fun find_first_of (s:string, e:string, i:size) : opt[size]
  fun find_first_of (s:string, e:+char) : opt[size]
  fun find_first_of (s:string, e:+char, i:size) : opt[size]
  fun find_first_of (s:string, e:char) : opt[size]
  fun find_first_of (s:string, e:char, i:size) : opt[size]

  fun stl_find_first_not_of: string * string -> size
  fun stl_find_first_not_of: string * string * size -> size
  fun stl_find_first_not_of: string * +char -> size
  fun stl_find_first_not_of: string * +char * size -> size
  fun stl_find_first_not_of: string * char -> size
  fun stl_find_first_not_of: string * char * size -> size

  fun find_first_not_of (s:string, e:string) : opt[size]
  fun find_first_not_of (s:string, e:string, i:size) : opt[size]
  fun find_first_not_of (s:string, e:+char) : opt[size]
  fun find_first_not_of (s:string, e:+char, i:size) : opt[size]
  fun find_first_not_of (s:string, e:char) : opt[size]
  fun find_first_not_of (s:string, e:char, i:size) : opt[size]

  fun stl_find_last_of: string * string -> size
  fun stl_find_last_of: string * string * size -> size
  fun stl_find_last_of: string * +char -> size
  fun stl_find_last_of: string * +char * size -> size
  fun stl_find_last_of: string * char -> size
  fun stl_find_last_of: string * char * size -> size

  fun find_last_of (s:string, e:string) : opt[size]
  fun find_last_of (s:string, e:string, i:size) : opt[size]
  fun find_last_of (s:string, e:+char) : opt[size]
  fun find_last_of (s:string, e:+char, i:size) : opt[size]
  fun find_last_of (s:string, e:char) : opt[size]
  fun find_last_of (s:string, e:char, i:size) : opt[size]

  fun stl_find_last_not_of: string * string -> size
  fun stl_find_last_not_of: string * string * size -> size
  fun stl_find_last_not_of: string * +char -> size
  fun stl_find_last_not_of: string * +char * size -> size
  fun stl_find_last_not_of: string * char -> size
  fun stl_find_last_not_of: string * char * size -> size

  fun find_last_not_of (s:string, e:string) : opt[size]
  fun find_last_not_of (s:string, e:string, i:size) : opt[size]
  fun find_last_not_of (s:string, e:+char) : opt[size]
  fun find_last_not_of (s:string, e:+char, i:size) : opt[size]
  fun find_last_not_of (s:string, e:char) : opt[size]
  fun find_last_not_of (s:string, e:char, i:size) : opt[size]
\end{minted}

  
\subsection{Construe {\tt string} as set of {\tt char}}
\begin{minted}{felix}
  instance Set[string,char] {
    fun \in (c:char, s:string) => stl_find (s,c) != stl_npos;
  }
\end{minted}
  
\subsection{Construe {\tt string} as stream of {\tt char}}
\begin{minted}{felix}
  instance Iterable[string, char] {
    gen iterator(var x:string) () = {
      for var i in 0 upto x.len.int - 1 do yield Some (x.[i]); done
      return None[char];
    }
  }
  inherit Streamable[string,char];
\end{minted}

\subsection{Test if a string has given prefix or suffix}
\begin{minted}{felix}
  fun prefix(arg:string,key:string)=
  fun suffix (arg:string,key:string)
  fun startswith (x:string) (e:string) : bool

  // as above: slices are faster
  fun endswith (x:string) (e:string) : bool

  fun startswith (x:string) (e:char) : bool 
  fun endswith (x:string) (e:char) : bool
\end{minted}

\subsection{h2 Trim off specified prefix or suffix or both}
\begin{minted}{felix}
  fun ltrim (x:string) (e:string) : string 
  fun rtrim (x:string) (e:string) : string
  fun trim (x:string) (e:string) : string
\end{minted}

\subsection{Strip characters from left, right, or both end of a string.}
\begin{minted}{felix}
  fun lstrip (x:string, e:string) : string
  fun rstrip (x:string, e:string) : string
  fun strip (x:string, e:string) : string
  fun lstrip (x:string) : string
  fun rstrip (x:string) : string
  fun strip (x:string) : string
\end{minted}

\subsection{Justify string contents}
\begin{minted}{felix}
  fun ljust(x:string, width:int) : string
  fun rjust(x:string, width:int) : string
\end{minted}

\section{Regexps}

\href{http://felix-lang.org/share/lib/grammar/regexps.fsyn}{Syntax}

\href{http://felix-lang.org/share/lib/std/strings/regdef.flx}{Combinators}

\href{http://felix-lang.org/share/lib/std/strings/re2.flx}{Google Re2 Binding}

\chapter{Introspection}

\chapter{Serialisation}\index{serialisation|textbf}
Felix allows certain data structures to be serialised into a string,
so that the data structure can be recovered from the string.

\section{Operation}
\subsection{{\tt encode\_varray}}\index{serialisation!encode}
This function accepts the address of a serialisable object,
and returns a string containing the serialised contents of the object.

\subsection{{\tt deccode\_varray}}\index{serialisation!decode}
This accepts a string previously serialised by \verb%encode_varray%,
rebuilds a copy of the data structure it encoded, and returns the
address of the object.

\subsection{Usage}
The data structure to be serialised must reside entirely on the heap
and satisfy other requirements. A pointer to the object is cast
to type \verb%address%. When deserialising, the address returned
should be cast to a pointer to the same data type used in serialisation.

\begin{minted}{felix}
include "std/felix/serialise";
var px: &list[int] = (1,2,3,4).list.new;
var serialised_x : string = px.addresss.Serialise::encode_varray;
var px_copy = serialised_x.Serialise::decode_varray.&list[int];
println$ *px == *px_copy;
\end{minted}

\subsection{Process Restrictions}
Serialisation functions depend on the machine address of the
Felix run time type information (RTTI) objects which also
control garbage collection.

Therefore serialisation is only guarranteed to work within a single process.
On some architectures with some linkage strategies it may be possible
to save the string to disk, and reloaded it in a different process
running the same program. In particular the core requirement must
be met by the operating system: to load the process at the same
virtual address. If you wish to try this, it is best to statically
link your program since different orders of dynamic linkage may
result in different addresses for shared library objects (DLLs).

If serialisation works across different processes, it should also
work if the serialised data is transmitted to an remote machine
running the same architecture, same operating system, and binary
identical Felix program.

\subsection{Data Restrictions}
All data to be serialised must either be a C data type bound
to Felix with supporting serialisation and deserialisation
functions provided in the binding, or, any Felix combination
of such primitive data types effected by a tuple, array, struct,
union, sum or unit sum type, or pointer type. Recursive
data types are supported. Circular data structures do not
present a problem.

Most Felix primitive types provided in the library are supported,
except for data types derived from Judy arrays.

You cannot serialise function or procedure closures because
these refer to the global thread frame object, which is the root
for most of the rest of the program data: serialisation would then
attempt to capture the whole program. Even if the thread frame
were serialisable, deserialisation would create a copy of it
which is not supported.

Pointers to C functions, however, are serialisable since they're
just addresses. Pointers to an foreign objects remain invariant.
Most simple data structures are serialised by simply using 
a blit operation, namely \verb%memcpy%.

C++ strings are serialisable. Since serialisation produces a C++
string, data structures containing serialised data are also serialisable.

\subsection{How it works}
Serialisation works by copying serialising primitives with user
supplied serialisation functions . These currently must be written
in C++. Non-primitives are serialised by copying the object
and finding subobjects to also serialised using the same
pointer location data the garbage collector uses from the 
types RTTI object.

Serialised data is tagged with the address of the RTTI object
so that the deserialisation function and location of
interior pointers can be found.

\subsection{Future Directions}
A more portable method of locating RTTI than using the
machine address would remove many of the restrictions,
in particular the restriction that the exact same program
must be used for deserialisation as for serialisation,
as well as the restriction that deserialisation should
be in the same processes.

In fact the primary difficulty here is finding a unique universal
representation of a data type. Were such a kind available a search
could be used to map universal type tokens to the RTTI address.

More portable representations of the data may also remove
the restriction that the same processor must be run the processes
with the same underlying binary ABI (so that, for example, an \verb%int%
has the same binary representation on both architectures).


\chapter{Fibres and Schannels}\label{Fibres and Schannels}
\href{http://felix-lang.org/share/src/packages/fibres.fdoc}{Library Package}

Fibres (or f-threads) are coroutines which exchange control with
other fibres by reading and writing synchronous channels (schannels).

\section{{\tt spawn\_fthread} procedure}
The \verb%spawn_fthread% procedure spawns a new fibre (fthread).
The procedure being spawned must have type \verb%unit->void%
also written \verb%1->0%.

The new fibre begins execution immediately and continues
until suspended by either an unmatched I/O request on
an schannel, or a read operation matched by a write.

Fibres are similar to threads, however control exchange is
synchronous, not pre-emptive. A communicating set of fibres
runs in a pre-emptive single thread.

Fibres can also be suspended waiting for asynchronous
events such as socket I/O or a timer.

\section{Schannel types}
There is only one kind of schannel, however the type system 
provides three views:

\begin{description}
\item[ischannel[T]] an input schannel for reading T values,
\item[oschannel[T]] an outputput schannel for writing T values,
\item[ioschannel[T]] an input/output schannel for either reading or writing T values.
\end{description}

Each of these types is also a constructor for the corresponding
type of schannel.

\section{{\tt mk\_ioschannel\_pair[T]} function}
This function creates a single schannel, returning a read end
and a write end in a tuple:

\begin{minted}{felix}
var ins, outs = mk_ioschannel_pair[int]();
\end{minted}

\section{{\tt ioschannel[T]} constructor}
A bidirectional channel can also be created:

\begin{minted}{felix}
var ios = ioschannel[int]();
\end{minted}


\section{{\tt read} generator and {\tt write} procedure}
The read and write operators on schannels are used for 
communication between fibres. 

\begin{itemize}
\item
When a read matches a write,
the reader continues with the data, and the writer is
placed by on the scheduler active list. If there is
more than one writer, the writer chosen is unspecified.

\item If a read has no matching write, the reading fibre joins
a set of fibres waiting on the schannel for a write,
and the scheduler begins execution of one fibre from
its active list.

\item A write will transfer control and data to one reader.
The reader chosen is unspecified. The writer is then
placed on the scheduler active list.

\item If a write has no matching read, the writing fibres joins
a set of fibres waiting for a read, and the scheduler
begins execution of one fibre from its active list.
\end{itemize}

Thus, an schannel is, in effect, a localised scheduler,
queueing either writers or readers until matching operations
are performed.

\section{Cross thread I/O}
Schannel read and write operations are not thread safe.
Schannels should usually not be shared between pthreads
to ensure I/O cannot be performed between pthreads using
schannel I/O. Pthreads can use \verb%pchannels% to 
communicate instead. On the other hand fibres must not
use pchannels to communicate within the same pthread
since this would be an automatic deadlock.


\section{{\tt broadcast} procedure}
This procedure writes to all fibres waiting
on an schannel, then returns. Unlike \verb%write%
it does not block. 

\verb%broadcast% can be used directly from C++. It is intended
to permit custom C++ driver code to communicate with a suspended
Felix system. The driver code must ensure the data sent with
a broadcast remains live whilst target fthreads are processing it.

Felix \verb%broadcast% is invoked by the member function
\verb%external_multi_write% of the object of type \verb%async_sched%
with type shown below, passing the schannel to broadcast over,
and a pointer to the data to broadcast.

\begin{minted}{C++}
void async_sched::external_multi_swrite(
  ::flx::rtl::schannel_t *chan, 
  void *data
)
\end{minted}

Felix can be embedded in C++ code, in particular it can run
in framework loop idle function as a callback.
It runs until there are no active fibres. The \verb%external_multi_write%
is used to schedule more active fibres so the next resumption
via the idle loop callback has work to do.

\subsection{Fibre States}
Fibres may be in one of five states:
\begin{description}
\item[Running] Only one fibre can be running at a time in a given pthread.
\item[Waiting to Run] A fibre is active if it is on the scheduler active list but
not running.
\item[Waiting on Schannel] A fibre may be waiting for either input or to output.
\item[Waiting for Asynchrons Event] Fibres may also wait for asynchronous events.
\end{description}

A fibre which is running or waiting to run is said to be {\em active}.
A fibre which is waiting on an schannel or asynchronous event is 
said to be suspended.

\subsection{Schannel States}
An schannel may be in one of three states:
\begin{description}
\item[Empty] The schannel has no fibres on its wait list.
\item[Waiting for a reader] The fibre has one or more writers
on its wait list.
\item[Waiting for a writer] The fibre has one or more readers
on its wait list.
\end{description}

\subsection{Abbreviated Type names}
Felix provides short hand notation for
input and output schannels as follows:

\begin{tabular}[c]{lll}
\hline
alias&full name&semantics\\
\hline
\verb"%<T"&ischannel[T]&input schannel\\
\verb"%>T"&oschannel[T]&output schannel\\
\verb"%<>T"&ioschannel[T]&bidirectional schannel\\
\end{tabular}

\subsection{Deadlock, Livelock, and Suicide}
Both fibres and schannels are garbage collected. Thus they
either be reachable or unreachable.

If a fibre is active it is a root, and is always 
reachable. If a fibre is waiting for an asynchronous event
it is also always reachable.

An schannel may be also be reachable if it is stored in a variable
in the global thread frame. This is the global data object which is
shared between all pre-emptive threads.

Alternatively it may be reachable from a procedure reachable
from an active fibre.

A fibre may terminate by returning. In this case it is unreachable
and will be collected. This is known as normal termination.

A fibre which can only be reached from an unreachable schannel
will be garbage collected. This is known as suicide and
ensures fibres cannot deadlock: deadlocked fibres suicide
and so aren't deadlocked because they no longer exist.
For example consider:

\begin{minted}{felix}
begin 
  var i1,o1 = ml_ioschannel_pair[int]();
  var i2,o2 = ml_ioschannel_pair[int]();
  spawn_fthread { write (o1,1); println$ read i2; };
  spawn_fthread { write (o2,2); println$ read i1; };
end
\end{minted}

The first fthread suspends on channel 1, waiting to write,
the second fthread suspends on channel 2, waiting to write.
The fthreads are suspended and deadlocked. However the main
fthread is active and now runs, and the begin/end block
returns so its data frame, containg the two schannels,
is no longer reachable. This means the two suspended
fthreads are not reachable either. So the two fibres
have suicided, they are no longer deadlocked because
they no longer exist.

Suicide is not necessarily an error, it is, in fact, a common
way to terminate fthreads. The problem above terminates correctly.

Now, consider the program above without the begin/end block.
In this case, the fthreads will not be collected because
they're reachable from global store (the so-called thread frame).
So that program will not terminate, and is said to be 
{\em live-locked}.

It is a live-lock because the main fthread can read the
channels and thus the suspended fibres and {\em could}
unsuspend them with suitable I/O operations.

It is the responsibility of the holder of an schannel
to ensure other holders I/O requests are satisfied.
If a fibre does not intended to perform I/O on an schannel
it should forget it. In the example the main thread forgets
the schannels by forgetting the begin/end blocks data frame
when it returns.

\subsection{Restrictions on use}
Fibres can only be spawned with a basic \verb%spawn_fthread%
by top level procedures or
procedural desendants, or, from functions or generator
which are inlined (eliminating the function).

This is because procedures spawned with a basic \verb%spawn_fthread%
with a service call which requires the machine stack be empty, since the service
calls are effected by {\em returning} control to the scheduler.

Similarly, the same restrictions apply to schannel I/O.

However these restrictions can be partly overcome with nested
schedulers.

\section{Nested Scheduling}
A nested scheduler of type \verb%fibre_scheduler%
may be created a constructor accepting a unit argument.
It is must be deleted when no longer required by
the procedure \verb%delete_fibre_scheduler%.

Fibres may be added to the scheduler using the
advanced \verb%spawn_fthread% procedure, this procedure
takes a \verb%fibre_scheduler% argument and returns a closure
accepting procedure of type \verb%1->0%. Fibres starting
with the procedure are added to the nested scheduler list,
but are not yet run.

The \verb%run% procedure takes a populated scheduler and
runs the fibres on it.

Fibres on nested schedulers may communicate with each other
using schannel operations.

The run procedure will not return until all the fibres on
the nominated scheduler have completed. The fibres may
not do asynchronous operations, or make service calls
other than schannel I/O operations,
however the scheduler handles schannel I/O properly.

The result of attempting to communicate with fibres on another
scheduler such as the master scheduler are undefined.

\begin{minted}{felix}
fun foo () = 
{
  var chin, chout = mk_ioschannel_pair[int]();
  var sched = fibre_scheduler();
  var res = 1;
  spawn_fthread sched { write (chout,42); };
  spawn_fthread sched { res = read chin; };
  run sched;
  delete_fibre_scheduler sched; 
  return res;
}
println$ foo();
\end{minted}

\section{Synchronous pipelines}
\href{http://felix-lang.org/share/lib/std/control/spipes.flx}{Library Reference}
lah.

A synchronous pipeline is a special case of fibres and channels
in which data flow from one fibre to a series of others in sequence
using a one schannel for each adjacent pair to connect them.

Special constructors are provided to make pipelines which ensure
that only the fibres in the pipeline know the connecting schannels.

This ensures that the pipeline collapses when there is no more
data to processes.

The starting point of a pipeline is known as a source,
the ending point a sink, and the pieces in the middle
are called transducers.

A transducer is a control inverted function. This means that
instead of being called with data, it reads it, and instead
of returning with a result it writes it.

Unlike functions, which form client/server relations,
tranduces are all peers.

\section{Buffers}
Synchronous I/O is unbuffered. This means the sequencing of reads
and writes is critical. If a fibre writes channel A then B, and another
reads B then A, the system is deadlocked and the fibres may suicide.

In order to read a set of inputs in an arbitrary order, buffers can
be added to each input.

A buffer is simply an fibre running an infinite loop which reads from one
channel and writes on another:

\begin{minted}{felix}
var Ain,Aout = mk_ioschannel_pair[int]();
var Bin,Bout = mk_ioschannel_pair[int]();
var Cin,Cout = mk_ioschannel_pair[int]();
var Din,Dout = mk_ioschannel_pair[int]();

proc buffer (r:%<int, w:%>int) () { 
  while true perform write (w,read r); 
}

spawn_fthread { write (Aout, 1); write (Bout,2); };
spawn_fthread { var d = read Din; c = read Cin; println$ c,d; };
spawn_fthread$ buffer (Ain,Cout);
spawn_fthread$ buffer (Bin,Dout);
\end{minted}

Now, as you can see, the reader procedure reads the data from A
in the reverse order to that in which it is written, but
this works because of the buffers. Note that the first fthread
can now proceed as soon as the buffer has read its output,
before the second fthread gets it. In fact the second fthread
gets what it wrote first.

Buffers are mandatory for a feedback.

\chapter{Circuits}
Felix provides a high level general purpose {\em chips and wires} sublanguage
which is designed to modularise use and deployment of safer iteracting
fibre based processing systems.

\section{Chip definition}
A chip is designated like:

\begin{minted}{felix}
open syntax chips;

chip filereader (filename:string)
  connector file
    pin outch: %>char
{
  var data = load filename;
  for ch in data do 
    write (file.outch,ch);
  done
}
\end{minted}

A chip is an ordinary procedure, which may have parameters 
such as \verb%filename% in the example. The parser creates another argument 
for you: designated the \verb%connector%,
and named \verb%file% in the example,
has the type of a record whose fields are the connectors \verb%pin%s,
in this case there is only one pin, \verb%outch%, which has type 
\verb$%>char$ which is an abbreviation for \verb%oschannel[char]%
and is therefore a synchronous output channel of \verb%char%.

Here is an output device which emits the characters of a string instead:

\begin{minted}{felix}
chip stringreader (s:string)
  connector file
    pin outch: %>char
{
  for ch in s do 
    write (file.outch,ch);
  done
}
\end{minted}

Such output devices are called {\em sources} or {\em suppliers}.

Now, we want to parse the data stream into words, so we will
make a transducer chip.

\begin{minted}{felix}
chip parser 
  connector pins
    pin inchar: %<char
    pin outword: %>string
{
    var ch = read inchar;
  while true do
    while isspace ch perform
      ch = read inchar
    ;
    var s = ""+ch;
    while not (isspace ch) do
      s += ch;
      ch = read inchar;
    done
    write (outword,s);
  done 
}
\end{minted}

Notice that if the input is exhausted in the middle of a word,
the final word is not emitted. This transducer will process
and character streams sent to it and then block waiting for
data that is not coming. This is the correct way to design
circuits. The absence is a explicit termination is the mark
of a good design. Circuits should simply collapse when there
is no work to do.

To complete the job, we will provide a simple sink or consumer
chip:

\begin{minted}{felix}
chip printer 
  connector pins
    pin inword: %<string
{
  while true do
    var s = read inword;
    println s;
  end
}
\end{minted}

Now, we have all the devices we need so lets wire them
together:

\begin{minted}{felix}
device src = string_reader "Hello World\n";
circuit
  connect src.outch,parser.inchar
  connect parser.outch, printer.inword
endcircuit
\end{minted}

Thats it!

[document wire clause and making a chip out of chips]

\chapter{Asychronous Events}\index{event!asynchronous|textbf}\index{asynchronous!event|textbf}
Felix uses asynchronous I/O with fibres to provide access to asynchronous
events. Currently two kinds of events are supported: alarm clock timeout events
and socket I/O buffer and connection events. 

Waiting for an event blocks the current fibre, but it does not
block other fibres in the same pthread.

\section{Event Service {\tt demux}}\index{event!demux}
The event delivery service is called \verb%demux% and runs in a
separate premptive thread. It interfaces with the OS native event
delivery services as shown in \fullref{Event Delivery Service Providers}.

\begin{table}
\caption{Event Delivery Service Providers\label{Event Delivery Service Providers}}
\centering
\begin{tabular}[c]{lll}
Service&Availability&Notes\\
\hline
\verb%select%&all&fallback\\
\verb%poll%&Posix&better than select\\
\verb%epoll%&Linux&better than poll\\
\verb%kqueue%&BSD Unix, OSX&better than poll\\
\verb%Solaris iocompletion ports%&Solaris&better than select\\
\verb%Windows iocompletion ports%&Win32&better than select\\
\end{tabular}
\end{table}

Felix automatically selects the best available service.

\section{Alarm Clock}
The alarm clock service allows a fibre to block for a selected
amout of time. A clock must be constructed first.

\begin{minted}{felix}
  fun mk_alarm_clock: 1 -> alarm_clock;

  // delay in seconds
  proc sleep(clock: alarm_clock, delta: double);
\end{minted}

For example to delay 5 seconds use:

\begin{minted}{felix}
  var clock = mk_alarm_clock();
  sleep (clock, 5.0);
\end{minted}

Note again this only causes the current active fibre to sleep.
Clocks are reusable and may delays can be queued by any number
of fibres within the same pthread.

\section{Sockets}
Socket operations are covered more completely in \fullref{Networking}.
Socket operations are synchronous unless shown in \fullref{Asynchronous Socket Operations}.

\begin{table}
\caption{Asynchronous Socket Operations\label{Asynchronous Socket Operations}}
\centering
\begin{tabular}[c]{lll}
Operation&Used By&Provider class\\
\hline
\multicolumn{3}{c}{Platform Specific}\\
\hline
\verb%connect%&Client&\verb%Faio_posix%\\
\verb%accept%&Server&\verb%Faio_posix%\\
\verb%async_rw%&Client and Server&\verb%Faio_posix%\\
\verb%async_read%&Client and Server&\verb%Faio_posix%\\
\verb%async_write%&Client and Server&\verb%Faio_posix%\\
\verb%Connect%&Client&\verb%Faio_win32%\\
\verb%Accept%&Server&\verb%Faio_win32%\\
\verb%WSARecv%&Client and Server&\verb%Faio_win32%\\
\verb%WSASend%&Client and Server&\verb%Faio_win32%\\
\verb%TransmitFile%&Client and Server&\verb%Faio_win32%\\
\multicolumn{3}{c}{Platform Independent}\\
\verb%connect%&Client&\verb%Socket%\\
\verb%accept%&Server&\verb%Socket%\\
\verb%read%&Server&\verb%Socket%\\
\verb%write%&Server&\verb%Socket%\\
\hline

\end{tabular}
\end{table}

\section{How Async I/O works}\index{asynchronous!IO}
Asynchronous I/O is performed as follows.
First an fthread

Unfortunately shutdown and close requests are synchronous
due to a design fault in sockets and TCP/IP protocol.
Although Posix supports asynchronous shutdown, this may cause
underlying socket buffers to be prematurely flushed.
The only way to asychronously shutdown a socket would be to
perform a synchronous shutdown in a dedicated pthread,
defeating the purpose of using fibres.


\chapter{Preemptive Threading}\label{Preemptive Threading}
\href{http://felix-lang.org/share/src/packages/pthreads.fdoc}{Library Package}
Felix provides pre-emptive threading based on OS native threads
called {\em pthreads}.

Felix pthreads are detached, there is no support for OS joinable
threads. The primary synchronisation primitve is the {\em pchannel}.

\section{{\tt spawn\_pthread} procedure}
The \verb%spawn_pthread% procedure spawns a new pthread.
The procedure being spawned must have type \verb%unit->void%
also written \verb%1->0%.



\section{Pchannel types}
There is only one kind of pchannel, however the type system 
provides three views:

\begin{description}
\item[ipchannel[T]] an input pchannel for reading T values,
\item[opchannel[T]] an outputput pchannel for writing T values,
\item[iopchannel[T]] an input/output pchannel for either reading or writing T values.
\end{description}

Each of these types is also a constructor for the corresponding
type of pchannel.

\section{{\tt mk\_iopchannel\_pair[T]} function}
This function creates a single pchannel, returning a read end
and a write end in a tuple:

\begin{minted}{felix}
var inp, outp = mk_iopchannel_pair[int]();
\end{minted}

\section{{\tt iopchannel[T]} constructor}
A bidirectional channel can also be created:

\begin{minted}{felix}
var ios = iopchannel[int]();
\end{minted}


\section{{\tt read} generator and {\tt write} procedure}
The read and write operators on pchannels are used for 
communication between pthreads. 

Pchannels are technically {\em monitors}. There is no buffering.
Unmatched reads block until a matching write occurs on the
same underlying pchannel. Unmatched writes block until a 
matching read occurs. Neither reader nor writer may proceed
until data is exchanged.

Unlike fibres and schannels, pthread can be deadlock by
out of order I/O.

\section{{\tt concurrently} procedure}
\begin{minted}{felix}
open class ForkJoin {
  proc concurrently_by_iterator (var it:1 -> opt[1->0]) 
  proc concurrently[T with Streamable[T,1->0]] (d:T)
}
\end{minted}

Felix provides the procedure \verb%concurrently_by_iterator%
which uses an iterator returning procedures of type \verb%1->0%
to launch pthreads.

A more convenient procedure, \verb%concurrently%, accepts any
data structure with an iterator.

\verb%concurrently% automatically joins the launched threads.

\begin{minted}{felix}
// using an array
concurrently (
  { println$ "Thread1"; },
  { println$ "Thread2"; },
  { println$ "Thread3"; }
);
\end{minted}

\section{Atomics}\index{type!atomic}\index{atomic!type}
Felix atomics utilise the C++11 standard library atomics.
Atomic variables are not managed by the GC. 

\begin{minted}{felix}
open class Atomic
{
  type atomic[T];
  ctor[T] atomic[T]: T;
  proc delete[T] : atomic[T;

  gen load[T] : atomic[T];
  proc store[T] : atomic[T];
  proc store[T] (a:atomic[T]) (v:T);

  // integer types only 
  proc pre_incr[T] : &atomic[T];
  proc pre_decr[T] : &atomic[T];

}
\end{minted}

\section{Mutual Exclusion Lock}
\begin{minted}{felix}
open class Mutex
{
  type mutex;
  ctor mutex: unit;
  proc lock: mutex;
  proc unlock: mutex;
  proc destroy: mutex;
}
\end{minted}

Felix provides a garbage collector aware mutual exclusion lock
\verb%mutex%. It is a timed mutex which periodically checks
for a GC world stop request.

Condition variables are not garbage collected.

\section{Condition Variables}
Felix provides a GC aware condition variable based 
on C++11 condition variables. It uses a timed variant
to periodically check for a GC world stop request.

\begin{minted}{felix}
open class Condition_Variable
{
  type condition_variable;
  ctor condition_variable: unit;
  proc destroy: condition_variable;

  //$ lock/unlock associated mutex
  proc lock : condition_variable;
  proc unlock : condition_variable;

  proc wait: condition_variable;
  proc signal: condition_variable;

  proc broadcast: condition_variable;

  gen timedwait: condition_variable * double -> int;
}
\end{minted}

Note that unlike Posix condition variables, the associated mutex is
built-in to the condition variable.

\section{Bound Queue}
Felix provides thread safe, GC aware, resizeble bound queue.
The queue itself is garbage collected. The queue stores pointers
to heap allocated copies of the queued objects, which are scanned
by the GC.

\begin{minted}{felix}
open class TS_Bound_Queue
{
  ctor[T] ts_bound_queue_t[T]: !ints;
  proc enqueue[T] (Q:ts_bound_queue_t[T])  (elt:T);
  gen dequeue[T] (Q:ts_bound_queue_t[T]): T;
  proc wait[T]: ts_bound_queue_t[T]; // until Q empty
  proc resize[T]: ts_bound_queue_t[T] * !ints;
 
}
\end{minted}

The constructor creates a queue with the specified maximum
capacity or {\em bound}.

The \verb%enqueue% operation may wait until there is 
space in the queue before depositing its payload and returning.

The \verb%dequeue% operation may wait whilst the queue is empty
until it can acquire a payload before returning.

The \verb%wait% procedure blocks until the queue is empty.

The \verb%resize% procedure changes the queue bound.

Note: When threads are enqueing and dequeing from the queue,
there is no defined method for closing down the use of the
queue. Neither \verb%enqueue% nor \verb%dequeue% can fail,
nor can they be cancelled.

\section{Thread Pool}
\begin{minted}{felix}
class ThreadPool
{
  typedef job_t = 1 -> 0;
  fun get_nthreads () => nthreads;

  proc barrier();
  proc start ();
  proc start (n:int);
  proc queue_job (job:job_t);
  proc stop ();
  proc post_barrier();
  proc notify (chan:opchannel[int]) ();
  proc join ();

  proc pfor_segment (first:int, last:int) (lbody: int * int -> 1 -> 0)
  inline proc forloop (lbody: int -> 0) (first:int, last:int) ()
}
\end{minted}

Felix provides a global thread pool which consists of a thread safe job queue
and a set of running Felix pthreads. The number of threads is
determined from the architecture if possible, otherwise set to a reasonable
value such as 8.

The job queue is bound to 1024 jobs.

The \verb%queue_job% procedure accepts a unit procedure of type \verb%1->0%
and adds it to the job queue. It may block whilst the job queue is full.
Queing a job automatically starts up the thread pool.

Threads in the pool dequeue jobs from the queue and run them.
When finished, a thread fetches another job if there is one,
and runs it. If there are no jobs, the thread waits until
a job is available.

The \verb%start% procedure can be used to manually start the thread pool.

The \verb%post_barrier% procedure posts $n$ \verb%barrier% jobs,
where $n$ is the number of threads in the pool.
A barrier job causes a thread to stop acquiring jobs and wait
until all threads are waiting on a barrier.

The \verb%join% procedure posts $n$ barriers, where $n$ is the number
of threads in the pool. This causes each thread to suspend on a barrier
until all threads have suspended. Then all threads may continue
processing jobs.

The \verb%stop% procedure attempts to stop the pool by posting a $n$ special
\verb%ThreadStop% jobs, where $n$ is the number of threads in the pool. 
A \verb%ThreadStop% job causes the thread executing it to suicide.

\subsection{Restrictions}
Jobs submitted to the thread pool should be either be entirely
independent or transiently coupled.  In general, jobs should not
communicate with each other or other threads using condition variables
or pchannels. It is counter productive for jobs to spawn new threads
although it is not disallowed, because the thread pool is designed
to saturate multi-core CPUs (that is, keep all the cores busy).

Transient coupling, however, is permitted. For example if jobs
are used to calculate some value which is then added to a single
variable, a mutex can be used to serialise access to the variable,
because the lock will only be held transiently.


\section{Parallel For Loop}
\subsection{Syntax}
\begin{minted}{felix}
   loop_stmt := 
     "pfor" sname "in" sexpr "upto" sexpr block
   ;
\end{minted}
\subsection{Operation}
The parallel for loop constructs $n$ subjobs which are submitted to the thread pool, 
where $n$ is approximately the number of threads in the thread pool.  Each job
runs a loop over a subrange of the loop range so that all the indices in
the range are covered.

The thread pool is automatically started
if it wasn't running. Note: it is not automatically stopped. You must
manually call \verb%ThreadPool::stop()% if you use a \verb%pfor% statement
otherwise Felix will not terminate.

\subsection{Example}
The \verb%pfor% loop was specifically designed for matrix multiplication.
The following code multiplies $a$ by $b^{T}$ where the latter is the transpose
of $b$, and puts the result in $r$. The matrices are square of dimension $N$.
The inner product is captured in a function and returns a single value.

\begin{minted}{felix}
  pfor i in 0 upto (N - 1) do
  for (var j=0; j<N; ++j;) 
    perform &r . i . j <- inner_product (&a.i, &b.j);
  done
\end{minted}

\subsection{Restrictions}
It is essential the body of the \verb%pfor% loop behaves independently
for each value of the control index, or uses lock free coupling
such as adding to a shared atomic variable.

\verb%pfor% loops must not be nested. It would be counter productive,
but it will also most likely cause a deadlock. 

\part{Platform Support}
\chapter{Platform Modelling}
The Felix language processor is a translator which outputs C++.
Because of this, the translator can produce code intended to run
on one platform on an unrelated platform. For example, Windows
native programs can be generated on Linux, although there may
not be a suitable compiler and library available to produce
a binary, or an emulator to run it.

So for platform related libraries, Felix provides three
kind of code:
\begin{itemize}
\item platform specific code
\item platform independent code
\item platform adaptive code
\end{itemize}

Platform specific code consists of functions which are
specific to the platform.

Platform independent code uses a type class to represent
a common subset of functions with the same interface.
The type parameter is used to represent the platform variable.

Class instances then bind the virtuals of the class for each
supported platform.

By convention the class name ends in \verb%_class%.

Platform adaptive code is the simplest to use, and simply
provides the platform dependent code for the host platform.
The host is specified by macro file which is included
with parse units, and conditional compilation is used
for the selection.

Note that {\em in addition} to the above Felix level
modelling, similar modelling methods can be used
in C++ libraries to which Felix binds. Therefore
a call to a function like \verb%Shell::system()%
can be mapped in Felix in a platform independent way
to the underlying call in the ISO C library.

The client may write platform independent code,
accepting the restrictions of common functionality,
but the code must reside in a class with OS control
parameter which can be passed on to the platform mapping
virtual functions.

Note that Felix with the C++ backend is a cross-cross
compiler, not just a cross compiler! There do exist
C++ compilers that run on Linux but produce Windows binaries.
Felix front end such a compiler, and produce Windows C++
for it.


\chapter{Processes}
Blah.

\chapter{Programs}
\section{Basic Program interface}
\begin{minted}{felix}
class System
{
  const argc:int;
  fun argv:int -> string;
  fun argv_dflt (x:int) (y:string) : string;
  fun args () : list[string];
  gen system (cmd:string) : int => Shell::system(cmd);
  gen exit: int -> any;
  gen abort: 1 -> any;
  const ptf:ptf_t;

  //$ pexit examines the return code from a system call.
  //$ If the code is 0 it exists with 0.
  //$ On Windows:
  //$    if the code is -1, it exits with errno.
  //$    otherwise code 3
  //$ On Unix:
  //$   if the code is non-zero then
  //$     if the callout aborted, return its abort code.
  //$     if the callout died due to a signal, exit with code 2
  //$     otherwise exit with code 3
  //$ In both these cases a non-zero return causes a message
  //$ to be printed on stderr.

  proc pexit(e:int)
  gen get_stdout(x:string) : int * string => Shell::get_stdout x;
}
\end{minted}

\section{Environment Variables}
\begin{minted}{felix}
class Env_class[os]
{
  //$ Separator for filename lists
  virtual fun pathsep: 1 -> string;

  //$ Get the value of a given variable.
  //$ Returns empty string if the variable doesn't exist.
  fun getenv:string -> string =
    "::flx::rtl::strutil::atostr(std::getenv($1.c_str()))"
    requires package "flx_strutil", Cxx_headers::cstdlib;

  //$ Get the value of a given variable.
  //$ Returns specified default if the variable doesn't exist.
  fun getenv(name:string,dflt:string):string=>let result = getenv(name) in 
    if String::len result != 0uz then result else dflt endif
  ;

  fun getenvlist (name:string) : list[string] =>
     split (getenv name, #pathsep)
  ;
}
\end{minted}

\section{Shell callouts}
\begin{minted}{felix}
class Shell_class[OS, process_status_t]
{
  // Quote a single argument.
  // Note: kills Bash wildcard replacement.
  virtual fun quote_arg:string->string;
  fun quote_args (s:list[string]) : string => catmap[string] ' ' quote_arg s;

  // Mainly for Windows we need a way to quote command line strings too.
  virtual fun quote_line_for_system: string->string;

  virtual fun parse: string -> list[string];

  //------------------------------------------------------------
  // system() function

  //$ System command is ISO C and C++ standard.
  gen raw_system: string -> int = "::std::system($1.c_str())"
    requires Cxx_headers::cstdlib
  ;
  //$ basic command with line quoting.
  gen basic_system (cmd: string) :int => 
    cmd.quote_line_for_system.raw_system
  ;

  // string argument
  // responds to FLX_SHELL_ECHO environment variable
  gen system (cmd:string) = {
    if Env::getenv "FLX_SHELL_ECHO" != "" do
      eprintln$ "[system] " + cmd;
    done
    return basic_system cmd;
  }

  // list of string argument
  gen system (args:list[string]) : int =>
    args.quote_args.system
  ;

  gen system[T with Iterable[T,string]] (args:T) : int =
  {
    var lst = Empty[string];
    for arg in args do 
      lst = lst + arg; 
    done
    return system lst;
  }  

  virtual fun quote_line_for_popen: string -> string;

  //$ get_stdout is a synchronous version of popen_in/pclose pair.
  virtual gen raw_get_stdout : string -> int * string;
 
  gen basic_get_stdout (cmd: string) : int * string =>
    cmd.quote_line_for_popen.raw_get_stdout
  ;

  gen get_stdout (cmd:string) : int * string = {
    if Env::getenv "FLX_SHELL_ECHO" != "" do
      eprintln$ "[get_stdout] " + cmd;
    done
    return basic_get_stdout cmd;
  }

  // arbitrary Streamable argument
  gen get_stdout (args:list[string]) : int * string =>
    args.quote_args.get_stdout
  ;
 
  gen get_stdout[T with Iterable[T,string]] (args:T) : int * string;
}
\end{minted}

\section{Signals}

\chapter{File System}

\chapter{Dynamic Linkage}
\section{Basic Library}
\section{Plugins}

\part{Data Structures}
\chapter{Iterators}

\chapter{Lists}
\section{Inductive lists}
\subsection{Description}
Lists are the core data structure for most functional programming languages.
A list is an inductive data type: it is defined by a base term, \verb%Empty[T]%,
and an induction which says if $x$ is a list of $T$, and $v$ is a value of $T$,
then \verb%Cons (v,x)% is a list.

The induction model of construction admits to a recursive model 
of destruction, for example:

\begin{minted}{felix}
fun str[T with Str[T]] (x:list[T]) : string =>  
  match
  | Empty => "Empty"
  | Cons (head, tail) => head.str + ", " + str tail
  endmatch
;
\end{minted}

In fact, Felix defines Snoc lists rather than Cons lists,
using a \verb%Snoc(list,value)% constructor based on a C++ implementation,
instead of a \verb%Cons (value, list)% constructor and then 
provides a translation with a user defined type constructor.
The idea is to get the tail pointer first in the underlying data type,
so that polymorphic visitors can be written without the offsets depending
on the value type T.

\begin{minted}{felix}
open class List 
{
  union list[T] = | Empty | Snoc of list[T] * T;
  fun _match_ctor_Cons[T] : list[T] -> bool = "!!$1"; 
  inline fun _ctor_arg_Cons[T]: list[T] -> T * list[T] = 
    "flx::list::snoc2cons<?1>($1)" 
    requires snoc2cons_h
  ;
  inline fun Cons[T] (h:T, t:list[T]) => Snoc (t,h);
\end{minted}

\subsection{Constructors}
\subsubsection{{\tt copy} function}
Make an entirely new copy of a list.
Primarily a helper.

\begin{minted}{felix}
  fun copy[T] (x:list[T]):list[T];
\end{minted}

\subsubsection{Copy and return last {\tt copy\_last}}
Yet another helper procedure.

\begin{minted}{felix}
  proc copy_last[T] (inp:list[T], out:&list[T], last:&list[T]) {
    out <- rev inp;
    rev_last (out, last);
  }
\end{minted}

\subsubsection{Named constructor for empty list}
\begin{minted}{felix}
  ctor[T] list[T] () => Empty[T];
\end{minted}

\subsubsection{Construct a singleton list}
Does not work if the argument is an array
or option iterator.

\begin{minted}{felix}
  //$ Make a list with one element.
  //$ NOTE: list (1,2) is a list of 2 ints.
  //$ To get a list of one pair use list[int*int] (1,2) instead!
  ctor[T] list[T] (x:T) => Snoc(Empty[T],x);
\end{minted}

\subsubsection{Construct a list from an array}
\begin{minted}{felix}
  //$ Make a list from an array.
  ctor[T,N] list[T] (x:array[T, N]) : list[T];
\end{minted}

\subsubsection{List comprehension.}
Make a list from an option stream.
Named variant.

\begin{minted}{felix}
  //$ List comprehension:
  //$ Make a list from a stream.
  fun list_comprehension[T] (f: (1->opt[T])) = {
\end{minted}

Constructor variant.

\begin{minted}{felix}
  //$ List comprehension:
  //$ Make a list from a stream.
  ctor[T] list[T](f: (1->opt[T])) => list_comprehension f;
\end{miinted}

\subsection{Construe a list as an array value}
\begin{minted}
  //$ Contrue a list as an array value
  instance[T] ArrayValue[list[T],T] {
    //$ Return umber of elements in a list.
    pure fun len (x:list[T]) = {

    //$ get n'th element
    pure fun unsafe_get: list[T] * size -> T =

    //$ Apply a procedure to each element of a list.
    proc iter (_f:T->void) (x:list[T]) {

    //$ Traditional left fold over list (tail rec).
    fun fold_left[U] (_f:U->T->U) (init:U) (x:list[T]):U =

    //$ Right fold over list (not tail rec!).
    fun fold_right[U] (_f:T->U->U) (x:list[T]) (init:U):U =

  }
\end{minted}

\subsection{Destructors}
\subsubsection{Test for empty list {\tt is\_empty}}
\begin{minted}{felix}
  //$ Test if a list is empty.
  pure fun is_empty[T] : list[T] -> 2 =
\end{minted}

\subsubsection{Tail of a list {\tt tail}}
\begin{minted}{felix}
  //$ Tail of a list, abort with match failure if list is empty.
  pure fun tail[T] (x:list[T]) : list[T] = {
\end{minted}

\subsubsection{Head of a list {\tt head}}
\begin{minted}{felix}
  //$ Head of a list, abort with match failure if list is empty.
  pure fun head[T] (x:list[T]) : T = {
\end{minted}

\subsection{Maps}
\subsubsection{Reverse map a list {\tt rev\_map}}
Tail recursive.

\begin{minted}{felix}
  //$ map a list, return mapped list in reverse order (tail rec).
  fun rev_map[T,U] (_f:T->U) (x:list[T]): list[U] = {
\end{minted}

\subsubsection{Map a list {\tt map}}
Tail recursive. Uses \verb%rev_map% and then inplace revseral.
This is safe because we enforce linearity by abstraction.

\begin{minted}{felix}
  //$ map a list (tail-rec).
  //  tail rec due to in-place reversal of result.
  fun map[T,U] (_f:T->U) (x:list[T]): list[U] =
\end{minted}

\subsubsection{Reverse a list {\tt rev}.}
Tail recursive.

\begin{minted}{felix}
  //$ reverse a list (tail rec).
  pure fun rev[T] (x:list[T]):list[T]= {
\end{minted}

\subsubsection{Zip a pair of lists to a list of pairs {\tt zip2}}
Returns a list the length of the shortest argument.

\begin{minted}{felix}
  //$ Zip two lists into a list of pairs.
  //$ Zips to length of shortest list.
  fun zip2[T1,T2] (l1: list[T1]) (l2: list[T2]) : list[T1 * T2] = 
\end{minted}

\subsection{Useful lists}
\subsubsection{A list of integers {\tt range}.}
From {\tt low} to @{high} exclusive with given @{step}.

\begin{minted}{felix}
  //$ Generate an ordered list of ints between low and high with given step.
  //$ Low included, high not included.
  fun range (low:int, high:int, step:int) =
\end{minted}

\subsubsection{Consecutive integers {\tt range}}
\begin{minted}{felix}
  //$ Range with step 1.
  fun range (low:int, high:int) => range(low, high, 1);
\end{minted}

\subsubsection{Non-negative integers to limit {\tt range}}
{\tt num} integers 0 to {\tt num-1}.

\begin{minted}{felix}
  //$ Range from 0 to num (excluded).
  fun range (num:int) => range(0, num, 1);
\end{minted}

\subsection{Operators}
\subsubsection{Concatenate two lists {\tt join}.}
\begin{minted}{felix}
  //$ Concatenate two lists.
  fun join[T] (x:list[T]) (y:list[T]):list[T] =
 
  //$ Concatenate two lists.
  pure fun + [T] (x:list[T], y: list[T]):list[T] => join x y;
\end{minted}

\subsubsection{Cons an element onto a list.}
\begin{minted}{felix}
  //$ Prepend element to head of list.
  pure fun + [T] (x:T, y:list[T]):list[T] => Snoc(y,x);
\end{minted}

\subsubsection{Append an element onto a list.}
O(N) slow.

\begin{minted}{felix}
  //$ Append element to tail of list (slow!).
  noinline fun + [T] (x:list[T], y:T):list[T] => rev$ Snoc (rev x,y);

  //$ Append element to tail of list (slow!).
  proc += [T] (x:&list[T], y:T) { x <- *x + y; }
\end{minted}

\subsubsection{Concatenate a list of lists {\tt cat}}
\begin{minted}{felix}
  //$ Concatenate all the lists in a list of lists.
  noinline fun cat[T] (x:list[list[T]]):list[T] =
\end{minted}

\subsection{Lists and Strings}
\subsubsection{Pack list of strings into a string with separator {\tt cat}}
\begin{minted}{felix}
  //$ Concatenate all the strings in a list with given separator.
  pure fun cat (sep:string) (x:list[string]):string =
\end{minted}

\subsubsection{Map a list to a list of strings and cat with separator {\tt catmap}}
\begin{minted}{felix}
  fun catmap[T] (sep:string) (f:T -> string) (ls: list[T]) =>

  fun strcat[T with Str[T]]  (sep: string) (ls: list[T]) =>

  fun strcat[T with Str[T]]  (ls: list[T]) =>
\end{minted}

 
\subsection{Searching}
\subsubsection{Value membership}
\begin{minted}{felix}
  //$ Return true if one value in a list satisfies the predicate.
  fun mem[T] (eq:T -> bool) (xs:list[T]) : bool =>

  //$ Return true if one value in the list satisfies the relation 
  //$ in the left slot with 
  //$ the given element on the right slot.
  fun mem[T, U] (eq:T * U -> bool) (xs:list[T]) (e:U) : bool =>

  //$ Construe a list as a set, imbuing it with a membership
  //$ test, provided the element type has an equality operator.
  instance[T with Eq[T]] Set[list[T],T] {
    fun \in (x:T, a:list[T]) => mem[T,T] eq of (T * T) a x;
  }
\end{minted}

\subsubsection{Value Find by relation {\tt find}}
Returns option.

\begin{minted}{felix}
  //$ return option of the first element in a list satisfying the predicate.
  fun find[T] (eq:T -> bool) (xs:list[T]) : opt[T] =>


  //$ Return option the first value in the list satisfies the relation 
  //$ in the left slot with 
  //$ the given element on the right slot.
  fun find[T, U] (eq:T * U -> bool) (xs:list[T]) (e:U) : opt[T] =>

  //$ Return a sub list with elements satisfying the given predicate.
  noinline fun filter[T] (P:T -> bool) (x:list[T]) : list[T] =

  //$ Push element onto front of list if there isn't one in the
  //$ list already satisfying the relation.
  fun prepend_unique[T] (eq: T * T -> bool) (x:list[T]) (e:T) : list[T] =>

  //$ Attach element to tail of list if there isn't one in the
  //$ list already satisfying the relation.
  fun insert_unique[T] (eq: T * T -> bool) (x:list[T]) (e:T) : list[T] =>

  //$ Remove all elements from a list satisfying relation.
  fun remove[T] (eq: T * T -> bool) (x:list[T]) (e:T) : list[T] =>

  //$ Attach element to tail of list if there isn't one in the
  //$ list already satisfying the relation (tail-rec).
  noinline fun append_unique[T] (eq: T * T -> bool) (x:list[T]) (e:T) : list[T] = {

  //$ Take the first k elements from a list.
  fun take[T] (k:int) (lst:list[T]) : list[T] =>

  //$ Drop the first k elements from a list.
  fun drop[T] (k:int) (lst:list[T]) : list[T] =>
 
  fun list_eq[T with Eq[T]] (a:list[T], b:list[T]): bool =>
  instance[T with Eq[T]] Eq[list[T]] { 
    fun ==(a:list[T], b:list[T])=> list_eq(a,b); 
  } 
\end{minted}
 
\subsection{Sort}
\begin{minted}{felix}
  //$ Sort a list with given less than operator, which must be
  //$ total order. Uses varray sort (which uses STL sort).
  fun sort[T] (lt:T*T->bool) (x:list[T])=

  //$ Sort a list with default total order.
  //$ Uses varray sort (which uses STL sort).
  fun sort[T with Tord[T]](x:list[T])=> sort lt x;
\end{minted}
    
\subsection{Streaming list}
\begin{minted}{felix}
  instance[T] Iterable[list[T],T] {
  //$ Convert a list to a stream.
    gen iterator (var xs:list[T]) ()
\end{minted}

\subsection{List syntax}

\begin{minted}{felix}
syntax listexpr
{
  //$ List cons, right associative.
  x[sarrow_pri] := x[>sarrow_pri] "!" x[sarrow_pri]; 
}
\end{minted}

\section{Association List}
A list of pairs

\begin{minted}{felix}
open class Assoc_list
{
  typedef assoc_list[A,B] = List::list[A*B];

  // check is the key (left element) of a pair
  // satisfies the predicate
  fun mem[A,B] (eq:A -> bool) (xs:assoc_list[A,B]) : bool =>

  // check is the key (left element) of a pair
  // satisfies the relation to given element 
  fun mem[A,B,T] (eq:A * T -> bool) (xs:assoc_list[A,B]) (e:T) : bool =>

  instance[A,B] Set[assoc_list[A,B], A] {
    fun mem[A,B with Eq[A]] (xs:assoc_list[A,B]) (e:A) : bool => 
  }

  // find optionally the first value whose associate key satisfies 
  // the given predicate
  fun find[A,B] (eq:A -> bool) (xs:assoc_list[A,B]) : opt[B] =>

  // find optionally the first value whose associate key (left slot)
  // satisfies the given relation to the given element (right slot) 
  fun find[A,B,T] (eq:A * T -> bool) (xs:assoc_list[A,B]) (e:T) : opt[B] =>

  fun find[A,B with Eq[A]] (xs:assoc_list[A,B]) (e:A) : opt[B] =>
}
\end{minted}

\section{Purely Functional Random Access List.}

\subsection{Definition}
\begin{minted}{felix}
//$ Purely functional Random Access List.
//$ Based on design from Okasaki, Purely Functional Datastructures.
//$ Transcribed from Hongwei Xi's encoding for ATS2 library.
//$
//$ An ralist provides O(log N) indexed access and amortised
//$ O(1) consing. This is roughly the closest thing to
//$ purely functional array available.

class Ralist
{

  //$ Auxilliary data structure.
  union pt[a] = | N1 of a | N2 of pt[a] * pt[a];

  //$ Type of an ralist.
  union ralist[a] = 
    | RAnil
    | RAevn of ralist[a]
    | RAodd of pt[a] * ralist[a]
  ;
\end{minted}
\subsection{Length}
\begin{minted}{felix}
  //$ Length of an ralist.
  fun ralist_length[a] : ralist[a] -> int =
\end{minted}
\subsection{Cons}
\begin{minted}{felix}
  //$ Cons: new list with extra value at the head.
  fun ralist_cons[a] (x:a, xs: ralist[a]) =>
\end{minted}
\subsection{Check if empty}
\begin{minted}{felix}
  //$ Check for an empty list.
  fun ralist_empty[a]: ralist[a] -> bool  =
\end{minted}
\subsection{Split into head and tail}
\begin{minted}{felix}
  //$ Proedure to split a non-empty ralist
  //$ into a head element and a tail.
  proc ralist_uncons[a] (xs: ralist[a], phd: &a, ptl: &ralist[a])
\end{minted}
\subsection{Pattern Matching Support}
\begin{minted}{felix}
  //$ User define pattern matching support
  fun _match_ctor_Cons[T] (x:ralist[T]) =>not ( ralist_empty x);
  fun _match_ctor_Empty[T] (x:ralist[T]) => ralist_empty x;
  fun _ctor_arg_Cons[T] (x:ralist[T]) : T * ralist[T] =
\end{minted}
\subsection{Head}
\begin{minted}{felix}
  //$ Head element of a non-empty ralist.
  fun ralist_head[a] (xs: ralist[a]) : a =
\end{minted}
\subsection{Tail}
\begin{minted}{felix}
  //$ Tail list of a non-empty ralist.
  fun ralist_tail[a] (xs: ralist[a]) : ralist[a] =
\end{minted}
\subsection{Random Access}
\begin{minted}{felix}
  //$ Random access to an ralist. Unchecked.
  fun ralist_lookup[a] (xs:ralist[a],i:int)=>
\end{minted}
\subsection{Update}
\begin{minted}{felix}
  //$ Return a list with the i'th element replaced by x0.
  //$ Index is unchecked.
  fun ralist_update[a] (xs:ralist[a], i:int, x0:a) =>
\end{minted}
\subsection{Iteration}
\begin{minted}{felix}
  //$ Callback based iteration.
  //$ Apply procedure to each element of the ralist.
  proc ralist_foreach[a] 
  (
    xs: ralist[a],
    f: a -> void
  )
\end{minted}
\subsection{Show}
\begin{minted}{felix}
  //$ Convert ralist to a string.
  instance[a with Str[a]] Str[ralist[a]] 
  {
    fun str (xx: ralist[a]):string = {
  }
\end{minted}


\section{Doubly Linked Mutable List}
A \verb%dlist\_t% is a doubly linked mutable list.
It is suitable for use as non-thread-safe queue.

\begin{minted}{felix}
class DList[T]
{
  typedef dnode_t=
  (
    data: T,
    next: cptr[dnode_t], // possibly NULL
    prev: cptr[dnode_t]  // possibly NULL
  );
  typedef dlist_t = (first:cptr[dnode_t], last:cptr[dnode_t]);
    // invariant: if first is null, so is last!

  ctor dlist_t () => (first=nullptr[dnode_t],last=nullptr[dnode_t]);
\end{minted}

\subsection{Length {\tt len}}

\begin{minted}{felix}
  fun len (x:dlist_t) = {
\end{minted}

\subsection{Inspection}

\begin{minted}{felix}
  fun peek_front (dl:dlist_t) : opt[T] => 
  fun peek_back (dl:dlist_t) : opt[T] => 
\end{minted}

\subsection{Insertion}

\begin{minted}{felix}
  proc push_front (dl:&dlist_t, v:T) { 
  proc push_back (dl:&dlist_t, v:T) {
\end{minted}

\subsection{Deletion }

\begin{minted}{felix}
  gen pop_front (dl:&dlist_t): opt[T] = {
  gen pop_back (dl:&dlist_t): opt[T] = {
\end{minted}

\subsection{Use as a queue}
We can implement enqueue and dequeue at either end, we'll make
enqueue \verb%push_front% and dequeue \verb%pop_back% for no particular reason.

\begin{minted}{felix}
  typedef queue_t = dlist_t;
  proc enqueue (q:&queue_t) (v:T) => push_front (q,v);
  gen dequeue (q:&queue_t) :opt[T] => pop_back q;
  ctor queue_t () => dlist_t ();
\end{minted}

\subsection{Queue iterator}
Fetch everything from a queue.

\begin{minted}{felix}
  gen iterator (q:&queue_t) () => dequeue q;
\end{minted}


\section{S-expressions}
A scheme like data structure.

\begin{minted}{felix}
class S_expr 
{
  union sexpr[T] = Leaf of T | Tree of list[sexpr[T]]; 

  fun fold_left[T,U] (_f:U->T->U) (init:U) (x:sexpr[T]):U =>
  proc iter[T] (_f:T->void) (x:sexpr[T]) {

  fun map[T,U] (_f:T->U) (x:sexpr[T]):sexpr[U] =>

  instance[T with Eq[T]] Set[sexpr[T],T] {
    fun \in (elt:T, x:sexpr[T]) => 
  }
  instance[T with Str[T]] Str[sexpr[T]] {
    noinline fun str(x:sexpr[T])=>
  }

}

\end{minted}

\section{LS-expressions}
A scheme like data structure, similar to sexpr, only in this variant
the tree nodes also have labels.

\begin{minted}{felix}
class LS_expr 
{
  union lsexpr[T,L] = | Leaf of T | Tree of L * list[lsexpr[T,L]]; 

  fun fold_left[T,L,U] (_f:U->T->U) (_g:U->L->U) (init:U) (x:lsexpr[T,L]):U =>

  proc iter[T,L] (_f:T->void) (_g:L->void) (x:lsexpr[T,L]) {
  fun map[T,L,U,V] (_f:T->U) (_g:L->V) (x:lsexpr[T,L]):lsexpr[U,V] =>
  instance[T,L with Str[T], Str[L]] Str[lsexpr[T,L]] {
    noinline fun str(x:lsexpr[T,L])=>
  }

}

\end{minted}

\chapter{Arrays}
\section{Native Arrays}
\section{Varrays}
\section{Darrays}
\section{Sarrays}
\section{C arrays}

\part{Appendices}
\appendix
\chapter{Special Names}
\begin{description}
\item[{\tt \_make\_}typename]
Cast the representation of an abstract type to the abstract type.

\item[{\tt \_repr\_}]
Cast an abstract type to its representation.

\item[{\tt \_map}]
Generic map.

\item[{\tt \_eq}]
Generic equality.

\item[{\tt \_rev}]
Tuple and tuple type reversal. 
Applied after monomorphisation.
Works on arrays too using library function.

\item[{\tt \_map}]
Tuple map and tuple type map. 
Applied after monomorphisation.
Works on arrays too.

\item[{\tt \_rev\_map}]
Reversing tuple map and tuple type map. 
Applied after monomorphisation.
Works on arrays too using library function.

\item[{\tt \_strr}]
Generic conversion to string.
Supports records and tuples.
Delegates primitives and arrays to library function \verb%str%.

\item[{\tt \_tie}]
Converts a pointer to a tuple, to a tuple of pointers to the
components of the original tuple.

\item[$\prod$]
Product of functions. Prefix quantifier over tuple.

\item[$\sum$]
Sum of functions. Prefix quantifier over tuple.

\item[$\langle\cdot\rangle$]
Mediating morphism of product of functions.

\item[$\lbrack\cdot\rbrack$]
Mediating morphism of sum of functions.

\item[{\tt apply}]
User defined application.

\item[{\tt typeof}]
Denotes type of an expression. The expression is not eveluated.


\item[{\tt \_match\_ctor\_}name]
User defined pattern match constructor check.

\item[{\tt \_ctor\_arg\_}name]
User defined pattern match extractor.


\item[{\tt \_ctor\_arg\_}typename]
User defined constructor/conversion function.

\item{{\tt TYPE}}
The kind of all types.

\item{{\tt GENERIC}}
A psuedo kind used to qualifiy a type variable
in a definition which causes the compiler to generate
a fresh definition for each instance type actually
used on demand.

\item{{\tt \_felix\_type\_name}}
Trait reporting bound type name as string.
\begin{minted}{felix}
  _felix_type_name[int * long] // "int * long"
\end{minted}

\item{{\tt \_scheme}}
An application of the \verb%_scheme% psuedo function
to a string causes the macro processor to execute the
string as embedded Scheme code, resulting in an S-expression.
The S-expression is then converted into Felix expression,
in the same way the S-expressions generated by the parser
are converted into expressions.

It can also be used as a procedure name in statement. 

\item{{\tt \_str}}
An application of \verb%_str% to any Felix expression causes
the macro processor to translate the expression into a string
representing the AST terms for the expression as would be
presented in a compiler diagnostic. Useful to examine exactly
how the parser translates a particular Felix expression.

\item{{\tt range\_check}}
Applied to a tuple of three expressions, this psuedo function
is translated by the macro processor into a system range
check expression which checks that the middle term is greater
than or equal to the first term and less than the third term.
If the conditions are not met, an exception will be raised
at run time.

 
\end{description}

\clearpage
\printglossaries
\clearpage
\addcontentsline{toc}{chapter}{Index}
\printindex
\backmatter
\end{document}
