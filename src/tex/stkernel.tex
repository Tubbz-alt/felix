

\documentclass[oneside]{book}
\usepackage{xcolor}
\definecolor{bg}{rgb}{0.95,0.95,0.95}
\definecolor{emphcolor}{rgb}{0.5,0.0,0.0}
\newcommand{\empha}{\bf\color{emphcolor}}
\usepackage{parskip}
\usepackage{minted}
\usepackage{caption}
\usepackage{amsmath}
\usepackage{amssymb}
\usepackage{amscd}
\usemintedstyle{friendly}
\setminted{bgcolor=bg,xleftmargin=15pt}
\usepackage{hyperref}
\hypersetup{pdftex,colorlinks=true,allcolors=blue}
\usepackage{hypcap}
\title{Type Systems}
\author{John Skaller}
\begin{document}
\maketitle
\tableofcontents
\part{Subtyping Kernel}
\chapter{Motivation}
Felix provides a domain specific sublanguage (DSSL) for binding Objective C APIs.
Consider the following simple ObjC class:
\begin{minted}{objc}
@interface SmallClass: NSObject {}
- (int)get1977;
@end

@implementation SmallClass
- (instancetype)init {
	self = [super init];
	return self;
}
- (int)get1977 { return 1977; }
@end
\end{minted}

To lift this code verbatim into Felix we have to bypass the parser
by creating text inclusions. These inclusions are emitted verbatim
inside one or more of the compiler emitted C++ files.

\begin{minted}{felix}
header small_class_interface = c"""
@interface SmallClass: NSObject { }
- (int)get1977;
@end
""";

body small_class_implementation = c"""
@implementation SmallClass
- (instancetype)init {
	self = [super init];
	return self;
}
- (int)get1977 {
	return 1977;
}
- (int)getsum: (int)toadd {
  return 1977 + toadd;
}

@end
""";
\end{minted}

We can now write a binding to lift the API into Felix:

\begin{minted}{felix}
type small_class_instance_t = "void*"  requires 
  small_class_interface, 
  small_class_implementation
;

fun make_small_class_instance:
  1 -> small_class_instance_t 
= 
  "[[SmallClass alloc] init]"
;

fun get1977 : small_class_instance_t -> int = "[$1 get1977]";

var small_class_instance = make_small_class_instance();
var result = get1977 small_class_instance;
println$ "Felix ran objc to get " + result.str;
\end{minted}

However this is the hard way! here's the easy way, using the
objC DSSL:

\begin{minted}{felix}
  objc-bind 
    @interface small_class 
    +(instancetype) alloc;
    -(instancetype) init;
    -(int) get1977;
    -(int) getsum: (int);
    @end
  ;
  println$ "NESTED " + (small_class'alloc.init.getsum' 44).str; 
\end{minted}

ObjC has a construction which specifies a constraint that an object 
respond to at least the nominated set of messages. Here is an example:

\begin{minted}{felix}
objc-bind 
  @protocol hasDescription
   -(NSString)description;
  @end
;
\end{minted}

and another:

\begin{minted}{felix}
objc-bind
  @protocol Cpy
  -(instancetype)cpy;
  -(instancetype)cpywithmsg:(NSString);
  @end
;
\end{minted}

Here is a class interface that uses these protocols:

\begin{minted}{felix}
objc-bind 
 @interface SmallClass<hasDescription, Cpy>
  {
     int x;
     y: int;
  }
   +(SmallClass)alloc;
   -(instancetype)init;
   -(int)get1977;
   -(int)getsum:(int);
   @property int z;
   @property (readonly) q:int;
 @end
;
\end{minted}

I must explain that what you see above is Felix code. In particular,
the protocols above have nothing to do with any Objective C protocols. 

Sure, they look objective C'ish. But actually Felix makes a 
monomorphic nominal primitive type for each protocol and does not
associate it with any methods! Instead, it defines a set of methods
overloaded on the type of the first argument, which is always present,
namely a pointer to an object instance.

\begin{minted}{felix}
type hasDescription = "void*";

fun description: hasDescription -> NSString = "[$1 $2]";
\end{minted}

and

\begin{minted}{felix}
type Cpy = "void*";

fun cpy: Cpy -> Cpy = "[$1 cpy]";
fun cpywithmsg (obj:Cpy) (msg:NSString) => 
  cexpr "[cpy: $1 msg: $2]" (obj,msg) endcexpr
; 
\end{minted}

This means it is possible to now write functions which accept any object
conforming to a protocol, for example given a pointer to a {\tt SmallClass} instance
the following function will work correctly:

\begin{minted}{felix}
fun require_description(x:hasDescription) => (x.description).str;
println$ sc.require_description;
\end{minted}

Now, suppose we have a protocol A, and another B, and we want one
that has all the methods of both:

\begin{minted}{felix}
objc-bind @protocol C<A,B> @end;
\end{minted}

The way Felix handles protocols .. and superclasses .. is universally done
with an opaque type and one more more subtyping coercions. Since ObjC
objects are universally machine pointers, C \verb+void*+ is a suitable
representation. Emitted subtyping coercions look like this:

\begin{minted}{felix}
  supertype C : A = "$1"; // A < C, A -> C
  supertype C : B = "$1"; // B < C, B -> C
\end{minted}

\section{Intersection Types Required}
And now the real crux of it. We want an {\em anonymous} type constructor,
and that is the intersection operator:

\begin{minted}{felix}
  A & B
\end{minted}

Intersection is symmetric and associative, and an empty intersection is
isomorphic to the universal type.  There is a normal form for intersections,
namely a list of types to intersect, none of which are, or contain, directly
or indirectly, any intersection. The intersection is void if any component
is void. 

\section{Subtyping Judgements}
\subsection{Primitives}
Subtyping judgements between primitives 
can be made easily by inspecting the graph of coercions, and finding, or failing to find, 
a path between two types. 

\subsection{Intersections}
With the introduction of intersection types, we need two extra rules: 
\begin{enumerate}
\item a type X is a subtype of A \& B if is a subtype of both A and B, and, 
\item A \& B is a subtype of Y if both A and B are subtypes of Y.
\end{enumerate}

With our power set the rule reads: 
\begin{enumerate}
\item a type X is a subset of $A \cap B$ if is a subset of both A and B, and, 
\item $A \cap B$ is a subset of Y if both A and B are subset of Y.
\end{enumerate}

\subsection{Pair Judgement}
It is worth unravelling the rule for making this judgement:
\begin{minted}{felix}
  A & B < C & D
\end{minted}
There are two ways:
\begin{enumerate}
\item Apply rule 1 then rule 2 twice:
\begin{enumerate}
  \item $A \& B < C \& D $
  \item $A \& B < C$ and $A \& B < D$
  \item $A < C$ and $B < C$ and $A < D$ and $B < D$
\end{enumerate}
\item Apply rule 2 then rule 1 twice
\begin{enumerate}
  \item $A \& B < C \& D$
  \item $A < C \& D$ and $B < C \& D$
  \item $A < C$ and $A < D$ and $B < D$ and $B < D$
\end{enumerate}
\end{enumerate}

Generalising to subtyping $n$ intersections shows the solution can
always be obtained with a quadratic number of primitive subtyping
judgements.

\subsection{Unification}
The unification engine can implement both rules also since the types are 
monomorphic the unification makes no contribution to the most general unifier
the algorithm must return.

\subsection{Application Binding}
When an overload is successful an application term must be bound, at which
time a mismatch between the function parameter type and the argument
type will be discovered. Since unification succeeded or we would not
reach this point, the path corresponding to the subtyping judgment
traces out at least one sequence of coercions which could be applied,
and the binder inserts a type coercion.

Note that these coercion terms are purely type coercions supported 
by an existential proof a composite coercion can be found at a later time.
We also defer the examination of the question: what happens if there
is more than one path tracing out distinct composite coercions?

\subsection{Overloading}
This leaves the machinery missing a description of how the right
methods are found. ObjC methods are C functions so they exist in
global scope in ObjC, so Felix puts them there too. This means
they can always be found, provided they're actually defined of course.

But now, overloading works by finding all the methods with the same
name and using subtyping judgements to find all candidates. If the argument
type is a subtype of the parameter type the function is a candidate.

The algorithm now attempts to find a best fit as follows: the first
candidate is added to the set of best fits. 
\begin{enumerate}
\item Now for each subsequent 
candidate, if its parameter type is a proper subtype of the parameter 
of any best fit function, that function is thrown out of the best fit set, 
and finally the current candidate is added. 

\item If the candidate is a proper supertype of any function
in the best fit set, the candidate is ignored. 

\item Otherwise the candidate is
added to the best fit set because it is either incomparable with all of them,
or equivalent to at least one of them.
\end{enumerate}

When we run out of candidates if the best fit set has only one element,
that element is the best fit. Otherwise if there are at least two elements,
there is at least two good fits neither of which is better than the other.

\section{Why is this fabulous?}

Now the {\em key point} in this design is that the whole of the method
selection system is done by compile time static type analysis,
so roughly if it compiles it must run without any possibility of
a missing method or nasty null pointer arising from the method binding.
Obviously if any objective C methods are called they have to be
correctly bound to maintain the guarrantee.

\section{Ok what's the catch?}
But there is a caveat, as always.

The machinery will not work in the general case
if the user defined set of coercions are not constrained.

It is vital to understand that because subtyping coercions implicit:

{\em they are perfomed silently} and so {\em we cannot allow 
unexpected behaviour to arise} because {\em there is no locally visible
culprit to pin the blame on.}

The rest of this paper therefore is primarily concerned with
establishing, in the abstract, what the constraints on
supertype definitions should be so as to obey the
{\em principle of least surprise}.


\chapter{Kernel Notions}

\section{Set Inclusions}
Let $U$ be some finite set and let $C$ be the power set of $U$, the set of all
subsets of $U$. Then $C$ is a category with an arrow $A\rightarrow B$ if $A \subseteq B$.
We can provide two maps $C \times C \rightarrow C$ called intersection and union 
with the usual binary operators $\cap$ and $\cup$ thereby constructing a lattice.
These maps can be shown to be bifunctors.

An concretised version of this kernel abstraction replaces the subtyping 
judgment by set inclusion arrows with a witness function which provides an actual embedding.

\section{Abstract Subtyping Kernel}
An abstract subtyping kernel is the interface of an implementation.
It has the same basic structure as the set inclusion lattice. 
However the concretised version, which uses actual functions, requires
an strong constraint: these functions must be monic.

\section{Generating Graph}
We will specify the subtyping kernel as the category generated by
a finite graph we present. In Felix notation we will write for example:

\begin{minted}{felix}
  type int = "int";
  type long = "long";
  type A= "A";
  type B= "B";
\end{minted}

to specify the vertices of the graph as primitive monomorphic types. 
The code on the RHS is the representation in C++.

The edges of the graph are specified like:

\begin{minted}{felix}
  supertype long : int = "(long)$1";
  supertype B : A = "(B)$1";
\end{minted}

which specifies the second type is a subtype of the first and
provides an implementation called a coercion. The two categories generated
by this graph consist of the subtyping judgement category and the more
concrete coercion category which consists of all compositions of 
coercions.

\section{Analysis}
It is important to note that no constraint is imposed on the set of 
coercions provided so that it is possible for the graph to be cyclic.
In this case two primitives may be judged equivalent. 

We now do the usual trick: equivalence, as the name suggests, is
an equivalence relation, therefore we can partition the set of
primitives into classes. At the same time, we're going to make
"bundles" of coercion arrows as follows: for each class all the
arrows between primitives of that class, and, for each class A and
each class B, a bundle of all the arrows from any primitive in A to
a primitive in B.

These classes and bundles now strongly resemble the structure of the
inclusion map on the powerset of some set.

Now consider two primitives X and Y of a class. There are any number
of coercions from one to the other, if we are going to identify the class
with a subset in the powerset model, we really need that all paths
from X to Y are equal, and all paths from Y to X. Furthermore,
the two paths that now remain must be inverses of each other so that
the primitives are isomorphic and the paths are isomorphisms.

Anything less than this would allow operations that caused suprises.

Now consider the bundle of arrows from X to Y again. Since every arrow
in X is an isomorphism a map from any primitive of X to some primitive
of Y is automatically a map from every object in X, via an isomorphism.

Furthermore in Y, any map $f:x\rightarrow y$ to some primitive of Y can be extended
to every other primitive $z$ via an isomorphism $h:y\rightarrow z$, and now we have a hard
constraint, that if $g:x\rightarrow z = f \odot h$. This is a commuting square. 
in other words every path from X to Y must be equal, give or take an isomorphism
at each end of the arrow for retargetting.

And so now by construction, the primitive classes and coercion arrow bundles
are a subcategory of the powerset category.

\section{Specifying the constraint}
We have established a necessary condition for a coherent subtyping
category. But the constraints are expressed in terms of equality
of all paths, whereas the client simple encodes edges of the generating
graph.

The basic semantic rule is that each coersion must be an embedding,
in other words, the function must be monic. We start with once coercion,
and consider a second. If the first coercion is from A to B, and the second
from B to A, they must be inverses. If we want to add a coercion from X to Y,
and there is already a path from X to Y, then the new coercion must be equal
to the composite. There is good reason for doing this: optimisation.

Saunders MacLean says an arrow $m: a\rightarrow b$  is monic in a category $C$
if when for any two parallel arrows $f,g: d\rightarrow a$, we have that $f\odot m = g\odot m$
implies $f=g$. Note that we are using reverse composition in OO style, so that above the first 
function is executed first, and the second applied to its result.

\section{Extension to intersections}
The critical reason for insisting on monic arrows arises from the introduction
of intersections. Users do not specify coercions from or two intersections,
they must be generated by the compiler. Since $A \& B$ is a subtype of $A$
a coercion from the intersection to $A$ is clearly just the embedding 
into A or those elements that weren't intersected away by B, but
how do we construct it? The answer is the coercion does nothing so we
do not need to. It is a type cast only.

\end{document}
